% Este arquivo .tex será incluído no arquivo .tex principal. Não é preciso
% declarar nenhum cabeçalho

\section{LKCAMP}

O LKCAMP é um grupo de estudos sobre o kernel Linux.
O kernel Linux é a parte principal de muitos dos sistemas operacionais
utilizados na atualidade, desde distribuições Linux (Ubuntu, Fedora, etc) para
desktops e até mesmo o Android.
O kernel é o núcleo do sistema operacional, sendo responsável por gerenciar o
hardware do dispositivo, além de oferecer uma interface abstraída do hardware
para os programas que estejam rodando, permitindo que eles utilizem os recursos
da máquina (GPU, tela, etc) de forma mais simples.

O objetivo do grupo é inicialmente fornecer os conhecimentos básicos sobre o
funcionamento do kernel Linux por meio de apresentações e atividades práticas.
Nessas atividades, os participantes irão tanto escrever drivers (programas que
gerenciam um dispositivo) quanto enviar patches (contribuições de código) para o
Linux.

Após esse período introdutório, os participantes serão incentivados a mergulhar
mais a fundo em algum dos subsistemas do kernel Linux, a fim de compreendê-lo
melhor e inclusive contribuir para ele.
O grupo oferecerá mentoria e direcionamento aos participantes durante essa
jornada.

Finalmente, espera-se que os participantes contribuam de volta com o grupo,
fazendo apresentações daquilo que aprenderam e realimentando conhecimento de
volta ao grupo.

Os encontros ocorrem semanalmente ao longo do semestre.

\begin{figure}[H]
  \centering
  \includegraphics[width=.24\textwidth]
  {img/alem_da_graduacao/lkcamp_logo.png}
\end{figure}


\begin{compactitemize}
\item Site: \url{lkcamp.dev}
\item Grupo do Telegram: \url{https://t.me/lkcamp}
\item Lista de email: \url{https://lists.libreplanetbr.org/mailman/listinfo/lkcamp}
\end{compactitemize}
