% Este arquivo .tex será incluído no arquivo .tex principal. Não é preciso
% declarar nenhum cabeçalho

\section{Enigma}

O Enigma é uma entidade formada por alunas e alunos da Unicamp com o propósito
de estudar privacidade, segurança e criptografia, além de outros assuntos
relacionados. Inauguramos nossas atividades com um \textit{Capture the Flag} e
uma palestra na Secomp 2018. Desde então, organizamos encontros, mini cursos,
palestras e desafios para incentivar o desenvolvimento desta área na
universidade.

A área de segurança da informação deixou de ser uma disciplina de nicho para
ser um pré-requisito para todo software que envolve conexão com internet e/ou
dados sensíveis. Todo site e programa pode ser vulnerável, e uma das formas de
aprender a se defender, é aprender o ataque. Por isso, em nossas atividades,
exploramos e discutimos vulnerabilidades, bem como meios de mitigá-las. A
privacidade é um direito fundamental do ser humano e que garante um
funcionamento democrático da sociedade. Esse direito tem sido cada vez mais
negligenciado na era digital, com abusos por parte de corporações e governos.
Através da criptografia é possível transmitir e armazenar dados de forma
segura e garantida que somente aqueles com os devidos poderes podem vir a
acessá-los, sendo uma ferramenta essencial tanto para privacidade quanto para a
segurança.

Todas as pessoas são bem vindas, independentemente do seu curso e experiência
prévia! Somos abertos, sem processo seletivo ou hierarquias e acreditamos que
o conhecimento deve ser livre.

\begin{figure}[H]
  \centering
  \includegraphics[width=.24\textwidth]
  {img/alem_da_graduacao/enigma_logo.jpg}
\end{figure}


\begin{compactitemize}
\item Site: \url{enigma.ic.unicamp.br/}
\item Email: \url{enigmaunicamp@tutanota.com}
\item Grupo do Telegram: \url{https://t.me/enigmaunicamp}
\end{compactitemize}
