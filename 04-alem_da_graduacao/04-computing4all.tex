% Este arquivo .tex será incluído no arquivo .tex principal. Não é preciso
% declarar nenhum cabeçalho

\section{Computing 4 All}

O grupo Computing 4 All foi criado para incentivar a diversidade de gênero nos
cursos de computação. Se você está se perguntando o motivo disso ser necessário,
talvez ainda não saiba que há pessoas que deixam de considerar a computação como
uma opção por serem levadas a pensar que essa área requer um perfil bem
específico no qual elas não se encaixam.

Na verdade, a diversidade, seja ela qual for (cultural, religiosa, racial, de
gênero, etc.), traz uma grande riqueza para qualquer área e a computação não é
diferente. Por isso, em setembro de 2015, professoras e alunas do Instituto de
Computação decidiram criar um grupo para trabalhar com assuntos relacionados a
esse tema.

Para alcançar a diversidade desejada, o que era a princípio um grupo de mulheres
na computação tornou-se um grupo para tod*s, com o objetivo de incentivar mais
pessoas, independente do gênero, a se interessarem pela área.

\begin{figure}[H]
    \centering
    \includegraphics[width=.24\textwidth]{img/alem_da_graduacao/computing4all_logo.png}
\end{figure}

O Computing 4 All realizou em 2015 seu primeiro evento com palestras, painéis e,
em parceria com o CACo, um CineCACo, reunindo professor*s, alun*s e empresas
para comemorar os 200 anos de Ada Lovelace, a pessoa que escreveu o primeiro
algoritmo para ser processado por uma máquina.

Como próximas metas, o grupo pretende realizar projetos, eventos e palestras e
também cursos em parceria com outros grupos da Unicamp para alun*s de ensino
médio e fundamental, com o objetivo de aumentar o interesse pela área através da
exploração do potencial da computação em suas várias facetas.

Se gostou da ideia, venha participar!

\begin{itemize}
    \sloppy
	\item Site: \url{computing4all.ic.unicamp.br/}
	\item Lista de e-mails:
          \url{groups.google.com/forum/\#!forum/computing4all}
	\item Facebook: \url{www.facebook.com/Computing4All.Unicamp/}
\end{itemize}