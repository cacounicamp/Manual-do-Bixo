% Este arquivo .tex será incluído no arquivo .tex principal. Não é preciso
% declarar nenhum cabeçalho

\section{Atlética -- AAACEC}

A \textbf{AAACEC -- Associação Atlética Acadêmica da Ciência e Engenharia de
Computação}, ou simplesmente \textbf{Atlética}, é a entidade estudantil que
promove a prática de esportes na Computação.

Uma entidade sem fins lucrativos, a AAACEC tem sua diretoria eleita anualmente
pel*s alun*s associad*s dos cursos de Engenharia e Ciência da Computação e da
pós-graduação no IC.

A Atlética é responsável pela participação da Computação em competições
esportivas, tanto dentro da Unicamp (Calouríadas, Interanos, Olimpíadas) quanto
fora (Intercomp).

\begin{figure}[H]
    \centering
    \includegraphics[scale=0.55]{img/alem_da_graduacao/aaacec_foto.jpg}
\end{figure}

A fim de possibilitar essa participação, a AAACEC promove treinos regulares de
basquete, vôlei, handebol e futsal e disponibiliza o material (bolas, redes
etc.) para a prática de tais esportes. A AAACEC se encarrega da reserva de
quadras para a realização dos treinos e competições em que isso se fizer
necessário. Os treinos são semanais e oferecidos para as modalidades masculina e
feminina, de forma que qualquer associad* da Atlética pode participar.

Para associar-se à AAACEC, *s ingressantes podem comprar o \textbf{Kit Bixo},
que contém produtos co\-mo camiseta, caneca, mouse pad e chaveiro. Veteran*s
podem associar-se mediante o pagamento de uma taxa. Além do Kit Bix*, a AAACEC
vende outros produtos, como agasalhos.

Além de promover a prática esportiva, a Atlética também realiza festas e eventos
de integração, como a tradicional \textbf{Choppada da Computação} no começo do
ano (gratuita para bix*s, não perca!).

Para saber mais sobre a Atlética e como participar, entre em contato por:
\begin{compactitemize}
\item E-mail: \email{aaacec@gmail.com}
\item Site: \url{aaacec.com.br}
\end{compactitemize}
