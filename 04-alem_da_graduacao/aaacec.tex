% Este arquivo .tex será incluído no arquivo .tex principal. Não é preciso
% declarar nenhum cabeçalho

\section{Atlética -- AAACEC}

A \textbf{AAACEC -- Associação Atlética Acadêmica da Ciência e Engenharia de
Computação}, ou simplesmente \textbf{Atlética}, é a entidade estudantil que
promove a prática de esportes na computação.

Uma entidade sem fins lucrativos, a AAACEC tem sua diretoria eleita anualmente
por alunas e alunos associadas dos cursos de engenharia e ciência da
computação, além da pós-graduação no IC.

A Atlética é responsável pela participação da computação em competições
esportivas, tanto dentro da Unicamp (Calouríadas, Interanos, Olimpíadas),
quanto fora (Intercomp).

\begin{figure}[H]
  \centering
  \includegraphics[scale=0.55]{img/alem_da_graduacao/aaacec_foto.jpg}
\end{figure}

A fim de possibilitar essa participação, a AAACEC promove através da Liga CEM
(Computação, Estatística e Matemática) treinos regulares de basquete, vôlei,
handebol e futsal e disponibiliza o material (bolas, redes etc.) para a 
prática de tais esportes. A CEM se encarrega da reserva de quadras para
a realização dos treinos e competições em que isso se fizer necessário. 
Os treinos são semanais e oferecidos para as modalidades masculina e 
feminina, de forma que qualquer aluno da computação pode participar.

Para associar-se à AAACEC, ingressantes podem comprar o \textbf{Kit Bixo},
que contém produtos co\-mo camiseta, caneca, mouse pad e chaveiro. Veteranas
e veteranos podem associar-se mediante o pagamento de uma taxa. Além do Kit
Bixo, a AAACEC vende outros produtos, como agasalhos, camisetas e sambas-
canções, que podem ser adquiridos com desconto por estudantes associados.

Além de promover a prática esportiva, a Atlética também realiza festas e
eventos de integração, como a tradicional \textbf{Bixoppada} no
começo do ano (gratuita para bixetes e bixos, não perca!).

Para saber mais sobre a Atlética e como participar, entre em contato por:
\begin{compactitemize}
\item E-mail: \email{aaacec@gmail.com}
\item Site: \url{aaacec.com.br}
\item Facebook: \url{facebook.com/aaacec}
\end{compactitemize}
