\section{Disciplinas}

\subsection{Matrícula}

Com exceção do primeiro semestre letivo, no qual você já entra matriculade em todas as matérias obrigatórias, \textbf{na Unicamp é você quem vai ter que montar sua grade de disciplinas}. A cada semestre, você deve enviar à DAC um requerimento de matrícula com as disciplinas que deseja cursar no semestre. Mas não é nada complicado, e ninguém vai te deixar esquecer! Você pode consultar as matérias sugeridas por semestre no catálogo de curso, disponibilizado pela DAC, na sua “Proposta para Cumprimento de Currículo”.

Para te auxiliar na parte mais legal do semestre – o planejamento da grade –, existe o GDE (\url{https://grade.daconline.unicamp.br/}): uma ferramenta criada por um aluno da engenharia e adotada pela DAC que disponibiliza um planejador já com as matérias sugeridas para o semestre e as turmas. Também é uma rede social interna, permitindo avaliação de oferecimentos de matérias e de professores, que serve de feedback às outras pessoas.

Recomendamos que peça ajuda a veteranes quando for montar seu horário! Informe-se sobre todes es professores que oferecem as matérias, se são mais tranquiles ou pegam no pé, se dão aula bem ou mal, se demoram para entregar as notas... Você vai poupar muita dor de cabeça! Os melhores lugares para essas discussões são os grupos de Facebook da Unicamp, grupos com veteranes no wpp/telegram e o servidor de \textit{discord Corredores do IC} – por onde centramos nossa comunicação durante o EAD. Dá um alô pra gente que te mandamos o link para entrar!

Porém, nem sempre você vai conseguir exatamente o que quer na matrícula. É possível que te joguem para uma turma em outro dia da semana, caso a turma que você pediu lote. Isso é muito comum e, nesses casos, pode tentar de novo no \textbf{período de alteração de matrícula}, que acontece, normalmente, próximo ao início das aulas. Nesse período, é possível pegar uma matéria que não conseguiu na matrícula normal, permutar a turma de uma matéria que já se matriculou ou até mesmo desistir de alguma delas, sem nenhum tipo de penalidade.

\subsection{Datas e prazos}

Para não ser pegue pelo arrependimento de ir a alguma aula em dia de folga ou de não poder mais trancar uma matéria que considera já perdida, anote as datas importantes todo início de semestre no seu calendário. Você pode encontrar essas datas no calendário da DAC: o site possui um filtro que seleciona o que é de interesse de alunes de graduação, da pós, de docentes e de funcionáries – o que torna sua vida muito mais fácil.

Isso não serve apenas para saber quais são os feriados: serve também para períodos de estudo para o exame – assim como o período para a aplicação da prova –, período para desistência de disciplinas, trancamento de matrícula, férias e matrícula no próximo semestre. Mesmo seguindo o calendário, às vezes há mudanças no calendário no meio do período! Fique atente!

\subsection{Primeiras Disciplinas}

A graduação nos cursos de computação é repleta de disciplinas avançadas que vão saciar as mais diversas vontades, mas não adianta tentar começar pelas coisas difíceis, é por isso que o primeiro semestre é idealizado para ser um momento de aprofundamento em bases importantes. Algumas matérias como Química e Física são um aprofundamento do conhecimento adquirido no ensino médio, já outras servem para apresentar uma base matemática e científica necessária para o curso como Cálculo, Geometria Analítica e Física Experimental (2º Semestre para Ciência da Computação). Mas além dessas matérias temos aquela que vai ser a porta de entrada para a computação para muitas pessoas:

\paragraph{MC102} é a disciplina de Algoritmos e Programação de Computadores, sua primeira disciplina no Instituto de Computação. É nessa matéria que aprendemos a primeira linguagem de programação (Python), bem como algoritmos fundamentais para o curso. Durante o semestre as aulas serão divididas em aulas expositivas teóricas e aulas práticas em laboratório lá no IC 3. É esse o momento de aprender a por a mão na massa, então não hesite em pedir ajuda nos momentos de laboratório a es monitores, mas tome cuidado: plágio e utilização de inteligência artificial vão resultar em reprovação. 