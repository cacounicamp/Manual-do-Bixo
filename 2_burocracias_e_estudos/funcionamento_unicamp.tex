\section{Como funciona a universidade}

\subsection{Instituição}

A universidade é uma autarquia pública - o que significa que é um órgão público do  estado, mas muito mais independente de governos (que trocam a cada 4 anos). Portanto, tem sua própria autonomia para orçamento e contratações, realizar obras, etc.

O que garante essa independência é a chamada Autonomia Universitária, garantida pela Constituição de 1988, mas que deriva da própria origem da "universidade" como
instrumento de avanço da sociedade como um todo. O famoso tripé que coordena (ou
deve coordenar) o trabalho da Universidade é o seguinte:

\textbf{Ensino}: formar cientistas de diversas áreas do conhecimento;

\textbf{Pesquisa}: orientar o trabalho desses profissionais à procura de novo conhecimento;

\textbf{Extensão}: retorno direto da universidade à comunidade que a sustenta, com iniciativas de impacto direto para os cidadãos e toda a sociedade. Hoje, há muitos cursos pagos (e caros) distorcendo a Extensão.

\subsection{Órgãos}

\subsection{Quem escolhe e Reitore?}

A cada 3 anos, há uma votação para escolher e reitore, o que faz muitos acharem que  é uma eleição. Contudo, é apenas uma consulta à comunidade, em que o voto des  professores tem maior peso (70\%) que o voto de funcionáries e alunes. Os nomes des 3 principais candidates mais votades nesta consulta são enviados ao Governador do  Estado, que escolhe oficialmente e reitore.

\subsection{Orçamento}

Pela Constituição, Educação é um direito da sociedade, provida pelo estado, através  dos (caros) impostos. Pela lei, as universidades estaduais paulistas recebem uma fatia do  ICMS, um imposto sobre consumo pago sobretudo por trabalhadores comuns;

Os governos têm calculado o repasse após descontar despesas do ICMS ou ignorar  pagamentos atrasados, diminuindo o bolo investido nas universidades. Um aumento  para a Unicamp, prometido com a expansão em Limeira, não foi executado; o repasse  total para as 3 estaduais poderia chegar a 9,57\% do ICMS total. 

Há batalhas judiciais e na ALESP para assegurar os 9,57\% ou ampliá-los para 11\%, tendo em vista o grande aumento de vagas nos últimos vários anos e os gastos fixos da  universidade, que atualmente dependem de um imposto variável. A resposta sempre é "não há dinheiro". O governo federal apresentou às suas universidades o Future-se,  programa pelo qual as universidades entregam sua administração a empresas gestoras (OSs) e investidoras de fundos. 

O modelo é duramente criticado, tanto pelos casos de corrupção envolvendo OSs em concessões de hospitais públicos quanto pela extinção, na prática, da Autonomia  Universitária, atacada pelo ministro da Educação. 

Na esteira de outras universidades, a Unicamp passou em votação no CONSU a  Política de Inovação Institucional, apelidado de "Inove-se", além de outros pacotes que vão assegurar um orçamento "extra". A regulamentação do Fundo Patrimonial da  Unicamp, por exemplo, prevê que um conselho composto por investidores e membros da  administração universitária gerencie recursos privados para prédios, bolsas e  contratações para fins específicos escolhidos pelos gestores do fundo.

Na prática, isso significa redirecionar a estrutura universitária para objetivos que se aproximam de interesses do mercado e podem se distanciar da pesquisa de interesse público (que não necessariamente dá lucro).

O ex-governador do estado João Doria promoveu, em 2019, na ALESP, uma CPI das Universidades para investigar "doutrinação ideológica", com deputados aliados ao  governo federal. Reitores, ex-reitores e funcionáries foram chamados para depor. O relatório final da CPI termina recomendando a cobrança de mensalidades. Em 2020, foi aprovada a cobrança de mensalidades na pós-graduação lato sensu após entendimento positivo do Superior Tribunal Federal quanto à legalidade disso em universidade  pública. 

\subsection{Como desmontar a educação para gastar menos}

Por acaso o governo está te pressionando pra cortar gastos e você não consegue dizer não? Calma, a gente tem uma solução bem simples pra isso e que quase não tem contraindicações: sucateie a educação da sua universidade! Não custa quase nada, só a produção científica, a educação de qualidade, a saúde mental e o preparo acadêmico de estudantes, a posição de referência científica e a autonomia universitária! 

Mas e aí? Qual a receita para 