\section{Cancelamento, trancamento e desistência}

Embora praticamente todes estudantes da Unicamp usem esses três termos
indiscriminadamente, como se fossem sinônimos, eles têm significados bastante
distintos para a DAC. Aí vai:

\subsection{Desistência de matrícula em disciplinas}

A desistência é o processo que é chamado peles alunes de “trancamento”. Você deixa de cursar essa disciplina no meio do semestre, tendo de cursá-la em algum semestre posterior (se for obrigatória). Só é possível desistir uma vez da disciplina e pode-se pedir desistência até que se tenha passado metade do semestre (confira a data limite da desistência no calendário!).

\subsection{Trancamento de matrícula}

Processo em que você não cursa nenhuma disciplina da Unicamp durante o semestre. É possível fazer até dois trancamentos de matrícula (em semestres seguidos ou não) e não se pode trancar nenhum dos dois semestres do ano de ingresso. A desistência de todas as disciplinas configura-se como trancamento. O trancamento é pedido na DAC, e pode ser pedido até que se tenha transcorrido 2/3 do semestre (de novo, veja o calendário!). Para cada trancamento, o prazo máximo de integralização é postergado, ou seja, o seu tempo fora da Unicamp não contará para o progresso do curso – isso pode ser muito útil em situações críticas.

\subsection{Cancelamento de matrícula}

Processo em que você se desliga da Unicamp, seja por motivo de jubilamento; por ter faltado às duas primeiras semanas do ano de ingresso; por ter sido reprovade em todas as disciplinas do primeiro ou do segundo semestre de ingresso; por ter sido expulse; por cursar outra universidade pública simultaneamente ou por vontade própria.

\subsection{Jubilamento? Mas eu acabei de entrar!}

É importante garantir a sua permanência na vaga! Segundo o regimento, o ingressante que tiver ausência em todas as aulas nas duas primeiras semanas do primeiro semestre terá a matrícula cancelada. Além disso, a não aprovação em nenhuma


em todas as aulas nas duas primeiras semanas do
primeiro semestre terá a matrícula cancelada. Além disso, a não aprovação em nenhuma

matéria obrigatória do curso em algum dos dois primeiros semestres resulta em jubilamento, independente do número de matérias eletivas ou completadas em outro semestre. Mantenha um CP (Coeficiente de Progressão) razoável: um CP menor que o CPE (Coeficiente de Progressão Exigido), também jubila (calma! tem que ser menos de metade dos créditos previstos). De resto, é sempre bom dar uma olhada nas outras regras no site da DAC, mas nada pra se preocupar demais agora no começo do curso. Elas podem ser encontradas no artigo 49 do regimento de graduação, disponível no site da DAC: \url{www.dac.unicamp.br/portal/graduacao/regimento-geral}.

\subsection{Mudança de catálogo}

Para cada ano, existe um catálogo correspondente com todas as disciplinas que os alunos devem fazer para se formar (currículo pleno). O seu catálogo é o 2023, por conta do seu ano de ingresso, mas é possível alterá-lo. Como a computação é uma área muito dinâmica, ocorrem várias mudanças no decorrer dos anos, como adição e remoção de disciplinas e aumento ou diminuição de créditos.

Como essas mudanças ocorrem num catálogo específico de um ano e só valem a partir desse, elas não são aplicadas a estudantes de anos anteriores. Para mudar isso, é possível mudar o ano de catálogo. Porém, existem ressalvas: se você mudar de catálogo, pode ser que você precise fazer mais matérias para se formar. Há também o risco de não conseguir equivalência de algumas matérias que são semelhantes entre si nos catálogos. Para mudar de catálogo, é só preencher o formulário \url{bit.ly/1LHPdyG} e entregar à DAC. Mas confira sempre com seu Centro Acadêmico e com veteranes se vale a pena!

\textbf{> Como ocorrem essas mudanças de catálogo?}

Todas feitas com muita discussão entre professores e representantes de alunes. Se
você não estiver antenade, pode ficar sabendo nas reuniões semestrais de avaliação de
curso. Fazemos nosso melhor para informar sobre todas as alterações e colher sugestões
nós mesmes, então fique atente!

\subsection{Avaliação de curso}

Avaliação de curso é uma reunião anual proposta pela coordenadoria que visa discutir todos esses pontos. Participe, pois é sua chance de tornar o curso melhor falando direta ou indiretamente com es coordenadores! Caso tenha sugestões, converse com a gente também e faremos o necessário para defender melhorias no curso!

Falamos um pouquinho mais sobre isso adiante, em \textbf{Avaliação de Professores}.