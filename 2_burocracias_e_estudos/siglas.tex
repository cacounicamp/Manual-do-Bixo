\section{Siglas!}

Antes de qualquer coisa, você vai logo perceber que, dentro da Unicamp, existem um zilhão de siglas diferentes Não se preocupe em decorar tudo!!! Com o tempo, você vai se familiarizar bem com elas, prometo!

Separamos, nessa sessão, explicações sobre algumas siglas, códigos e outros termos que são muito usados dentro da Unicamp Assim, enquanto estiver lendo o manual ou vivendo sua vida de ingressante, terá em mãos um glossário para consultar! Se houverem mais dúvidas ou siglas que não constam aqui, não hesite em perguntar a sues veterenes ou a nós do Centro Acadêmico. ;)

\begin{itemize}
    \item \textbf{Currículo pleno}: é o conjunto de disciplinas do curso que e alune tem que cursar para se formar.
    
    \item \textbf{Créditos}: unidade elementar de horas-aula de qualquer curso da Unicamp. Um crédito equivale a uma hora-aula semanal, ou a 15 horas-aula semestrais. Cada disciplina possui um “número de créditos” equivalente a ela Normalmente, esse número de créditos indica quantas horas semanais você terá de aulas dessa disciplina Por exemplo: cálculo 1 é uma matéria de 6 créditos; você terá, portanto, 3 aulas de 2 horas por semana (ou 6 horas-aula por semana). A cada semestre, você irá acumulando créditos. Para se formar no curso, você precisa completar um número específico de créditos (explicitados no currículo pleno).
    
    \item \textbf{CR (Coeficiente de Rendimento)}: valor entre 0 e 1 da média das notas em todas as disciplinas cursadas ponderada pelos créditos (ou seja, ir mal numa matéria de 6 créditos pode prejudicar muito seu CR). PS: atenção redobrada se sua intenção é seguir carreira acadêmica e/ou intercâmbio (mais detalhes sobre isso à frente).
    
    \item \textbf{CP (Coeficiente de Progressão)}: parte do curso que você já cumpriu (CP = 0,6123 significa que se cumpriu 61,23\% do curso).
    
    \item \textbf{CPF (Coeficiente de Progressão Futuro)}: é o CP que você terá no fim do semestre caso passe em todas as disciplinas que você se matriculou.
    
    \item \textbf{CPE (Coeficiente de Progressão Exigido)}: usado para fins de cancelamento de matrícula. Para que o aluno possa continuar a fazer o curso, ele precisa ter um CP maior ou igual ao CPE daquele semestre.
    
    \item \textbf{GDE}: incrível facilitador de vidas! É um site, e também rede social, onde você pode usar seu login da DAC para navegar entre disciplinas, salas de aula e listas de colegas. O Planejador e a avaliação de professores (extra-oficial) são a principal ferramenta. Projeto de Felipe Guaycuru, um ex-aluno da Engenharia de Computação, reconhecido pela DAC. O link é \textbf{\url{gde.ir}}.
    
    \item \textbf{Pré-requisito}: matéria(s) que precisa(m) ter sido cursada(s) para que se possa fazer outra(s) matéria(s). Existem dois tipos de pré-requisitos: Os pré-requisitos totais, mais comuns, do qual é exigido tanto a aprovação por nota como por frequência e os pré-requisitos parciais, mais raros, do qual o aluno não precisa ter sido aprovado por nota, mas tem que ter tido aprovação por frequência e nota final maior ou igual a 3,0. Os pré-requisitos parciais são identificados com um asterisco na frente do código da disciplina.
    
    \item \textbf{AA4xy}: um tipo de pré-requisito. Não se trata de nenhuma disciplina. Para fazer disciplinas com esse pré-requisito, e alune tem que tem um CP maior ou igual a 0,xy.
    
    \item \textbf{AA200}: outro tipo de pré-requisito existente, mais presente em disciplinas eletivas. Também não se trata de nenhuma disciplina. É apenas uma autorização da coordenadoria do curso. Se sobrar vagas para a disciplina e a coordenadoria do curso for com a sua cara, você faz a disciplina.
    
    \item \textbf{CCG (Comissão Central de Graduação)}: órgão colegiado da Unicamp, é encarregada da orientação, supervisão e revisão periódica do ensino na Universidade. Cabe recurso à CCG de quaisquer decisões das Unidades afetando o ensino.
    
    \item \textbf{CCPG (Comissão Central de Pós-Graduação)}: órgão colegiado da Unicamp, é encarregada da orientação, supervisão e revisão periódica da pós-graduação na Universidade. Cabe recurso à CCPG de quaisquer decisões das Unidades afetando o ensino.
    
    \item \textbf{Consu (Conselho Universitário)}: o Consu é o órgão máximo da Universidade, acima do reitor, embora ele faça parte e influencie fortemente suas decisões. Existe representação discente no Consu, eleita juntamente com a Coordenadoria do DCE.

    \item \textbf{Congregação}: é o órgão colegiado máximo do instituto ou faculdade.  Cabe recurso à Congregação da Unidade de Ensino de quaisquer decisões dos Departamentos e das Coordenações de Curso.

    \item \textbf{Departamento}: é administrado por um professor-chefe e um Conselho Departa-mental, é a menor unidade administrativa, didática e científica da Universidade, sendo responsável pelo desenvolvimento dos programas de ensino, pesquisa e extensão dos serviços à comunidade. Todo instituto e faculdade da universidade possui o seu conjunto de departamentos, conhecidos através de siglas.

    \item \textbf{CI (Conselho Interdepartamental)}: este é um “braço” da Congregação, responsável por tratar de assuntos menores, como despesas e atribuições de sala. É nele que ocorre atualizações sobre o estado das obras do IC-4 (pergunte a veteranes sobre esse mitológico prédio). Fazem parte deste órgão, além de um representante discente, o diretor do instituto, os coordenadores e os chefes de departamentos.

    \item \textbf{CDI (Comissão Diretora de Informática)}: outro braço da congregação, responsável por tratar de assuntos relacionados aos ambientes computacionais, deliberando sobre a atualização de infraestrutura, a organização da rede, endereços de internet e similares.

    \item \textbf{CG (Coordenadoria/Comissão de Graduação)}: é o órgão da unidade responsável pelos seus cursos de graduação. Sempre que houver algum problema ou deficiência no curso, é este órgão que vocês devem procurar.  Cada curso tem ume coordenadore (que faz parte da CG) e conta com representantes discentes, que devem ser sua primeira forma de comunicação com a CG. 

    \item \textbf{CPG (Coordenadoria/Comissão de Pós-Graduação)}: o órgão responsável pela pós-graduação no Instituto, coordenando as disciplinas oferecidas e as matrículas na pós.

    \item \textbf{DCE (Diretório Central de Estudantes)}: é a entidade de representação de estudantes de graduação da Unicamp, competindo-lhe ainda designar representantes estudantis para os órgãos colegiados da Universidade. É autoorganizada e eleita pelo corpo discente, sem interferência institucional.

    \item \textbf{DAC (Diretoria Acadêmica)}: é o órgão executivo e informativo, incumbido do registro e controle das atividades discentes da Unicamp. Cuida das matrículas, alteração de matrícula, emissão de documentos e relatórios, como o histórico escolar, realiza reserva de salas, entre outras atividades.

    \item \textbf{SAE (Serviço de Apoio ao Estudante)}: é encarregado da execução de programas de assistência desenvolvidas pela Universidade, por iniciativa própria ou mediante convênios firmados com entidades especializadas. Falaremos mais dele adiante também!

    \item \textbf{PB (Pavilhão Básico)}: também conhecido como Ciclo Básico II, é um prédio com várias salas de aula, que fica em frente ao Bandejão, e serve várias unidades que não possuem espaço físico suficiente para comportar seus alunos. No segundo andar ficam as salas de aula (PB01 a PB12) e no terceiro andar ficam os auditórios (PB13 a PB18).

    \item \textbf{CB (Ciclo Básico I)}: tem finalidade idêntica ao PB, só que é muito melhor equipado, tem uma acústica muito melhor e um ar condicionado bem frio. Fica na mesma praça que o PB, só que no outro extremo. À esquerda da entrada pela rua ficam as salas ímpares e à direita ficam as salas pares. No primeiro andar ficam os auditórios (CB01 a CB06) que possuem 140 e 160 lugares e no segundo andar ficam as salas de aula (CB07 a CB18) que possuem 60 e 80 lugares.  O CB e o PB são os lugares onde você vai ter a maioria das suas aulas (especialmente nos dois primeiros anos de curso, então não tem porque procurar morar perto do IC ou da FEEC, cuidado!).

    \item \textbf{RA}: o RA é seu registro acadêmico! Consiste num número de 6 dígitos que te identificará dentro da Unicamp. Você também receberá seu cartão universitário – que também costumamos chamar de RA! Não confundir com RA (Restaurante Administrativo).
\end{itemize}