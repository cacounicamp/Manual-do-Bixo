\section{CR e Reprovação}

Durante o curso, você provavelmente vai ouvir que se preocupar com seu CR é  bobagem; que estudar para tirar nota não leva a lugar nenhum; que depois de formade  não é seu CR que te colocará no mercado de trabalho; etc. \textbf{Cuidado! Trabalhar numa  empresa não é a sua única opção de vida após formade.} E atenção, pois “após formade”  não significa durante o curso. 

Durante o curso, você vai ter a possibilidade de participar de várias atividades  acadêmicas e algumas delas vão exigir bom aproveitamento acadêmico. \textbf{Por exemplo, para se candidatar à monitoria de uma disciplina, você precisa ter CR acima do da média  da turma. Para pleitear uma bolsa de iniciação científica, onde há concorrência entre alunes do país todo, também será exigido bom aproveitamento – assim como para uma  bolsa de mestrado.} Caso você não saiba, mestrado faz parte da pós-graduação, ou seja, o  seu CR vai te influenciar até após formade. 

Cuidado também com a reprovação. Há instituições, como a \textbf{FAPESP} (Fundação de  Amparo à Pesquisa do Estado de São Paulo – que é a maior fomentadora de pesquisas  do estado de SP e que paga os maiores valores de bolsas do país) \textbf{que te excluem de  qualquer disputa só por ter uma reprovação no seu histórico escolar da graduação}. Não  te excluem oficialmente, mas como é muito concorrido por ser a melhor pagadora, seu  nome vai para o final da lista. 

Parece óbvio para quem estuda tira boas notas, mas até você aprender a estudar  como a universidade exige, pode demorar um pouco; e há pessoas que nunca aprendem.  Cuidado também com o estudo exagerado! É um curso de 5 a 8 anos; não dá para  manter o ritmo de estudo para vestibular por todo esse tempo. Entenda e respeite seus  limites, não exija tanto de si.  

Há certos períodos (semestres) em que poderá sentir-se à vontade para desistir de  uma disciplina em que esteja matriculade, deixando para completá-la posteriormente.  Quando você fizer isso há a possibilidade de desistir da disciplina, desmatriculando-se  oficialmente. Mas há pessoas que simplesmente deixam de cursar a disciplina,  reprovando por nota e falta e ficando com uma nota baixa em seu histórico. Cuidado  com isso, pode ser frustrante para você no futuro. Por isso, se for desistir de cursar uma  disciplina após matriculade, sempre tente pedir a desistência e para não reprovar. 

Lembre-se de que o período de graduação é muito grande, você pode mudar de  ideia a qualquer momento sobre o que pretende fazer no futuro!
