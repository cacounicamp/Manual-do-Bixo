\section{Bem-vinde à academia!}

Academia é o nome que se dá à comunidade internacional de pesquisadores e  estudantes de ensino superior, atuando em todas as áreas do conhecimento. Geralmente  centrada ao redor de universidades, porém outras organizações públicas e privadas e  também empresas fazem parte. A Academia é dividida principalmente entre os pilares de  Pesquisa, Ensino e Extensão, sendo os dois primeiros mais acessíveis para nós da  graduação. A Unicamp está muito bem situada no cenário acadêmico, sendo a  responsável por parte considerável da produção científica brasileira.

\subsection{Iniciação Científica - IC}

A iniciação científica é um programa para alunes de graduação (você, no caso) ter  uma experiência acadêmica mais séria, sentir um pouco como é o clima de pesquisa.  Interessou? O que fazer? Geralmente o que se faz é conversar com professores da área  com a qual você se identifica mais (criptografia, computação teórica, processamento de  imagens, inteligência artificial etc.) e ver se estão desenvolvendo algum projeto  interessante – ou você pode propor alguma ideia sua mesmo. Depois, você começa a  estudar para redigir um projeto e o encaminha para alguma instituição de fomento à  pesquisa (CNPq ou FAPESP), pedindo uma bolsa de iniciação científica.

Além disso, você pode pegar a disciplina MC040 e posteriormente MC041, cada uma  com 12 créditos. São 24 créditos praticamente “de bandeja” para ajudá-le. Note bem as  aspas: iniciação científica geralmente consome muito tempo de estudo e dedicação, não  vá pensando que é moleza, não.

Na FEEC, você pode conseguir as matérias de iniciação (EE015 e EE016). Lá, a  iniciação científica também substitui o estágio, mas não tem equivalência com a do IC. Note que sua iniciação científica não precisa estar vinculada à computação. Como  falamos ainda neste capítulo, a Unicamp permite que você faça disciplinas de qualquer  instituto com créditos eletivos, isso pode te estimular a fazer matérias de áreas que  gosta fora da computação e nada te impede de fazer alguma iniciação científica nisso,  aproveite!

\subsection{Monitoria}

Além da iniciação científica, monitoria é uma forma muito comum de participar da  Academia, mais focada no pilar de Ensino. Não vai demorar muito para você descobrir a  importância da monitoria para as matérias. Logo no primeiro semestre, boa parte das  disciplinas possuem pessoas responsáveis por esse trabalho, que são dividas em dois  tipos: PED (Programa de Estágio Docente) e PAD (Programa de Apoio Didático).

PED é alguém da pós-graduação que é responsável por auxiliar docentes na disciplina - normalmente, criando materiais, exercícios, laboratórios e ajudando nas correções.  Já PAD é alguém da graduação e geralmente só é responsável por ajudar nos labo relatórios e tirar dúvidas, não deve incluir correção de trabalhos.

Por enquanto, o mais importante pra quem quer participar é o PAD. Todo mundo po de se inscrever, a forma e momento das inscrições varia de acordo com o instituto, mas  costuma ser no final do semestre. No IC e FEEC, recebemos um e-mail da Secretaria de  Graduação com um formulário para preencher.

O programa de PAD costuma incluir uma bolsa que fica em torno de R\$570,00. Os  pré-requisitos pra conseguir são (1) ter cursado a disciplina ou alguma equivalente, (2)  não ter reprovado na disciplina e (3) ter disponibilidade para participar das atividades,  que para o nosso curso costuma ser estar livre durante os laboratórios, (4) ter o CR  acima da média da sua turma ou ter a maior nota na matéria dentre quem se candidatou  para a bolsa.

Algumas disciplinas são mais concorridas que outras para monitoria, como é o caso  das primeiras matérias da computação. Algo interessante é que, assim como iniciação,  monitoria também tem duas disciplinas, MC050 e MC051 – ambas de 8 créditos. Mesmo  sem bolsa, é possível exercer monitoria, conseguindo os créditos. 

Mas vá com calma, como é preciso ter cursado a disciplina e ter um CR para se  candidatar pra monitoria, só vai ser possível se candidatar no final do seu segundo  semestre, já que as inscrições acontecem no fim do semestre (geralmente próximo da  matrícula). É bom se atentar que monitoria exige um certo tempo de dedicação, em  especial porque você estará ajudando outres estudantes na disciplina. É uma grande  responsabilidade e uma ótima forma de ter contato com um dos lados da Academia que  muitas vezes é esquecido: justamente o lado que lhe trouxe para cá!

\subsection{Intercâmbio}

A Unicamp é uma das universidades brasileiras que têm maior prestígio fora do país  e a VRERI-Unicamp (Vice-Reitoria Executiva de Relações Internacionais, antigo CORI), IC  e FEEC têm vários acordos bilaterais de intercâmbio. Então, para você que quer dar um  salto em algum idioma, conhecer outras culturas e buscar expandir suas oportunidades,  comece a se preparar desde já.

Muita gente fez intercâmbio pelo Ciência Sem Fronteiras, mas ele foi encerrado para  a graduação em 2017.

A França, hoje, recebe um número razoável de estudantes de computação, devido a  acordos que a Unicamp tem com os INSA e com as Écoles Centrales – e também graças a  bolsas de estudo oferecidas pela Capes e pelo governo francês. Os intercâmbios para a  França são bem mais concorridos que o Ciência sem Fronteiras, pois são na modalidade  de duplo diploma (você se forma pela Unicamp e pela instituição francesa); há um  processo seletivo envolvendo uma entrevista, e vai durar mais tempo, dois anos ou mais,  numa instituição de prestígio como a École Polytechnique, por exemplo.

No site da VRERI existem oportunidades para ir para Estados Unidos, Japão, outros  países da América Latina, Alemanha, Espanha etc, muitos com boas com incentivos que valem a pena caso você tenha um pouco de grana para se sustentar  no início. Visite sempre o site da VRERI e participe das reuniões que ela faz, pois ficar  ligade é a chave para conseguir encontrar uma boa oportunidade. Existem outras opções  de intercâmbio, como a AIESEC, que promove um intercâmbio para estágios no exterior.  Se seu interesse é mais profissional, procure se informar.

“Mas vale a pena? Poxa, vou atrasar meu curso, me deslocar da turma, vou ficar em  um país estranho, para quê? Acho que não vale a pena.”

Vamos começar pelos motivos profissionais: ter no currículo que você fala uma língua estrangeira fluentemente devido à sua imersão no país é algo muito valorizado  pelas empresas, além do fato que o pessoal do RH vai ver que você tem capacidade de  se virar sozinhe, uma vez que não é tão óbvio sair do país e recomeçar sua vida fora.  Você não atrasará tanto seu curso, pois a Unicamp conta com um sistema de  equivalências de matérias, e, se você escolher bem, pode fazer matérias que serão  convalidadas na Unicamp. Agora, o que realmente é importante: você está na faculdade,  é um bom momento de partir pro mundo! A experiência de conhecer outras culturas,  criar laços de amizade internacionais, viajar por terras desconhecidas não tem preço!  Pense que é no seu tempo de facul que terá oportunidade de fazer uma aventura destas,  não desperdice.

Se você abriu um sorriso e pensa que está preparade para sair do país, comece a  estudar! Não que ter uma boa nota seja a única forma de conseguir uma vaga em uma  bolsa de estudos, mas com certeza é a mais fácil. Busque atirar em todas as frentes,  mantenha seu CR num bom nível, busque conhecer organismos como AIESEC e procure  grupos de trabalho (no IC existem vários) que podem levar alunos ao exterior. Boa sorte!

Se você realmente se interessou, aí vão uns links com mais informações:

VRERI: \url{www.internationaloffice.unicamp.br}

AIESEC: \url{aiesec.org.br} 

Fique atente aos e-mails que você receberá do IC e da FEEC. Alguns deles são sobre  programas de intercâmbio.
