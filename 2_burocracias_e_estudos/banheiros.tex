\section{Melhores banheiros}

Uma das maiores necessidades do ser humano pode ser potencializada se for realizada num banheiro decente. Portanto, é muito importante que você saiba onde ir!  Alguns dos melhores banheiros da Unicamp são:

\textbf{IC-3}: Geralmente estão limpos e utilizáveis, mas de vez em quando têm um cheiro  ruim. E em dia de chuva ficam imundos. Exceto nos finais de semana, sempre possui  papel higiênico: é uma boa pedida na hora do apuro.

\textbf{IC-3,5}: Os banheiros do térreo não são flor que se cheire, porém, devido à localização  um pouco mais distante, pode ser suficiente se você prefere um pouco mais de  privacidade. O grande segredo são os banheiros da pós-graduação, localizados no  primeiro andar. Estes são ainda menos frequentados e a qualidade é extremamente  superior aos do andar inferior. Além disso, a torneira é excelente pra escovar os dentes  depois do almoço. Nesses casos, subir algumas escadas compensa bastante.

Observação: se o bebedouro perto dos laboratórios estiver com um gosto na água, o  bebedouro do primeiro andar do IC-3,5 tem uma vazão excelente e é um ótimo  substituto. Perfeito para encher garrafinhas de água também.

\textbf{IC-2}: Quase sempre estão limpos e utilizáveis e tem um odor melhor que os do IC-3.  Só precisa tomar cuidado pois às vezes falta papel higiênico.

\textbf{FEEC}: Possui excelentes banheiros escondidos por lá, principalmente após as refor mas de 2013. Procure bem! 

\textbf{PB}: Os banheiros do segundo e do terceiro andar do Pavilhão Básico são muito  melhores que os do térreo (especialmente os do terceiro andar, por quase não serem  usados). Só tome cuidado, porque às vezes não tem papel higiênico. 

\textbf{FE}: A Faculdade de Educação tem poucos banheiros masculinos. Estão entre os  melhores da Unicamp pelo pouco uso. 

\textbf{CB}: Estes banheiros ficam escondidos próximo às escadas do CB (no térreo). Se você  tiver sorte de chegar bem após a limpeza, o banheiro estará em excelentes condições.  Porém, na maior parte do tempo, eles ficam bem sujinhos. 

\textbf{DEQ}: Departamento de Eletrônica Quântica, no IFGW. Dizem que ninguém os usa. 

\textbf{DRCC}: Departamento de Raios Cósmicos e Cronologia, no IFGW. Um dos melhores  banheiros existentes na Unicamp (se não O melhor). Dizem que ninguém os usa. 

\textbf{DFA}: Departamento de Física Aplicada, no IFGW. Os dois andares do departamento  tem banheiros bons e utilizáveis, mas algumas vezes falta papel higiênico. 

\textbf{IMECC}: Todos os três departamentos (andares) do IMECC têm banheiros bons e  utilizáveis, mas vez ou outra falta papel higiênico. Evite os banheiros do térreo,  geralmente estão bem nojentos.
