\section{Avaliação de professores}

Achou que e professore ensinou muito mal? Falou da vida, do universo e tudo mais  ,exceto da disciplina? Foi incoerente? Ou, pelo contrário, achou e professore incrível? Se você sentir que as coisas simplesmente não estão funcionando em sala de aula ou  que algo de errado não está certo, traga o assunto para o CACo. Seu problema com a  matéria, na maioria das vezes, não é só seu. Como representantes dos seus interesses,  temos o dever e o prazer de fazer tudo que estiver ao nosso alcance para melhorar a  situação. 

O GDE também tem um sistema de avaliação de professores, cuja nota costuma ser  usada peles alunes como um dos critérios no momento de decidir com que professore  puxar uma matéria – mas prefira o feedback de veteranes. 

Além disso, temos a \textbf{Reunião de Avaliação de Curso}, já comentada. A data e o horário  são divulgados pelas unidades e pelo CACo. Essa é uma oportunidade de passar para às  coordenadorias do curso não só suas impressões sobre professores e disciplinas, mas  sobre qualquer assunto relacionado ao curso. Complementarmente, todes professores  devem disponibilizar um formulário de avaliação nas últimas aulas de cada semestre.  Esse é o momento para que você possa separar os acertos dos erros, portanto preencha  com seriedade. Os dados serão analisados pelas Comissões de Graduação de cada  unidade e os comentários escritos serão repassados para e professore. 

Tenha sempre em mente que a nossa percepção sobre o oferecimento de uma  disciplina não é óbvia para es professores. Preencha todas as avaliações com  sinceridade, use sempre os espaços dedicados a comentários com críticas construtivas. 
