% Este arquivo tex vai ser incluído no arquivo tex principal, não pe preciso
% declarar nenhum cabeçalho

\section{Comida}
\subsection{Bandejão}

Um dos momentos de glória do dia de um futuro engenheiro, cientista ou bacharel
é o bandejão. É a hora de intensas e indiscutíveis emoções. Caso sua salada
corra sobre a mesa, mantenha-se calmo. Evite discussões, jamais tente descobrir
o sabor do suco pelo paladar (limão ou pêssego?) é mais cômodo ler no cardápio
do dia. Uma dica: para cortar o bife faça muita força e quando começar
a amolecer pare, você chegou na bandeja.

Falando sério agora: O bandejão (Restaurante Universitário), Bandex, ou Bandeco,
fica ao lado da Biblioteca Central, bem em frente ao PB (Prédio Básico, ou Ciclo
Básico II) e, a menos que você não queira economizar uma boa grana com comida,
vai ser o lugar onde você vai estar na maioria dos seus horários de almoço. Com
o tempo, você vai ver que o Bandex é o "coração da UNICAMP": É o local de você
se encontrar com os amigos (combinando ou não antes), contar os micos nas aulas,
jogar conversa fora, e falar mal da comida, que nem é tão ruim assim como muitos
dizem. Sem dúvida, é o melhor custo-benefício da UNICAMP: Por R\$2,00, você tem
direito a arroz, feijão, salada, proteína de soja, suco, chá e café à vontade,
a carne e a sobremesa tem que dar uma choradinha para a tiazinha para poder
repetir, mas geralmente dá certo.

Existe também o RA (Restaurante Administrativo), também conhecido como Pratex,
pelo fato da comida ser servida em pratos e não em bandejas. Fica atrás da
Faculdade de Engenharia Elétrica e de Computação (FEEC), perto do prédio da Engenharia Básica. Tem algumas diferenças em
relação ao Bandex: O espaço físico é bem menor, por exemplo. No RA você mesmo se
serve, apesar da carne às vezes ser servida pela tia que trabalha lá. Se você
for com um amigo, vá com paciência para esperar, porque é difícil pra arrumar
lugar, além ser ultra apertado. Dependendo de onde você vai ter aula antes ou
depois do almoço, é mais negócio almoçar no Pratex. Para poder usar o Bandex
e o RA, você deve estar com o seu cartão (C.U.) carregado.

Nota: se vocês derem sorte, pode ser que o novo Pratex já esteja pronto e tenha
sido aberto. Ele é localizado próximo ao IC3 e ao prédio azul da Civil.

\subsubsection{Como funciona o esquema de carregar o cartão?}

Simples. Você vai ao guichê ao lado esquerdo da entrada do Bandex, e faz, por
exemplo, um depósito de R\$20,00 para 10 créditos. Outra maneira de colocar
créditos no C.U. é fazer um depósito na conta do Bandex no Banespa (Ag.: 207
/ Conta: 43.010.009-2) ou no Banco do Brasil (Ag.: 4203-X / Conta: 66.315-8)
e depois carregar o seu cartão, no guichê do Bandex ou na Prefeitura do Campus
(próximo à Reitoria). Fique esperto para não ir ao RA sem créditos, porque lá
não dá pra carregar o cartão e você terá que andar até o Bandeco.

Os bandejões funcionam de segunda a sexta, nos seguintes horários:

\begin{itemize}
\item  RU, das 10h30 às 14h00 (almoço) e das 17h30 às 19h40 (jantar).
\item  RA, das 11h30 às 14h00 (almoço) e das 17h30 às 19h00 (jantar).
\end{itemize}
Para saber previamente o cardápio do Bandejão, acesse o site do CACo (\url{http://www.caco.ic.unicamp.br}).

Uma opção para consultar o cardápio do bandejão é o bandecowap. Ele foi feito
pensando na possibilidade de consultar o cardápio através do telefone celular,
mesmo aqueles mais simples que oferecem acesso à internet mais básico, na forma
de WAP. Há também a possibilidade de instalar um gadget na página inicial do
Google, a iGoogle. Acesse pelo \url{www.tinyurl.com/bandecowap} ou, se preferir
adicionar aos favoritos, use o \url{www.students.ic.unicamp.br/~ra092856/bandecowap}
- o acesso por este é ligeiramente mais rápido, pois não há o redirecionamento.

\subsection{Outros lugares para as refeições}

Lugares que servem pratos feitos são a Física (um os melhores da UNICAMP, serve
também meio-prato), o IFCH (a comida lá é bem barata) e a Química (bem parecido
com o da Física).

A cantina do DCE tem self-service barato e com variedade no almoço. Outros
lugares que tem self-service são a Artes, a Educação, a Física e a Mecânica.

Se você é vegetariano, uma boa dica é o Gatti (que fica do lado do IC-2, na
Cênicas/Dança).

Fora da Unicamp: próximo ao balão da Av. 1, temos também o Terraço, que vende
marmitex e tem self-service a um preço bom, além de churrasco às terças, quintas
e sábados. Um pouco mais acima na Av. 1, tem o Bardana (um com a fachada toda
verde), que está na mesma faixa de preço do Terraço, e costuma ser considerado
bem melhor; tem churrasco de carne bovina meio que dia-sim-dia-não, e nos outros
dias é de frango. Próximo ao Bardana, tem o Pepe Loco, que serve comida mexicana
no estilo fast-food. Na frente da reitoria há o Del Sol, o Ginza e o Moriá.
O Del Sol serve comida por quilo, sendo parecido (em preço e pratos) com
o Bardana, enquanto que o Ginza serve a la carte com preços bons (uma dica
é a feijoada completa às quartas, que sai por R\$10,00 e inclui uma
mini-capirinha!) e o Moriá serve pratos feitos a preços mais baratos. Próximo ao
Ginza, em frente à guarita do HC, há o Campus Grill, com comida boa a um preço
um tanto alto (um pouco mais caro que a cantina da Física). Tem o Aulus, na Av.
2, próximo ao balão, que é o mais caro dos citados aqui, mas é muito bom (e
bonito).

\subsection{Lanches e sucos}

Tá de tarde, bateu fome, quer comer um lanche (hamburger, pão-na-chapa, queijo
quente, x-salada, croissant, qualquer coisa do gênero)? Quase todas as cantinas
da UNICAMP servem lanche. Algumas muito boas são o IFCH, a Física, a lanchonete
do IEL e a lanchonete da Economia.

Quase todas as cantinas servem salgados prontos, lanches naturais, doces
e demais coisas do gênero.

Para sucos, tem dois lugares muito bons: a cantina da Física e a famosíssima
banca de sucos do CB, que tem milhões de sucos, vende frutas e também salgados.
Se você precisa almoçar rápido, provavelmente sua escolha será salgado
+ vitamina na banca de sucos do CB. Todo dia a banca de sucos do CB tem um sabor
na oferta, que é ótimo pra sair do tradicional suco de laranja.

Açaí: a cantina da física tem um açaí na tigela, caro, mas bom. Se por algum
motivo você tiver de andar até o quarteirão de salas de aula da medicina,
estiver cansado, e quiser um açaí, o da cantina de lá é caro
e inacreditavelmente zoado.

Nas quartas-feiras há uma feira no centro da praça do CB, na qual há opções bem
variadas, desde pastéis a comida japonesa, embora geralmente mais caras que as
cantinas. Algumas das barracas abrem também na quinta-feira.

\subsection{Padarias e café da manhã}

Cinco cantinas da UNICAMP abrem bem cedo e servem o bom pingado + pão na chapa
matinal. São elas a Mecânica, a cantina do IFCH, a cantina do DCE, a da Química
e a da Física.

A Padaria Alemã serve uma bandeja de café da manhã com suco,
café-com-leite/chocolate, croissant, mamão, bolo, pão francês, torradas,
manteiga e geléia. Ainda há a possibilidade de fazer trocas como: suco por
chocolate, croissant por dois pães-na-chapa, mamão por banana, coisas do gênero.
Também são servidos lanches gigantescos, com muitas opções de recheio, por um
preço relativamente barato, então tenha alguém para dividir (acredite, meio
lanche já serve como um almoço completo). Dependendo do recheio, a pizza é muito
barata, também, embora eles não façam delivery. A Alemã fica na Avenida 1 (a da
saída da FEEC). É bom lembrar que eles servem café-da-manhã das 7h até às 13h
(mas a padaria só fecha às 22h), então é uma boa pedida para se você não quiser
almoçar ou para sábado e domingo, acordar tarde e tomar um café da manhã para
valer pelo almoço.

Na Estrada da Rhodia, próximo à entrada da Cidade Universitária II, há
a Paneteria Di Capri, que tem um pão francês muito bom (a um preço legal)
e também muita variedade (incluindo tortas e lanches). Além disso você também
pode tomar seu café da manhã lá, pois como quase toda padaria eles também
oferecem um cardápio bom para logo cedo. Se você estiver com bastante apetite,
de sexta a domingo eles servem um buffet de café-da-manhã com muitas opções
e a um preço fixo (em torno de R\$10). Na hora do almoço também são preparados
alguns pratos (para comer no local e para levar) e também há um esquema onde
você pede um grelhado e tem acesso livre a um balcão com saladas e outras
coisas, como petiscos. À noite eles servem pizzas e também há o esquema do
grelhado, exceto no inverno, quando eles servem um buffet de sopas.

Já se você está na Unicamp e quer uma padaria, a dica é a Padaria da FEA (fica
próxima à Cantina da Mecânica). Lá eles tem pães, doces e bolos. Com uma
diferença: há produtos especiais, como pão de queijo com linhaça ou alho e pão
francês com soja. Mas não se assuste: por mais estranho que pareçam, os produtos
de lá são muito bons! E não deixe para ir lá depois das aulas, pois a Padaria da
FEA fecha às 17h.

\subsection{E no fim de semana?}

Nos fins de semana, nem o Bandex nem quase nenhuma cantina da Unicamp abrem (e
as que abrem só o fazem no sábado). Você vai ter que se virar fora da Unicamp.
Na Av. 1 e proximidades tem o Terraço, o Bardana e a Padaria Alemã já citados,
o Lilly (que assim como o Terraço e o Bardana não abre de domingo), além de
vários restaurantes próximos à Alemã. Na Av. 2 tem o Aulus (mais caro no sábado
que durante a semana; domingo, então, mais ainda, mas costuma ter camarão
à milanesa), o Clos Vert (também é caro), um pouco mais pra cima na avenida e,
pouco depois, há o Yaki-Ten, que serve comida chinesa por quilo e japonesa por
pessoa. Logo mais abaixo há o Ilha do Barão. No centro de Barão não faltam
opções. Tem (indo da entrada de Barão pela Estrada da Rhodia) o Estância Grill,
o Barão da Picanha, o Gordão Burguers, o Solar dos Pampas, o Universo Massas,
o Vila Santo Antonio, o Ki-Pizza, o restaurante Baroneza, o Salsinha
e Cebolinha, o Pão de Açúcar e vários restaurantes no Tilli Center (a dica
é o Subway, por menos de 10 reais você come bem). Na Av. Santa Isabel
e adjacências tem o Cronópio (numa rua paralela à Santa Isabel), o Frangonete
(próximo ao Banespa), o HotDog Central e as Pizzarias Sapore Pizza e Pizza
Fiori. Perto da moradia tem a Tonha (Canto do Acarajé), o Kalunga Lanches
e o famoso dogão da moradia. Por fim, próximo à padaria Di Capri, há alguns
restaurantes mais caros, como a Romana (serviço parecido com o da Di Capri,
porém um bocado mais cara), Pizzaria Gregória, o TBONE (eles também tem
marmitex), o Greg Burgers (o hambúrguer e o milk-shake são excelentes), o Tábua
dos Mares e o Morena-flor.

\subsubsection{Alguns telefones:}

\begin{itemize}
\item  \textbf{Restaurante Baroneza}
\\\underline{Telefone:} (19) 3289-9087.
\\\underline{Endereço:} Rua Benedito Alves Aranha, 44 -- Centro de Barão Geraldo.
\\\underline{Site:} \url{http://www.restaurantebaronesa.embarao.com/}
\end{itemize}

\begin{itemize}
\item  \textbf{China In Box} (Faz entrega em Barão)
\\\underline{Telefone:} (19) 3254-5601
\\\underline{Endereço:} Rua Romualdo Andreazzi, 333 -- Jd. Trevo.
\end{itemize}

\begin{itemize}
\item  \textbf{TBONE Steak Bar}
\\\underline{Telefone:} (19) 3289-0485.
\\\underline{Endereço:} Rua Maria Tereza Dias da Silva, 700.
\end{itemize}
\end{itemize}

\begin{itemize}
\item  \textbf{Habibs} (Não está mais entregando em Barão)
\\\underline{Telefone:} 0800-778-2828
\end{itemize}

\begin{itemize}
\item  \textbf{Ginza Bar}
\\\underline{Telefone:} (19)3289-9281
\\\underline{Endereço:} Rua Roxo Moreira, 1768.
\end{itemize}
\end{itemize}

\begin{itemize}
\item  \textbf{Bardana}
\\\underline{Telefone:} (19)3289-9073.
\\\underline{Endereço:} Avenida Dr. Romeu Tórtima, 1500.
\end{itemize}

\begin{itemize}
\item  \textbf{Terraço}
\\\underline{Telefone:} (19) 3289-7920
\\\underline{Endereço:} Rua Roxo Moreira, 1344
\end{itemize}

\begin{itemize}
\item  \textbf{Pastelaria Oba-Oba}
\\\underline{Telefone:} (19) 3249-1908
\\\underline{Endereço:} Rua Benedito Alves Aranha, 115
\end{itemize}

\begin{itemize}
\item  \textbf{Estância Grill}
\\\underline{Telefone:} (19) 3289-6055
\\\underline{Endereço:} Avenida Albino José Barbosa de Oliveira, 271
\end{itemize}

\begin{itemize}
\item  \textbf{Pizzaria Borda de Ouro.}
\\\underline{Telefone:} (19) 3289-0867.
\\\underline{Endereço:} Luiz Vicentin Sobrinho, 457.
\end{itemize}

\begin{itemize}
\item  \textbf{Barão das Pizzas.}
\\\underline{Telefone:} (19) 3249-1630.
\\\underline{Endereço:} Rua Agostinho Pattaro, 187.
\end{itemize}

\begin{itemize}
\item  \textbf{Pizza Fiori.}
\\\underline{Telefone:} (19) 3289-3514.
\\\underline{Endereço:} Avenida Santa Isabel, 405.
\end{itemize}

\begin{itemize}
\item  \textbf{Ki-Pizza.}
\\\underline{Telefone:} (19) 3289-0863.
\\\underline{Endereço:} Rua Horácio Leonardi, 76.
\end{itemize}

\begin{itemize}
\item  \textbf{Bella Pizza.}
\\\underline{Telefone:} (19) 3289-7777.
\end{itemize}

\begin{itemize}
\item  \textbf{Pizza Show}
\\\underline{Telefone:} (19) 3324-7480
\end{itemize}

\begin{itemize}
\item  \textbf{Quero Mais.}
\\\underline{Telefone:} (19) 3289-4072.
\end{itemize}

\begin{itemize}
\item  \textbf{Estação Santa Fe Pizza.}
\\\underline{Telefone:} (19) 3289-4800.
\end{itemize}

\begin{itemize}
\item  \textbf{Pizza Gigante (Mega Pizza).}
\\\underline{Telefone:} (19) 3289-0320.
\end{itemize}

\begin{itemize}
\item  \textbf{Vila Ré Pizza.}
\\\underline{Telefone:} (19) 3289-0321.
\end{itemize}

\begin{itemize}
\item  \textbf{NADOG'S -- HOT DOG DO NADO}
\\\underline{Telefone:} (19) 3029-2270.
\end{itemize}

\begin{itemize}
\item  \textbf{Casa da Moqueca} (prato mais caro, mas serve duas pessoas)
\\\underline{Telefone:} (19) 3289-3131.
\end{itemize}

\subsection{E de madrugada?}
\begin{itemize}
\item  \textbf{Rob's Burgers:} (entregas até as 23:30).
\begin{itemize}
\item  Telefone: (19)3289-6541.
\item  Endereço: Avenida Santa Isabel 1510 (em frente ao portao 2 da moradia da UNICAMP).
\item  Tem vários tipos de lanches, não cobram taxa de entrega e é aberto das 18:30 as 23:30. E para quem estiver com muita fome tem o x-especial que é grande e muito bom.
\end{itemize}
\end{itemize}

\begin{itemize}
\item  \textbf{Hot-dog Independência:}
\begin{itemize}
\item  Telefone: (19)3289-8805
\item  Endereço: Rua Angela Signol Grigol, 742
\item  Tem vários tipos de hot-dogs (com catupiry, com cheddar, com frango{\dots}) e tem preços menores que os do Rod Burguers. O único problema é que eles cobram taxa de entrega para um lanche e fecham à meia-noite.
\end{itemize}
\end{itemize}

\begin{itemize}
\item  \textbf{Kalunga Lanches:}
\begin{itemize}
\item  Telefone: (19)3289-5236
\item  Endereço: Rua Sebastião Bonomi, 40
\item  Perto da moradia, eles não entregam, mas ficam abertos até altas horas. Destaque para o caldinho de feijão. Obs: o lugar é limpo e bom.
\end{itemize}
\end{itemize}

\begin{itemize}
\item  \textbf{Ponto Final:}
\begin{itemize}
\item  Telefone: (19) 3288-0204.
\item  Endereço: Avenida Albino José Barbosa de Oliveira, 2287
\item  Lanchonete localizada na estrada da Rhodia e entrega lanches até a meia noite. Tem tradição de ter preços caros, por isso não se estranhe. Muitos gostam bastante dessa lanchonete pela famosa maionese temperada que eles servem, não se esqueça de pedir quando for comprar lanches.
\end{itemize}
\end{itemize}

\begin{itemize}
\item  \textbf{Gordão:}
\begin{itemize}
\item  Telefone: (19)3289-9753
\item  Endereço: Avenida Albino José Barbosa de Oliveira, 476
\item  Rua Localizada na entrada de Barão Geraldo servem lanches parecidos com o do Ponto Final, lá eles dão outro tipo de maionese e em geral os preços são tão caros quanto do Ponto Final. Também entregam até meia noite.
\end{itemize}
\end{itemize}

\begin{itemize}
\item  \textbf{Lanchão \& Cia:}
\begin{itemize}
\item  Telefone: (19)3289-3665
\item  Endereço: Avenida Albino José Barbosa de Oliveira, 1214
\item  Site: \url{http://www.lanchaoecia.com.br/}
\item  (Fechou, abriu um novo barzinho no lugar em Barão). Um dos melhores lanches de Campinas (quiçá o melhor). Os lanches geralmente são grandes e muito bons, e os preços são compatíveis com a qualidade e quantidade. Eles servem no carro se você preferir, com uma bandeja que fica presa no vidro. Fica no centro de Barão Geraldo, proximo ao Banespa e Pão de Açucar. Destaque para a batata frita, feita de uma forma muito diferente, extremamente crocante e quase cremosa por dentro. Há outras duas lojas próximas da avenida Norte-Sul. Uma na Rua Oriente, e outra na Orozimbo Maia. O Art Lanches (antigo Lanchão 2), que fica no Taquaral, serve lanches parecidos.
\end{itemize}
\end{itemize}

\begin{itemize}
\item  \textbf{Ponto 1:}
\begin{itemize}
\item  Telefone: (19)3289-2378.
\item  Rua Eduardo Modesto, 54
\end{itemize}
\end{itemize}

\begin{itemize}
\item  \textbf{Sapore Pizza:}
\begin{itemize}
\item  Telefone: (19)3289-0228
\item  Endereço: Avenida Santa Isabel, 326
\item  Para quando você estiver com pelo menos mais um amigo para rachar a pizza, acaba sendo uma boa pedida. Geralmente as pizzas de mussarela e de calabreza estão com preços bem acessíveis. Também entregam até meia-noite.
\end{itemize}
\end{itemize}

\begin{itemize}
\item  \textbf{Barraquinhas:}
\begin{itemize}
\item  Há várias barraquinhas de hot-dog no centro de Barão e perto da moradia. Destaque para o dog do terminal, o Hot Dog Central, o Pedrogue e o dogão da moradia. Se você quiser um lanche, uma boa pedida é o Star Trash (Raimundão ou Guarujá, chame como você quiser), que fica perto do balão da avenida 2 e costuma ficar aberto até altas horas. Perto da UNICAMP, ao lado do posto Ipiranga que fica na avenida 1 também tem um dog prensado muito bom e barato.
\end{itemize}
\end{itemize}

\subsection{Marmitex}

Entrega em casa. Bom e barato.

\begin{itemize}
\item  \textbf{Tia Rita.}
\begin{itemize}
\item  Telefone: (19) 3249-2899.
\end{itemize}
\end{itemize}

\begin{itemize}
\item  \textbf{Hailton}
\begin{itemize}
\item  Telefone: (19) 3249-0153.
\end{itemize}
\end{itemize}

\begin{itemize}
\item  \textbf{Copa e Cozinha}
\begin{itemize}
\item  Telefone: (19) 3249-0153.
\end{itemize}
\end{itemize}

\begin{itemize}
\item  \textbf{Império do Barão}
\begin{itemize}
\item  Telefones: (19) 3289-8054 e (19) 3289-2170.
\end{itemize}
\end{itemize}

Obs: A Sapore Pizza também entrega Marmitex.

\subsection{Bares, lanchonetes e restaurantes}
\begin{itemize}
\item  \textbf{Açaizeiro Brasil:} Serve um açaí muito bom e vários tipos de comidas mais leves, como lanches naturais, crepes e saladas, além de vários sucos. O preço não é caro e a comida é boa. Endereço: Avenida Santa Isabel, 518. Telefone: (19) 3365-6555. Site: \url{http://www.portalbaraogeraldo.com.br/anunciantes/acaizeiro-brasil/}
\end{itemize}

\begin{itemize}
\item  \textbf{Alkobar:} Comida árabe e pizzas no início da Santa Isabel. Bom atendimento, comida legal e preço razoável. Telefone: 3289-2755.
\end{itemize}

\begin{itemize}
\item  \textbf{Aulus VideoBar \& Restaurant:} A comida lá é muito boa, só que é muito caro também (especialmente no final de semana), exceto pelo marmitex. Um ambiente diferente, com bicicletas e ferroramas no teto, por exemplo.
\end{itemize}

\begin{itemize}
\item  \textbf{Bagdá Café -- Bar \& Esfiharia:} Esfihas boas, mas um pouco caras. Entregam em Barão (cardápio no site), mas em horários de pico costumam demorar um pouco. A música ambiente inclui música ao vivo e ritmos variados, desde a MPB ao Blues. Endereço: Av. Santa Isabel, 246. Telefones: (19)3289-0541 e (19)3289-1842. Site: \url{http://www.carlinoamaral.com.br/bagda/}
\end{itemize}

\begin{itemize}
\item  \textbf{Bar do Coxinha:} Famoso pela coxinha (realmente boa), vale a pena ir lá, mas é relativamente caro. Localiza-se perto da avenida Santa Isabel, na rua da Sapore Pizza.
\end{itemize}

\begin{itemize}
\item  \textbf{Bar do Jair:} Outro lugar famoso pela coxinha: só que esta é de carne seca. Fica na Rua Eduardo Modesto, 212, relativamente perto da Moradia.
\end{itemize}

\begin{itemize}
\item  \textbf{Barão da picanha:} Churrascaria rodízio localizada na avenida Albino José Barbosa de Oliveira, logo na entrada de Barão.
\end{itemize}

\begin{itemize}
\item  \textbf{Batataria Suiça:} Do lado do Ponto Final, serve batatas recheadas bem diferentes. É um pouco caro, mas vale a pena conferir. Uma dica é que às terças-feiras você compra uma batata, mas recebe duas. Endereço: Estrada da Rhodia -- Praça José Geraldi, a 50m do posto Esso. Telefone: (19) 3201-1174. Site: \url{http://www.battataria.com.br}
\end{itemize}

\begin{itemize}
\item  \textbf{Boi Falô:} O restaurante é uma rancho, com comida típica do interior. É excelente, mas um pouco caro (cerca de R\$30,00 por pessoa), um lugar perfeito para levar seus pais quando eles vêm te visitar (e pagam o almoço!). Abre apenas nos almoços de sábado e domingo. Endereço: Rua do Sol, 600. Telefone: (19) 3289-6671 e (19) 3287-6342. Site: \url{http://www.boifalo.com.br/}
\end{itemize}

\begin{itemize}
\item  \textbf{Cachaçaria Água Doce:} Localizada na avenida 1, é um lugar frequentado por pessoas mais velhas, ótimo para comida e bebida (pinga, especialmente), mas é bem caro.
\end{itemize}

\begin{itemize}
\item  \textbf{Casa São Jorge:} Música ao vivo todas as noites, com boa variedade. Localiza-se na rua Santa Isabel, mais ou menos perto da moradia.
\end{itemize}

\begin{itemize}
\item  \textbf{Empório Nono:} Caro, tem um chopp muito bem tirado e os melhores petiscos de Campinas. Localiza-se na avenida Albino José Barbosa de Oliveira, quase em frente ao terminal.
\end{itemize}

\begin{itemize}
\item  \textbf{Estância Grill:} Logo na entrada de Barão. Tem rodízios de carne e de pizza à noite. Endereço: Avenida Albino José Barbosa Oliveira, 271. Telefone: (19) 3289-8697/3289-6055/3289-1511
\end{itemize}

\begin{itemize}
\item  \textbf{Fernando's:} No centro de Barão, perto do Banespa, serve cerveja e lanches baratos e muito bons principalmente porque vêm acompanhados de uma porção pequena de fritas! Um lugar simples mas muito limpo e agradável principalmente em relação ao atendimento. Fecha as 23h se segunda a quinta e sábado, tem música ao vivo na sexta e por enquando ainda não abre nos domingos.
\end{itemize}

\begin{itemize}
\item  \textbf{Fran's Café:} Cafeteria. Vende lanches, cafés, doces, salgados e bebidas (quentes ou geladas). Fazem também cafés da manhã. Mas é um pouco caro. Localizado na avenida Albino José Barbosa de Oliveira, 1600.
\end{itemize}

\begin{itemize}
\item  \textbf{Greg Burguers:} Uma lanchenete muito boa, mas também muito cara. Uma das especialidades lá é o milk-shake (realmente muito bom). Fica na estrada da Rhodia (na esquina da Paneteria Di Capri). Só funciona à noite, de terça a domingo. Endereço: Rua Maria Tereza Dias da Silva 664, Barão Geraldo, 13083-820 Campinas. Telefone: (19) 3289-6400. Site: \url{http://www.gregburgers.com.br}
\end{itemize}

\begin{itemize}
\item  \textbf{Ilha do Barão:} Lá eles fazem entrega de marmita de graça, a 5,00, e a comida é boa. Também tem cerveja barata. Fica na avenida 2, perto do Texaco e do Mc Donalds.
\end{itemize}

\begin{itemize}
\item  \textbf{La Salamandra:} Restaurante mexicano, localizado ao lado do Makis Place. Comida boa e preço compatível, ele tem uma barraquinha na feirinha do CB, às quartas. Endereço: Avenida Albino José Barbosa de Oliveira, 998. Telefone: (19) 3289-2011/(19) 9277-4340/(19)3365-0354. Site: \url{http://www.portalbaraogeraldo.com.br/anunciantes/la-salamandra-culinaria-mexicana-/}
\end{itemize}

\begin{itemize}
\item  \textbf{Makis Place:} Temakeria próxima ao terminal. Endereço: Avenida Albino José Barbosa de Oliveira, 976. Telefone: (19) 3367-3077. Site: \url{http://www.makis.com.br/}
\end{itemize}

\begin{itemize}
\item  \textbf{Ponto Final:} Já comentado acima, fica aberto até altas horas. Mas à noite serve cerveja também a um bom preço. Localiza-se na estrada da Rhodia (continuação da avenida Albino José de Oliveira). Endereço: Av. Albino J. B. Oliveira, 2287. Telefone: (19)3288-0204.
\end{itemize}

\begin{itemize}
\item  \textbf{Quintal do Neto:} No alto da avenida 1, perto do balão de entrada em Barão Geraldo, tem cerveja a preços razoáveis, salgados (coxinha e quibe) grandes, e mesas de sinuca (de ficha e por hora).
\end{itemize}

\begin{itemize}
\item  \textbf{Rudá:} Localizado na Santa Isabel, é um bar novo, com música ambiente.
\end{itemize}

\begin{itemize}
\item  \textbf{Santa Fé Pizza Bar:} Local um pouco caro, mas normalmente com música ao vivo de ótima qualidade. A melhor pizza de Barão (também cara mas mais barata que a Pizzaria Vila Ré). Próximo ao Pão de Açúcar.
\end{itemize}

\begin{itemize}
\item  \textbf{Seu Pimenta:} Muito caro, mais caro até que o Santa Fé e o Nono. Bom atendimento, porções de boteco, e às vezes música ao vivo. Em frente ao Zé Espaço e Bar.
\end{itemize}

\begin{itemize}
\item  \textbf{Solar dos Pampas:} Fazem um esquema no aniversário das pessoas que sai por R\$ 18,00 com rodízio, cerveja, refrigerante, buffet, sorvete e pinga a vontade. Ao lado do Universo Massas. Endereço: Av. Romeu Tórtima, 165. Telefones: (19)3289-1484 e (19)3289-7869.
\end{itemize}

\begin{itemize}
\item  \textbf{Star Clean:} É o bar mais próximo à UNICAMP, e por isso está sempre cheio. Principal ponto de encontro depois da aula e tem um bom preço.
\end{itemize}

\begin{itemize}
\item  \textbf{Star Trash:} Do lado do Star Clean, conhecido como Star Trash, Raimundo Raimundão, ou ainda, Star Trailer, é um trailer que serve cerveja e lanches baratos.
\end{itemize}

\begin{itemize}
\item  \textbf{Subway:} Lanchonete. Vende dos mais variados tipos de lanches. Lanches muito bons, mas caros. Localiza-se no Tilli Center (avenida Albino José Barbosa de Oliveira, 1556, esquina com a avenida 2). Do lado do Subway tem um caixa 24 horas que trabalha com os principais bancos.
\end{itemize}

\begin{itemize}
\item  \textbf{Temakeria:} Lugar relativamente novo, meio caro. Vende só temaki e bebidas. O horário de funcionamento é bastante conveniente. Endereço: Av. Romeu Tórtima, 1259 (relativamente próximo à Unicamp, um pouco pra cima do Bardana). Telefone: (19) 3289-0802. Horário de funcionamento: domingo a terça das 11h30 às 0h, quarta a sábado das 11h30 às 6h. Site: \url{http://www.tmkr.com.br/}
\end{itemize}

\begin{itemize}
\item  \textbf{Universo Massas:} Rodízio de massas perto do terminal. Bom e não é caro. De domingo à noite é o horário mais barato e dá pra encher bem o bucho de massa. Depois de ir até lá, você não vai querer saber de comer massas por um bom tempo. Endereço: Avenida Albino José Barbosa de Oliveira, 576. Telefone: (19) 3289-5369.
\end{itemize}

\begin{itemize}
\item  \textbf{Vila Ré -- Pizza:} Pizzaria próxima do terminal e do supermercado Dalben. Tem alguns sabores diferentes, as pizzas são boas e o preço não é alto. Possui serviço de entrega das 18h às 23h. Endereço: Avenida Albino José Barbosa de Oliveira, 658. Telefone: (19) 3289-0319
\end{itemize}

\begin{itemize}
\item  \textbf{Zé Espaço e Bar:} Barato, mas bem pequeno. Vende chopp Heineken por R\$2,50 de terça a sexta, até às 20h. Localiza-se também na avenida Albino José de Oliveira, bem em frente ao Pão de Açúcar.
\end{itemize}
