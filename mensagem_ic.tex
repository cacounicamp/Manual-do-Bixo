% Este arquivo .tex será incluído no arquivo .tex principal. Não é preciso
% declarar nenhum cabeçalho

\section{Mensagem do Diretor do IC}

Prezado ingressante,

Parabéns pela conquista de uma vaga em uma das mais renomadas universidades do
país e bem-vindo à vida universitária!

O Instituto de Computação (IC) será uma de suas referências importantes nos
próximos cinco anos. O IC já possui uma ``longa vida'' na história da Unicamp. Em
1968, o Prof. Rubens Murillo Marques, então Diretor do IMECC, propôs a criação
de um curso de Computação na Unicamp. ``Ciência da Computação? Que negócio
é esse?'' perguntou o Prof. Zeferino Vaz, Reitor desde os primórdios da Unicamp
até a sua aposentadoria em 1978, ao que o Prof. Murillo respondeu, de forma
sucinta: ``É o futuro, Zeferino!''.

Em 1969 foi criado o Curso de Bacharelado em Ciência da Computação juntamente
com o Departamento de Ciência da Computação (DCC) no então Instituto de
Matemática, Estatística e Ciência da Computação (IMECC). Ao longo dos anos,
o DCC foi se fortalecendo. Juntamente com tal fortalecimento também começou
a nascer e amadurecer o anseio do DCC em se tornar uma unidade independente.
A saída do IMECC foi negociada em todas as instâncias decisórias da Unicamp e,
finalmente, em 1996, o Instituto foi formalmente criado.

Você agora faz parte do ``futuro'' vislumbrado pelo Prof. Murillo, da trajetória
do IC e da história da Unicamp! O IC tem por objetivo formar lideranças na área
de Computação. A preparação de bons profissionais, contudo, requer muito estudo
e trabalho. Aulas bem como exercícios e projetos são recursos para auxiliá-lo na
reconstrução do conhecimento acumulado na área nas últimas décadas e em franca
expansão. A aquisição do conhecimento se dá através da transformação de
informações fornecidas em sala, disponibilizadas nas bibliotecas da Unicamp e na
internet bem como através da construção de artefatos, como código e modelos, que
visam uma melhor fixação de novos conhecimentos.

O sistema universitário difere daquilo que você já viveu no Ensino Médio. A sua
vida acadêmica não é planejada em grande detalhe e acompanhada de perto.
É delegado ao aluno a responsabilidade de gerenciar o seu tempo e organizar as
suas atividades estudantis. No início, tudo parece ser fácil e aparentemente as
cobranças são poucas. Ledo engano! As cobranças, quando ocorrem, são duras. Você
será tratado como adulto e, como tal, as suas responsabilidades são muito
maiores do que na sua vida pregressa. Ninguém vai correr atrás de você para
cobrar estudo ou a realização de atividades extra-classe. A falta de compreensão
dessas discrepâncias entre o Ensino Superior e o Ensino Médio resulta em muitas
reprovações no primeiro semestre, um número ainda significativo no segundo
e decrescendo conforme os alunos vão entendendo melhor como as ``coisas
funcionam''.

A recomendação que posso dar a você para se tornar um profissional de primeira
linha é que você mantenha os estudos em dia e procure ajuda tão logo as
dificuldades apareçam. Não deixe as dúvidas se avolumarem, pois elas aumentam de
forma vertiginosa. A Unicamp provê muitos mecanismos de apoio tais como
monitoria e atendimento docente. Você também pode e deve procurar o Coordenador
de seu curso se você não conseguir resolver os seus problemas acadêmicos ou não
junto a monitores e docentes. Em última instância, procure a Diretoria do
Instituto.

O IC quer formar com qualidade o maior número possível de profissionais, mas
você precisa fazer a sua parte.  A Unicamp não se resume só ao IC. Relacione-se
com o maior número de pessoas de áreas diferentes para, assim, aumentar o seu
``capital social'': participe de atividades interdisciplinares e de extensão
universitária, atue nas associações estudantis e nos diferentes colegiados que
definem os rumos da Universidade. A teia de relacionamentos que você estabelecer
durante a sua passagem na Unicamp será de grande valia e poderá propulsionar
muito sua futura carreira profissional.

Bem-vindo, então, à comunidade do IC e espero que você viva uma rica
e proveitosa experiência acadêmica!

Hans Liesenberg

Diretor do IC