% Este arquivo .tex será incluído no arquivo .tex principal. Não é preciso
% declarar nenhum cabeçalho
\section{Mensagem da FEEC}

Prezados ingressantes, alunos e alunas, do curso de Engenharia de Computação,\\

Parabéns pela sua conquista! É com muita alegria que lhes acolhemos na UNICAMP
e, em especial, na Faculdade de Engenharia Elétrica e de Computação (FEEC). A
vida universitária é uma fase muito especial de nossas vidas: alguns poucos
anos, tão intensos quanto breves, mas que costumam ser determinantes para
nossas escolhas, para que forjemos nosso modo de agir, pensar e ver o mundo.
Nesta carta de boas-vindas, gostaria de lhes falar brevemente sobre alguns
temas que considero importante para vocês: dar-lhes a conhecer um pouco da FEEC
e de sua história; comentar sobre o excelente curso que agora começam; e tecer
algumas reflexões sobre a expectativa que a sociedade coloca nas pessoas que,
como nós, temos ou tivemos o privilégio de fazer um curso de excelência numa
universidade pública da qualidade da UNICAMP.

Pode-se dizer que a FEEC começou oficialmente suas atividades acadêmicas no
início de 1967, quando ingressou a primeira turma de Engenharia Elétrica da
UNICAMP. Desde então, nossa Escola cresceu em pessoal, recursos e prestígio,
consolidando-se como referência e liderança, tanto no ensino de graduação como
de pós-graduação, ambos fortemente alicerçados na excelência de nossa atividade
em pesquisa. Temos a felicidade de poder contar com um corpo docente de
primeira linha, no qual convivem, em sinergia, a experiência de vários
professores que praticamente começaram a FEEC com o dinamismo de jovens
docentes, recentemente contratados. Contamos com funcionários dedicados e
comprometidos, alguns dos quais especialmente envolvidos com o ensino de
graduação. Temos também uma infraestrutura que, embora constantemente
necessitada de melhorias, lhes dará condições adequadas de estudo, tanto
teórico como em laboratório. Mas, sobretudo, sabemos que contamos com os
melhores estudantes. A partir de agora, vocês também fazem parte do principal
patrimônio de nossa Faculdade, que é nosso corpo discente.

O curso de Engenharia de Computação teve início em 1990, surgindo como uma
consequência natural do bom nível de atividades que já realizávamos, à época,
nesta área, e das necessidades de mercado de uma sociedade que começava a
orientar-se intensamente para as tecnologias digitais. É um curso que já nasceu
com o selo da excelência e da exigência. Compartilhamos este curso com os
colegas do Instituto de Computação (IC) da UNICAMP, unidade de ensino e
pesquisa do mais alto prestígio, na qual vocês também encontrarão um corpo
docente extremamente qualificado. Como as demais engenharias, é um curso que
requer uma base forte, de matemática e física. É importante aproveitar ao
máximo esses primeiros semestres de curso básico, sem se deixar abater por
dificuldades que são naturais, sem perder o ''brilho nos olhos`` desses
primeiros dias de UNICAMP. Problemas sempre existem e os professores, os
coordenadores de curso, assim como a Diretoria, tanto da FEEC como do IC,
estarão sempre à disposição de vocês, prontos para ouvir e remediar qualquer
situação que possa lhes afetar. Na FEEC vocês podem contar também com o apoio
específico do ''Espaço de Acolhimento ao Estudante`` (EAE-FEEC); informem-se
com a coordenação de curso ou com seus colegas veteranos. A Engenharia de
Computação é um curso exigente, mas também intelectualmente estimulante e bem
estruturado. Não deixem de colocar todo o esforço, e acudir às ajudas que forem
necessárias, para leva-lo a termo com sucesso e motivação constante.

Juntamente com os estudos, vocês descobrirão a vida universitária. A
Universidade é também um lugar de cultura e de debates e, sobretudo, oferece
oportunidades únicas para se fazer amizades para a vida. Desejo que aproveitem
muito bem cada instante de convivência, que participem com empenho e alegria
das atividades e das entidades estudantis, nas quais vocês descobrirão um
imenso leque de opções para contribuir com a universidade e, a partir dela, com
o país. Mas desejo igualmente que não percam o foco no essencial, que é a
própria formação, de modo a não deixar arrefecer seus ideais, nem frustrar as
expectativas que agora não são só de seus familiares, mas de toda a sociedade.
A universidade pública, passa por momentos difíceis. Nossas atividades,
gratuitas e de excelência, se sustentam graças ao trabalho de milhões de
cidadãos. Esta sociedade tem direito a nos cobrar eficiência, dedicação e
qualidade; a nós, gestores e professores, mas também aos estudantes, cujo
comprometimento ético deve se pautar pela dedicação ao aprendizado e por
fomentar o desejo de saber sempre mais, qualificando-se assim para, no futuro,
por meio de seu trabalho profissional, dar o justo retorno a quem nos financia.

Eu termino com uma citação que gosto muito, é de um autor clássico da
antiguidade grega, Píndaro, que num de seus versos dizia: ``torna-te aquilo que
tu és''. É um chamado do poeta para que o leitor tome consciência de quem é, de
aonde está, e saia assim de um possível momento de alienação ou prostração.
Vocês são hoje estudantes ingressantes do curso de Engenharia de Computação da
UNICAMP. Não é pouca coisa! São certamente orgulho para seus familiares e agora
para nós também. Desejo sinceramente que tenham esta realidade sempre presente
ao longo dos anos em que estiverem aqui. Sejam bem-vindos, sejam bem-vindas à
FEEC e, sobretudo, sejam muito felizes aqui conosco.\\

\begin{flushright}
João Marcos Travassos Romano

Diretor da FEEC
\end{flushright}
