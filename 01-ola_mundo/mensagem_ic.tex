% Este arquivo .tex será incluído no arquivo .tex principal. Não é preciso
% declarar nenhum cabeçalho

\section{Mensagem do IC}

Olá,\\

Cabe-me parte da tarefa de recepcionar sua chegada ao Instituto de Computação
(IC) e parabenizar esta importante conquista da sua vida.

Em 2019 celebraremos 50 anos de Computação na Unicamp, um dos cursos mais
antigos do Brasil e que sempre foi reconhecido como um dos melhores. Este
reconhecimento não acontece sem o ingresso de alunos e alunas de imensa
capacidade e da formação sólida e avançada fornecida por professores(as) com
significativa atuação acadêmica e científica. Agora se inicia uma nova fase de
suas vidas: um novo local, novos desafios e novas amizades. Tudo isto dentro
das mesmas 24 horas por dia. Aprimorar sua capacidade de organização será
essencial para aproveitar oportunidades como Centro Acadêmico, Atlética,
Empresa Júnior, iniciação científica, estágio, palestras e competições
científicas, que permearão seu tempo nos próximos anos.

Vou contar um segredo para vocês: nos discursos de formatura, sempre é dito que
“a última prova fácil foi o vestibular”. Vocês serão desafiados em todos os
semestres por matérias que formarão a base da sua futura carreira. Certamente
algumas serão bem mais difíceis que outras e, se você chegou até aqui,
acreditamos que seja capaz de supera-las. Sempre que enfrentar dificuldades,
lembre-se de olhar para os lados, dentro da mesma turma, para os monitores(as),
professores(as) e até mesmo para os(as) coordenadores(as) de seu curso.

Aproveite o momento para pensar nas suas vidas. Seu ingresso numa universidade
de nível mundial permite resultados impressionantes. A cada semestre, reserve
um tempo para planejar qual será sua marca na vida dos que o cercam, dentro e
fora da sala de aula. No mundo de hoje, “fora da sala de aula” representa muito
mais que a Unicamp. Você é capaz e deve aproveitar esta excelência e causar um
grande impacto positivo no mundo.

Sempre que eu puder ajudar em alguma coisa, meu e-mail é
\email{rodolfo@ic.unicamp.br} e minha sala é a número 55 do IC.\\


\begin{flushright}
Rodolfo Azevedo

Diretor do IC
\end{flushright}
