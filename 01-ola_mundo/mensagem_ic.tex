% Este arquivo .tex será incluído no arquivo .tex principal. Não é preciso
% declarar nenhum cabeçalho

\section{Mensagem do IC}

Car*s Aluno*s,

Sejam bem-vind*s à Unicamp e ao Instituto de Computação (IC).

Parabenizo a tod*s vocês pela conquista em estarem nesta instituição. No ano de
2014, o IC celebrou os 45 anos da Computação na Unicamp, um dos cursos mais
antigos do Brasil. Vale destacar que, desde sempre em sua trajetória, os nossos
cursos são reconhecidos dentre os melhores.  Trata-se de cursos de excelência
que reúnem com muita frequência *s melhores alun*s de várias escolas do Brasil e
que contam com professores com significativa atuação acadêmica e científica.

Neste momento se inicia uma nova etapa na vida de cada um(a): novo ambiente, novos
amigos, novos desafios, novas possibilidades; tudo isso acompanhado de mais
liberdade, independência e responsabilidade.  O ambiente universitário, vocês
verão, é extremamente fértil; oferece não apenas oportunidades acadêmicas de
qualidade (como aulas, palestras, mini-cursos, visitas técnicas e científicas,
estágios e atividades de iniciação científica), como também um universo de
experiências extracurriculares (Atlética, Centro Acadêmico, Bateria Valorosa,
Empresa Júnior, feiras de recrutamento, olimpíadas e maratonas de programação,
etc.) que possibilitam momentos inspiradores para aquisição de conhecimento,
maturidade e aperfeiçoamentos pessoal e profissional. Muitas destas atividades
extracurriculares são oferecidas ao mesmo tempo e não são raras as situações em
que conflitam com as atividades acadêmicas regulares. O principal desafio é
aprender a escolher e planejar, a gerenciar o tempo e os interesses, a cuidar
das relações pessoais, a considerar os prós e contras. Estejam atentos...

O que tenho para dizer a vocês, de ainda maior relevância, é algo que
erroneamente deixamos para o final da trajetória acadêmica quando, na verdade,
deveria ser a motivação de todas as nossas ações e que deveria permear todas as
nossas escolhas:

\begin{enumerate}
\item Sejam autor*s de suas próprias histórias;

\item Deixem vir à superfície a melhor versão de vocês mesm*s;

\item Guiem-se pelos preceitos do Bem e do Belo e os defendam com a convicção de
  estarem colaborando pela alegria de se viver;

\item Não se acostumem nunca com aquilo que gostariam de mudar, com o que
  gostariam de romper, com o que lhes conduz para fora de seus caminhos no
  encontro consigo mesm*s.
\end{enumerate}

Isso é o que devemos dizer para vocês, car*s alun*s, a todo instante durante
seus percursos nesses anos de universidade; é isso o que devemos dizer sempre a
nós mesm*s enquanto docentes, orientador*s, pesquisador*s, cidadãos, cidadãs, enfim, como
pessoas. Proponho um desafio: que tod*s nós nos engajemos em perseguir e
converter em atos essas palavras; proponho que cada um(a) de nós nos compromissamos
a transpor a preguiça, a desesperança, o medo, a incompreensão, a rivalidade sem
sentido, etc..., e busquemos realizar com responsabilidade o que nos faz
felizes, simplesmente. E que esse estado de contentamento irradie.

Parabéns a tod*s pela trajetória que se segue a partir daqui.

\begin{flushright}
Ricardo da Silva Torres

Ex-diretor do IC

\end{flushright}
