% Documento do tipo report (tese, dissertações, relatórios) em A4
\documentclass[a4paper,oneside,10pt]{memoir}


% Permite a escrita com caracteres UTF-8
\usepackage[utf8]{inputenc}
% Documento com "Capítulo", "Seção" escrito em português
\usepackage[brazil]{babel}
% Indentar o primeiro parágrafo após seções
\usepackage{indentfirst}
% Adicionamos enumerações
\usepackage{enumitem}
% Adicionamos links
\usepackage{hyperref}
% 2 colunas para listas
\usepackage{multicol}
% Fonte Pagella
\usepackage{tgpagella}

\hypersetup{
  %bookmarks=true,
  pdftitle={Manual de edição do Manual d* Bix* - \the\year},
  pdfauthor={Centro Acadêmico da Computação - CACo},
  hidelinks
}


% Título e autores do documento, sem mostrar data
\title{Manual de Edição do Manual d* Bix*}
% Os autores foram escritos manualmente, pois não cabia na página. Pular linha
% não funcionou
\author{
  Centro Acadêmico da Computação da Unicamp -- CACo
  \and
  Rafael Sartori Martins dos Santos
  \and
  Henrique Noronha Facioli
}
\date{\the\year}


% Iniciamos o documento
\begin{document}

% Definimos o título da página
\thispdfpagelabel{Capa}
% Incluímos o título
\maketitle

\frontmatter
% Adicionamos um sumário
\tableofcontents

\chapter{Prefácio}

Como você já viu, o Manual d* Bix* é uma das tradições do CACo que tem um
grande impacto, em especial por conter diversas informações úteis para quem
está chegando na Unicamp, além disso, é um ótimo jeito de apresentar o centro
acadêmico. Nosso manual é, de certa forma, uma referência na Unicamp por ter um
conteúdo bem atualizado e completo, mas isso tem um custo: temos que revisá-lo
todo ano. Nossa tática é concentrar o trabalho de edição no fim do ano,
começando entre outubro e novembro (na calmaria, antes do caos das provas
finais) e dando um gás em dezembro, logo após o fim das aulas.

É fácil perceber que isso não é um trabalho simples e requer a participação em
peso da gestão, mas, ao mesmo tempo, por utilizar \texttt{git} e \LaTeX, isso
pode ser um tanto quanto impeditivo para a participação de novas pessoas que
não tem tanta familiaridade com essas ferramentas. Por isso, aqui tento deixar
um guia de como começar a colaborar. Tornando assim mais a fácil a participação
de outros, além de ter uma forma de registrar as coisas que funcionaram bem no
passado.

Consideração importante: \texttt{git} é uma ferramenta muito poderosa e, como
sabemos, com grande poder, vem grandes responsabilidades. Fazer caquinha é
\textbf{MUITO} fácil em \texttt{git}, por isso tenha cuidado e pesquise tudo
antes de executar um comando. A parte ruim dessa ferramenta é que os tutoriais
e a documentação, apesar de rigorosos e claros, não são didáticos, então boa
sorte! Espero que este manual ajude a entender ao menos o que você deve fazer e
sirva como um guia para os vários nomes que eu tive que descobrir com várias
horas de pesquisas.

Justamente por ser uma ferramenta muito poderosa, há muitos caminhos para fazer
diversas coisas. O caminho mostrado aqui foi o que eu conheci e achei mais
didático. Muito provavelmente existe um caminho mais fácil para fazer o que fiz
e, com certeza, há um caminho mais difícil, então use o manual para aprender a
colaborar, mas não use como guia definitivo de \texttt{git}. No entanto, para o
melhor aproveitamento, veja todo o manual, pois irá conhecer mais recursos dessa
ferramenta.

Nós, membros do CACo, agradecemos a colaboração! Espero que goste do manual.

\mainmatter

% Incluímos o manual ;)
\chapter{Preparando para edição}

\section{Instalação de ferramentas}

No caso do Linux, além do seu editor de texto preferido, precisará de alguns
pacotes para compilar o manual (e a maioria dos projetos) em \LaTeX:

\begin{multicols}{2}
\begin{itemize}[noitemsep] % tirar espaço entre os itens
\item \texttt{git}
\item \texttt{scons}
\item \texttt{texlive-bin}
\item \texttt{texlive-core}
\item \texttt{texlive-pictures}
\item \texttt{texlive-latexextra}
\item \texttt{texlive-bibtexextra}
\item \texttt{texlive-science}
\end{itemize}
\end{multicols}

Crie, se não possuir, uma conta no GitHub (lembre-se de que, se você adicionar
algum e-mail da Unicamp -- seja como secundário ou principal --, poderá
adquirir uma conta de desenvolvedor, que permite repositórios privados).

\section{Configuração de projeto}

Em GitHub, a rede social de compartilhamento de código com fundação na
ferramenta \texttt{git}, há algo como a clonagem de projetos para a sua conta,
assim você pode fazer coisas que para você talvez pareçam complexas, como pedir
para seu código ser anexado ao projeto principal ou ainda criar um projeto
paralelo com base em outro já existente. Este processo de clonagem chama-se
\emph{forking}.

Abra a \href{https://github.com/cacounicamp/Manual-do-Bixo/}{página do manual no
GitHub} e crie uma \emph{fork} do manual na sua conta (clique em \emph{“Fork”}
no topo da página). Após ser redirecionado ao repositório -- nome dado ao
conjunto de arquivos do seu projeto, seja local ou remoto -- na sua conta, copie
o endereço dado em \emph{“Clone or download”}. Abra, então, o terminal, navegue
até a pasta que gostaria de possuir a pasta do manual e utilize o comando:

\begin{center}
\texttt{git clone (endereço copiado)}
\end{center}

O download do repositório irá começar (e irá demorar um pouco). O clone feito
com o endereço do seu repositório irá configurar um \emph{“remote”}, que é uma
ligação entre o repositório local e o repositório remoto (na nuvem do GitHub,
no nosso caso). O \texttt{git} irá enviar/receber informações, sincronizando o
projeto quando executar certos comandos usando essa ligação.

Você pode verificar os \emph{remotes} disponíveis utilizando o comando:
\begin{center}
\texttt{git remote -v}, onde \texttt{-v}, \texttt{--verbose} é usado para
mostrar detalhes
\end{center}

Notará que há apenas a ligação para o seu repositório -- nomeada por conven\-ção
de \emph{origin}. O \emph{link} para o qualquer outro \emph{fork} ou para o
repositório remoto original será necessário para a colaboração no projeto
comunitário. Por convenção, adicionaremos o ponto zero do projeto, o repositório
do CACo, com o nome de \emph{upstream}. Vá até a
\href{https://github.com/cacounicamp/Manual-do-Bixo/}{página do projeto no
GitHub do CACo}, pegue o endereço do repositório através do \emph{“Clone or
download”} e utilize o comando:

\begin{center}
\texttt{git remote add (nome) (endereço)}
\end{center}

\noindent Para o manual, usando HTTPS:
\begin{center}
\texttt{git remote add upstream
https://github.com/cacounicamp/Manual-do-Bixo.git}
\end{center}
Ou, usando SSH (é mais seguro e não precisa de senha a cada comando, porém
requer uma configuração adicional):
\begin{center}
\texttt{git remote add upstream git@github.com:cacounicamp/Manual-do-Bixo.git}
\end{center}

Com isso, será possível sincronizar com o projeto comunitário através da
ligação nomeada \emph{upstream}, já que o \emph{origin} refere-se apenas ao
\textbf{seu} repositório no GitHub, que não é o ponto central do projeto. Oh,
desculpe, não quis parecer rude, sua participação é importante. Se te faz
sentir melhor, você é o centro do nosso projeto como centro acadêmico.
\\

Agora você tem tudo pronto! Pode conferir o estado da sua \emph{branch} --
algo similar a uma ``linha do tempo'' -- com relação aos arquivos modificados
e sincronização com algum \emph{remote} usando:

\begin{center}
\texttt{git status}
\end{center}

\section{Conceitos importantes de \texttt{git}}

\subsection{O que é uma \emph{branch}?}

Uma \emph{branch} é algo criado a partir da ramificação do projeto que permite
alterá-lo independentemente de outros ramos. Por padrão, o ramo principal é
chamado de \emph{master}, aqui você aprenderá a fazer suas contribuições para o
manual através de \emph{branches} nomeadas de maneira breve com o fim de
descrever o que ela altera.

Imagine que a \emph{branch master}, o ramo principal do nosso projeto visto
como uma árvore, é o tronco. Imagine que cada centímetro de altura é uma
mudança no projeto. As ramificações criadas são galhos que saem de algum ponto
do tronco, ou seja, \emph{branches} precisam de um ponto de partida, uma altura
da qual o galho começa a crescer.

A partir do ponto mais recente do projeto, o topo do tronco, a alteração mais
recente da \emph{branch master}, é onde você (normalmente) cria as ramificações.
A partir delas, você pode fazer mudanças que não interferem no caminho principal
do projeto, o que é útil para desenvolvimento, teste ou mesmo alterações
experimentais. Isso é importante em projetos comunitários pois permite a
implementação, correção ou até remoção de alguma parte do projeto facilmente.

\subsection{Entendendo por que uma \emph{branch} se assemelha a uma ``linha do
tempo''}

Eu chamei \emph{branch} de linha do tempo pois, ao criar uma, você passa a
modificar seu trabalho independente das outras \emph{branches}, as suas
alterações são totalmente independentes do resto do projeto. Imagine um
aplicativo como o \emph{WhatsApp}. Imagine que os desenvolvedores querem fazer
algo decente finalmente: uma versão desktop que não depende do celular conectado
a todo momento (estou ignorando encriptação de ponta-a-ponta).

Imagine que usam a \emph{branch master} para a linha do tempo que irá para as
lojas de aplicativos, para os usuários. Eles criam a \emph{branch desktop-app},
por exemplo, para modificar o aplicativo de tal forma que permita você conectar
com os servidores a partir de um aplicativo de um computador (sem precisar do
celular). Eles desenvolvem essa \emph{feature} enquanto outros desenvolvedores
corrigem bugs, melhoram a performance, ou seja, sem interagir com a versão
desktop. Isso evita, ainda, a liberação para o público antes de tudo funcionar
perfeitamente, pois o desenvolvimento não será feito na \emph{branch master}.

Para imaginarmos a situação em que não usam \emph{branches}, todos os
desenvolvedores estariam alterando os mesmos arquivos juntos, misturando todo o
código antigo com o código de correção de bugs e melhora performance com o
código da nova \emph{feature} e, além disso, não poderiam liberar as correções
até absolutamente tudo estiver funcionando, já que, neste mundo, tudo estaria na
\emph{master} e não iriam liberar uma \emph{feature} quebrada. Reverter
alterações seria muito mais complexo sem ramificações, pois em \texttt{git} há
tal funcionalidade.
\\

Enfim, \emph{branches} permitem o desenvolvimento sustentável, colaboração
melhor organizada, um histórico de alterações limpo e mais.

\subsection{Entendendo a ideia de \emph{pull request}}

Espero que tenha entendido o que é uma \emph{branch}, pois iremos utilizá-la!
Já que o centro do manual fica na conta do GitHub do CACo, a ideia é que você
faça uma \emph{branch} apenas com a alteração direta do que quer fazer para
fazer uma \emph{pull request} no GitHub. Você já vai entender o que queremos
dizer.

Pode não ser o caso, mas imagine que pegou a versão impressa do Manual d* Bix*
do ano anterior e marcou com marca texto as partes que gostaria de alterar, ou
seja, você tem uma lista de coisas para alterar na versão deste ano, coisas que
podem estar na mesma categoria ou não. Para não precisar fazer 54 \emph{pull
requests} sobre cada coisinha que você quer alterar, mas, ao mesmo tempo, não
juntar tudo em apenas uma, tornando todas as alterações um pouco vagas e
impedindo que o projeto seja retrocedido caso precisasse, você precisa juntar
em mesma categoria as mudanças parecidas.

Por exemplo, tem 5 referências a restaurantes que não existem mais, a ideia da
\emph{pull request} é fazer a alteração direta e bem descrita em uma linha. No
nosso exemplo, a \emph{pull request} se chamaria ``Removendo restaurantes que
fecharam'' e a \emph{branch}, algo como \texttt{remover-restaurantes}. A
\emph{pull request} alteraria quantos arquivos precisarem, poderia haver
inúmeras alterações, mas abordaria todas as remoções de restaurantes que
fecharam entre este ano e o ano pasado, fazendo jus ao nome da \emph{branch} e
da \emph{pull request}.

O caso problemático seria se fizesse uma \emph{pull requests} para cada
restaurante removido, precisando da aprovação individual de cada uma para ser
aceito na \emph{branch master} do projeto. Ou ainda, se fizesse uma \emph{pull
request} para todas as 54 alterações, mas tivesse removido por engano um
restaurante que continua aberto, fazendo outra pessoa (ou você mesmo) ter que
reescrever a parte do manual, já que não seria prático reverter uma \emph{pull
request} tão grande.
\\

Ou seja, faça as alterações em conjunto, assim podem ser revertidas e
re-alteradas se precisar, sem a necessidade de movimentar uma quantidade
enor\-me de mudanças ou movimentar dezenas de vezes pequenas mudanças. Trata-se
de um equilíbrio.


\chapter{Modificando o manual}

\section{Preparando o seu repositório para mudanças}

Há alguns passos a seguir para evitar a desordem do seu repositório local. Essa
desordem pode te obrigar a recomeçar o projeto em uma outra pasta ou ainda
fazer uma complicada sequência de comandos. A primeira dica é não utilizar a
\emph{branch master}, deixe ela parada para ser uma cópia da \emph{branch
upstream/master} (esta é a notação usada para identificação de uma
\emph{branch} em algum \emph{remote}), assim você sempre terá o estado atual do
projeto comunitário sem precisar baixar mais nada.

\subsection{Sincronização com o projeto}

Antes de sair criando \emph{branches} para todas as suas 54 alterações, já que
você descobriu que é ruim fazer tudo na \emph{master}, é necessário uma
sincronização com o repositório do CACo, aquele cujo \emph{remote} é nomeado
\emph{upstream}. Isso deve ser feito no mínimo 1 vez: quando todas as alterações
tiverem sido realizadas, logo antes da \emph{pull request} ser criada, assim as
suas alterações se fundirão automaticamente, visto que a sincronização resolve
qualquer tipo de colisão que poderá enfrentar -- você entenderá o que são
colisões em breve.

Para realizar a sincronização, você tem algumas opções:

\begin{itemize}%[noitemsep]
\item Deixar a sua \emph{branch master} sincronizada com a
  \emph{upstream/master} e fazer qualquer nova \emph{branch} partindo da sua
  \emph{master} local.
\item Ignorar a existência da sua \emph{branch master} e fazer a nova
  \emph{branch} a partir da \emph{upstream/master}.
\item Misturar os dois, mantendo sua \emph{branch master} atualizada (caso você
  precise do PDF atual do manual, por exemplo), mas partindo da \emph{branch
  upstream/master} na criação da sua nova \emph{branch}, para ter certeza que
  sempre partirá do estado mais recente do projeto.
\end{itemize}

Recomendamos a última, visto que o trabalho de sincronização não é difícil
quando as mudanças não colidem. É recomendável que você esteja \textbf{SEMPRE}
sincronizado, evitando uma bola de neve de colisões. Mas, afinal, o que são
colisões?

Colisões ocorrem quando há alterações na \emph{upstream/master} em arquivos que
você já modificou. Quando o \texttt{git} não consegue solucioná-las
automaticamen\-te, você precisará fazer a fusão manualmente. Parece difícil,
mas, na maioria das vezes, basta adicionar as duas versões (a sua e a da
\emph{upstream}, já que geralmente são adições de parágrafos independentes),
raramente haverá um caso em que duas pessoas alteraram o mesmo parágrafo ao
mesmo tempo. Se acontecer, você precisará reescrever o parágrafo fazendo uma
fusão das ideias das duas modificações ou manter apenas uma ou outra versão. É
fácil.
\\

Agora que talvez tenha entendido a necessidade da sincronização, explicarei
como fazê-la. Basta executar o comando a seguir:

\begin{center}
\texttt{git pull (\emph{remote}) (\emph{branch})}
\end{center}

Ou, no nosso caso:

\begin{center}
\texttt{git pull upstream master}
\end{center}

Esse comando irá baixar (o mesmo que executar \texttt{git fetch upstream
mas\-ter}) e juntar os arquivos baixados com os arquivos locais verificando
alterações, identificando colisões (o mesmo que \texttt{git merge FETCH\_HEAD},
onde \emph{\texttt{FETCH\_\-HEAD}} é o nome que se dá ao estado do projeto
baixado, ou seja, é o nome do ponto da linha do tempo que foi baixada --
provavelmente não usaremos esses termos, utilize \texttt{git pull} que é mais
prático).

\subsection{Criando uma \emph{branch}}

Para criar uma \emph{branch}, é necessário um ponto de partida. Como disse
anteriormente, pode ser um ponto local ou não. Dê preferência ao
\emph{upstream/master}, já que é o estado atual do projeto comunitário. Usamos
o comando:

\begin{center}
\texttt{git branch (nome) \emph{remote}/\emph{branch}}
\end{center}

Temos vários exemplos:

\begin{center}
\texttt{git branch manual-de-edicao upstream/master}

\texttt{git branch manual-de-edicao origin/master}

\texttt{git branch manual-de-edicao master}, onde \emph{master} é a local
\end{center}

Note como é sutil a diferença do repositório ligado pelo \emph{remote origin} e
o repositório local. Em maioria, o repositório em nuvem estará desatualizado em
relação ao seu repositório local pois é você quem deve enviar as sincronizações
para ele. Ou seja, se o repositório comunitário (\emph{upstream}) fosse
atualizado e você tivesse sincronizado localmente, ainda precisaria enviar as
mudanças para o repositório na nuvem (logo verá como enviar alterações para
a nuvem).

Lembra-se que o comando \texttt{git status} mostrava o estado atual dos
arquivos e a \emph{branch} que estava ativa? Não se esqueça de mudar o estado
dos arquivos para a \emph{branch} desejada usando:

\begin{center}
\texttt{git checkout (branch)}
\end{center}

Se houver arquivos modificados, o comando irá falhar, pois mudar entre
\emph{branches} apagaria as mudanças que não foram guardadas através de
\emph{commits} (relatório de alterações em um instante de tempo, é similar a um
ponto na linha do tempo descrito por um breve texto). Você pode guardar as
mudanças numa pilha e restaurá-las quando quiser, seja no mesmo \emph{branch} ou
em outro, usando os comandos:

\begin{itemize}
\item Para guardar: \texttt{git stash} ou \texttt{git stash
  --include-untracked} para incluir os arquivos que não foram adicionados ao
  projeto do \texttt{git} como falarei no futuro.
\item Para restaurar: \texttt{git stash pop}, que colocará todos os arquivos
  no estado que foram guardados, independente da \emph{branch} que estiver ser
  ou não ser a mesma em que o \emph{stash} foi feito.
\end{itemize}

\section{Alterando de fato o manual}

\subsection{Modificando uma \emph{branch}}

Agora você sabe criar e pular de \emph{branch} em \emph{branch}, como um
verdadeiro macaco (não confundir com o movimento anti-CACo!). Basta, então,
modificar os arquivos e usar o comando \texttt{scons} na raiz do projeto para
compilar o PDF do manual e testar suas modificações. Verá que \LaTeX\, é uma
linguagem fácil de entender e escrever, ``ver e repetir'' é uma técnica que
funciona incrivelmente bem!

Leia as considerações importantes no \emph{README.md} no projeto do Manual d*
Bix* no GitHub! Respeite as convenções de edição, inclusive a de que toda regra
tem exceção se você achar necessário.

\subsection{Salvando as modificações}

Em \texttt{git}, apenas alguns arquivos são guardados como do projeto (arquivos
temporários, auxiliares ou de configuração são muitas vezes deixados de fora).
Arquivos importantes são chamados de \emph{staged files}, são aqueles
adicionados e modificados em cada \emph{commit}, aqueles baixados pelo GitHub,
considerados arquivos de fato ``\textbf{do} projeto''.

Para adicionar um arquivo novo ao projeto ou adicionar as modificações a uma
\emph{commit}, você utiliza o comando:

\begin{center}
\texttt{git add (arquivo 1) (arquivo 2) ... (arquivo n)}
\end{center}

Em cada \texttt{arquivo}, você pode utilizar um caminho como em \texttt{bash}.
Por exemplo:

\begin{center}
\texttt{src/*} irá adicionar todos os arquivos presentes na pasta \texttt{src}
como \emph{staged files}

\texttt{src/*.java} irá adicionar apenas os arquivos com o final \texttt{.java}
presentes na pasta \texttt{src}
\end{center}

Poderá verificar os arquivos adicionados/alterações consideradas como
\emph{staged files} usando o comando \texttt{git status}. Isso é útil antes de
criar a \emph{commit}, já que revertê-la no caso de um acidente é inconveniente
ao histórico de alterações e apagá-la é complexo (sobretudo se já está na
nuvem, no GitHub).

Renomear um arquivo pode ser algo complexo, pois \texttt{git} não perceberá tão
automaticamente que tal arquivo que foi removido é aproximadamente igual àquele
que foi adicionado, sobretudo quando foi alterado após renomeado. Para evitar
essas atrapalhadas, use para renomear:

\begin{center}
\texttt{git mv}
\end{center}

ao invés de utilizar o seu gerenciador de arquivos. Para remover algum arquivo
do conjunto de \emph{staged files}, utilize o comando:

\begin{center}
\texttt{git rm --cached}
\end{center}

Trocando \texttt{--cached} por \texttt{--force, -f}, o arquivo será apagado do
seu sistema.
\\

Finalmente, para criar uma \emph{commit}, guardando todas as alterações dos
arquivos adicionados em relação ao ``estado'' anterior da \emph{branch},
utilize o comando:

\begin{center}
\texttt{git commit}
\end{center}

O \texttt{git} abrirá o seu editor preferido (você pode mudar qual é usando as
configurações globais, pesquise como fazer se quiser) para escrever a mensagem
que descreve as alterações. O GitHub reconhecerá a primeira linha (dependendo
do seu comprimento) como o nome da \emph{commit}, as outras linhas só
aparecerão ao abrí-la no navegador e servem de descrição, comentários.

A \emph{commit} estará associada a \emph{branch} e guardará as diferenças dos
arquivos adicionados à \emph{commit} em relação ao estado anterior do projeto.
Recomendei o uso dos comandos de renomear e excluir do \texttt{git} pois para
adicionar um novo arquivo e apagar outro manualmente como \emph{staged file} é
um pouco mais trabalhoso.

\section{Enviando as alterações para a nuvem}

Agora que você já fez várias \emph{commits} e talvez tenha acabado tudo o que
sua \emph{branch} representava, está na hora de enviar para a nuvem! Para isso,
basta utilizar:

\begin{center}
\texttt{git push (\emph{remote}) (\emph{branch})}
\end{center}

Observação: é necessário para enviar à nuvem com sucesso caso a \emph{branch}
já exista, o \emph{merge} entre sua \emph{branch} local e a remota. Ou seja, é
necessário sincronizar os dois estados do projeto utilizando \texttt{git pull
(\emph{remote}) (\emph{branch})}. Isso só será necessário se houver alguma
mudança que não esteja no seu repositório local (você desenvolvendo em dois
lugares diferentes, outra pessoa aplicando um \emph{commit} são exemplos disso).

Apesar de talvez achar que é improvável você alterar várias \emph{branches}
simultaneamente, precisar enviar várias alterações juntas à nuvem é mais
comum do que parece. Um exemplo comum que acontecerá é o caso da
\emph{upstream/master} ter sido atualizada, onde é recomendável que você
sincronize todas as suas \emph{branches}, fazendo o que \texttt{git} chama de
\emph{merge}, o trabalho de juntar \emph{branches} em um ponto comum resolvendo
conflitos e criando uma \emph{commit} que unirá essas duas \emph{branches}.
Então, para facilitar e enviar as \emph{commits} de todas as \emph{branches} (ao
invés de mandar uma por uma), poderá usar o comando:

\begin{center}
\texttt{git push (remote) --all}
\end{center}

Se abrir seu GitHub após criar a \emph{branch} e enviar algumas \emph{commits},
verá que agora existe a \emph{branch} criada na lista e que há \emph{commits}
que não estão na fonte do seu projeto (``\emph{\texttt{x} commits ahead of
\texttt{cacounicamp/Manual-do-Bixo'}}''), ou seja, na \emph{upstream}. Haverá um
botão para criar uma \emph{pull request}, onde você pode descrever as mudanças
ou até pedir participação de outras pessoas se precisar, pois, assim como você
usou a \texttt{cacounicamp/Ma\-nual-Do-Bixo} como \emph{upstream}, fonte
do projeto, outra pessoa pode usar o seu repositório e participar de forma
recursiva.

Essa participação recursiva é útil para alterações grandes, como uma
re\-estruturação de alguma seção do manual. A pessoa que te ajudará configurará
da mesma forma que você o projeto \texttt{git}, porém haverá um \emph{remote}
para o seu projeto (assim como o seu possui um \emph{remote} para a página do
manual na conta do GitHub do CACo). Por conta do seu \emph{pull request} criado,
qualquer pessoa com um \emph{fork} do projeto poderá colaborar e enviar
\emph{commits} diretamente na sua \emph{branch}, como explicitado pelo GitHub na
página de qualquer \emph{pull request}: ``\emph{Add more commits by pushing to
the \texttt{(nome da branch)} branch on \texttt{(usuário)/Manual-do-Bixo}.}''
\\

Finalmente, após criada a \emph{pull request}, membros do CACo poderão revisar e
aprovar as modificações ao manual, verificando regras, se tudo foi compilado com
excelência através de uma métodologia rígida de \emph{try and error} que estamos
acostumados a fazer dentro do centro acadêmico.

\subsection{O que aprendemos até aqui}

Agora você aprendeu a sincronizar o repositório local com o remoto, seja ele
a \emph{upstream} ou o seu \emph{fork}, a \emph{origin}, seja ainda o de outra
pessoa (para colabora\-ção em conjunto), seja enviando ou recebendo informações.
Aprendeu também a criar \emph{branches} e alterá-las adicionando \emph{commits}.
Por fim, aprendeu a enviar as mudanças ao GitHub e criar \emph{pull requests},
que era o passo final para a participação individual no projeto.

No próximo capítulo, aprenderemos a limpar o seu repositório, local e remoto,
a \emph{origin}, após suas mudanças forem verificadas e aprovadas no projeto
principal. Veremos em outro ainda como participar em grupo mais profundamente
numa \emph{pull request}, que servirá também aos membros da gestão a manterem o
manual, ajudando os colaboradores com as regras.


\chapter{Cuidando do repositório após mudanças}

Antes de sair apagando \emph{branches} que já foram aceitas no projeto
principal, deixe-as abertas por um tempo. Caso algum problema estivesse passado
pela revisão dos mantenedores do projeto, a sua alteração poderá ser revertida
e revisada. A maneira mais rápida e fácil para arrumar o erro acontece se a
\emph{branch} estiver ainda ativa, pois bastará fazer algumas alterações para
tornar a \emph{pull request} válida novamente.

O que quero dizer é que você deve esperar alguns dias ou pelo menos algumas
horas até a sua alteração ser consolidada no projeto. A razão para isso é que,
no projeto do Manual d* Bix* do CACo, as \emph{commits} do \emph{pull request}
são resumidas em apenas uma (\emph{merge} usando \emph{squash}), tornando mais
limpo o histórico do projeto no geral, mas com o custo de desorganizar os
repositórios locais após terem suas mudanças aceitas (já que a \emph{branch}
possuirá um número diferente de \emph{commits} -- a não ser que você faça o
\emph{squash} localmente). Ou seja, após uma \emph{pull request} ter sido
aceita, a sua \emph{branch} se tornará incompatível com o projeto e não poderá
ser facilmente re-utilizada, senão com a reversão da \emph{pull request} em
casos de erros.

\section{Escrevendo a \emph{pull request}}

A \emph{pull request} deve conter um relatório breve de todas as informações que
foram alteradas, com os motivos para tais mudanças. Membros do CACo avaliarão o
resultado do projeto compilado, corrigindo (ou pedindo para que você corriga)
possíveis erros de margem (causado pelo dicionário do \LaTeX, que não possui
divisões silábicas para algumas palavrass), de compilação (como uma imagem
faltando, comando incorreto) ou até do \emph{status} do \texttt{git} (para toda
\emph{pull request} ser aceita, é necessário que as \emph{branches} sejam
compatíveis).

Se tudo ocorrer bem, sua \emph{pull request} será aceita e o membro do CACo fará
o GitHub executar um \emph{merge} provavelmente com \emph{squash}, que juntará
todas as alterações em apenas uma grande \emph{commit}.

Caso o \emph{squash} não for feito, a sua \emph{branch} poderá ser utilizada
normalmente (execute \texttt{git pull upstream master} para chegar ao mesmo
estado que a \texttt{ca\-co\-unicamp/master}). Caso for feito, você poderá excluir a
\emph{branch} se desejar, pois ela irá se tornar inútil.

\section{Excluindo a \emph{branch} local e remota}

Se já passou um tempo desde quando sua \emph{pull request} foi revisada e
aceita, podemos começar a limpar o seu repositório local. Primeiro, apagamos a
\emph{branch} que originou a \emph{pull request} utilizando o comando:

\begin{center}
\texttt{git branch -D (\emph{branch})}
\end{center}

A opção \texttt{-D} vem de \texttt{--delete --force}, você
precisa forçar a exclusão já que tornar as várias \emph{commits} em uma única
no repositório do CACo implicará na incompatibilidade da sua \emph{branch},
visto que \texttt{git} não reconhecerá que os arquivos estão identicos no
final, mas sim que a quantidade de \emph{commits} são diferentes.

Para enviar uma exclusão de \emph{branch} para o repositório remoto, utilize o
comando:

\begin{center}
\texttt{git push (\emph{remote}) -d (\emph{branch})}, onde \texttt{-d} significa
deletar
\end{center}


\chapter{Mantendo o repositório remoto e revisando colaborações}

A necessidade disso tudo (separar em \emph{branches}, fazer \emph{pull
requests}) já foi comentada. É uma maneira de desenvolver de forma organizada em
grupo, seja mantendo o código como membro do CACo ou ainda como colaborador. Por
mais que haja a necessidade de aprender uma nova ferramenta, é muito mais
organizado dp que permitir inúmeras alterações aleatórias de qualquer pessoa
(mesmo limitado apenas aos que querem colaborar) no Google Drive.

Nesta seção, além de aprender a participar de projetos comunitários, iremos
revisar e utilizar tudo que foi ensinado até agora. Trabalharemos em cima de uma
\emph{branch} que está sendo considerada em alguma \emph{pull request}.

\section{Configuração do projeto num ambiente comunitário}

É necessário, sobretudo como mantenedor do código, sendo membro do CACo que está
revisando o manual, adicionar diversos \emph{remotes} de \emph{forks} de outras
pessoas, além das inúmeras \emph{branches} de cada \emph{pull request}.

Como já vimos, adicionamos algum \emph{remote} utilizando o comando \texttt{git
re\-mo\-te add (nome do \emph{remote}) (endereço do \emph{remote}, HTTPS ou
SSH)} e o \\\texttt{git} irá reconhecê-lo através do comando \texttt{git fetch
(nome do \emph{remote})} (irá baixar todas as informações utilizando o endereço
dessa ligação).

Com essas informações baixadas do \emph{remote}, é possível mudar para a
\emph{branch} de qualquer \emph{fork} e colaborar em conjunto, corrigindo,
testando e complementando \emph{pull requests} de outros usuários.

\section{Colaboração com a comunidade}

Para mudar para a \emph{branch} de outro usuário, é necessário criar uma nova
\emph{branch} local com base no estado atual daquela no \emph{remote}. Utilize
\texttt{git branch (nome da \emph{branch}) (usuário)/(nome da \emph{branch})} e
mude o projeto para os arquivos dela usando \texttt{git checkout (nome da
\emph{branch})}.

Agora basta fazer as alterações, adicioná-las e criar o \emph{commit}.
\\

Para que outras pessoas tenham acesso, possam alterar, fazer uso ou simplesmente
ver suas alterações, é necessário envià-las à nuvem. Utilize o endere\-ço do
\emph{remote} (que é a base da \emph{pull request}) para enviar: \texttt{git
push (nome do \emph{re\-mo\-te}) (nome da \emph{branch})}. Isso enviará o estado
atual da sua \emph{branch} local para a nuvem, mas falhará caso haja alguma
divergência diferente da criação das suas \emph{commits}. Ou seja, falhará caso
não houver sincronização (caso outra pessoa fizer um \emph{commit} entre quando
que você usou \texttt{git pull} e quando usou \texttt{git commit}).

Para corrigir isso, antes de usar \texttt{git push}, utilize \texttt{git pull
(nome da \emph{re\-mo\-te}) (nome da \emph{branch})}, pois isso mesclará a sua
\emph{branch} local com a remota, garantindo sincronia. O mesmo deve ser feito
com a \emph{branch master} do projeto principal antes da \emph{pull request} ser
aceita, como dito antes.
\\

Resumindo: é necessário ter certeza de que os projetos locais estão de
a\-cor\-do com os remotos antes de tentar enviar o estado atual do local para o
remoto, seja para finalizar a \emph{pull request}, seja para o simples envio de
novas \emph{commits} à qualquer \emph{branch} em qualquer \emph{remote}.


\end{document}
