\section{Mulheres no Conhecimento}

Quantes cientistas você conhece? E quantas são mulheres? Temos nomes de homens cientistas bem conhecidos, como Stephen Hawking, Carl Sagan, Albert Einstein... mas e as mulheres?

Dá pra perceber o fato de que o acesso ao conhecimento em geral, na cultura ocidental em que estamos inseridos, foi historicamente garantido aos homens e afastado das mulheres. Desde o início da Ciência Moderna, que é o que conhecemos como ciência, as mulheres têm um papel de coadjuvante, sendo impedidas de estudar ou sendo somente ajudantes de seus maridos cientistas.

A partir disto foi construído um imaginário da mulher irracional, submissa, que era incapaz de ser coerente, que deveria ficar em casa. Foi construída uma diferenciação da capacidade feminina e masculina e as mulheres não tiveram espaço para se desenvolver.

Mesmo com essa realidade tão rígida, diversas mulheres já iam contra os princípios de sua época, estudando escondido. Dá pra ver que pode ser difícil, mas, em todas as épocas, as mulheres resistiram. Resistiram a um papel definido, a julgamentos, foram contra uma sociedade inteira que desejava oprimí-las.

Com o desenvolvimento da ciência, conseguimos perceber que as mulheres estão presentes nas áreas exatas e tecnológicas há tanto tempo quanto os homens.

\subsection{Algumas das Grandes Mulheres na Ciência}

% Hipátia de Alexandria
\noindent
\begin{minipage}{0.2\textwidth}
    \includegraphics[width=\textwidth]{placeholder.jpg} % Replace "placeholder.jpg" with your image
\end{minipage}
\hfill
\begin{minipage}{0.75\textwidth}
    \textbf{Hipátia de Alexandria} \\
    Nascida no ano de 370, Hipátia de Alexandria (370-415) foi a primeira mulher da história a ser conhecida por ser matemática.
\end{minipage}

\vspace{1em}

% Ada Lovelace
\noindent
\begin{minipage}{0.2\textwidth}
    \includegraphics[width=\textwidth]{placeholder.jpg} % Replace "placeholder.jpg" with your image
\end{minipage}
\hfill
\begin{minipage}{0.75\textwidth}
    \textbf{Ada Lovelace} \\
    Um exemplo clássico da computação pra inspirar essa nova jornada que você tá começando: Ada Lovelace (1815-1852) escreveu o primeiro algoritmo do mundo!
\end{minipage}

\vspace{1em}

% Marie Curie
\noindent
\begin{minipage}{0.2\textwidth}
    \includegraphics[width=\textwidth]{placeholder.jpg} % Replace "placeholder.jpg" with your image
\end{minipage}
\hfill
\begin{minipage}{0.75\textwidth}
    \textbf{Marie Curie} \\
    Marie Curie (1867-1934) fez descobertas sobre radioatividade e foi a primeira mulher a receber o prêmio Nobel, em Física, em 1903. Também foi a primeira pessoa a recebê-lo duas vezes: ganhou também o Nobel de Química, em 1911.
\end{minipage}

\vspace{1em}

% Katherine Johnson
\noindent
\begin{minipage}{0.2\textwidth}
    \includegraphics[width=\textwidth]{placeholder.jpg} % Replace "placeholder.jpg" with your image
\end{minipage}
\hfill
\begin{minipage}{0.75\textwidth}
    \textbf{Katherine Johnson} \\
    Tida como "computador humano", Katherine (1918-2020) foi liderança técnica na Agência Espacial dos EUA (NASA) cujas contribuições foram essenciais para a missão Apollo, que enviou o homem à Lua, entre outros projetos. Foi retratada no filme Estrelas Além do Tempo.
\end{minipage}

\vspace{1em}

% Katie Bouman
\noindent
\begin{minipage}{0.2\textwidth}
    \includegraphics[width=\textwidth]{placeholder.jpg} % Replace "placeholder.jpg" with your image
\end{minipage}
\hfill
\begin{minipage}{0.75\textwidth}
    \textbf{Katie Bouman} \\
    Para completar esta lista, deixaremos registrada Katie Bouman (1989-), que recentemente foi responsável pela primeira foto real de um buraco negro, no projeto EHT (Event Horizon Telescope). Ela criou o algoritmo CHIRP, que combina os dados de oito telescópios ao redor do mundo e produz a foto.
\end{minipage}
