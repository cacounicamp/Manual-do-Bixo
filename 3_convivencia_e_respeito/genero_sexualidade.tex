\section{Gênero e Sexualidade}

Muitas vezes, a entrada no Ensino Superior coincide com o início da vida adulta. Por esse motivo, frequentemente a graduação é repleta de descobertas das liberdades que vêm com a vida adulta. Isso significa que você, ingressante, pode ter um novo mundo a descobrir, experimentar e, enfim, encontrar as respostas para aquelas questões que vêm indagando a sua curiosidade desde o início da sua adolescência.

Como essas coisas funcionam? Como eu funciono? O que quero? Todo mundo já se perguntou ou vai se perguntar essas coisas, independentemente da orientação sexual ou identidade de gênero!

Muito disso vem na forma que a vida universitária permite que você exerça sua liberdade sexual, já que está se tornando uma pessoa que é dona de seu próprio nariz, entre pessoas que também o são.

Nós apostamos que conversar sobre o assunto é muito melhor do que deixar virar um tabu (e abandonar as pessoas para navegar esse alto mar sozinhas, sem informação alguma). Vamos conversar?

\subsection{”Opção sexual” e ”Ideologia de gênero”}

\textbf{Vamos fazer um teste:}

\begin{itemize}
    \item Se você é hétere, responda rápido: quando escolheu ser hétere?
    \item Se você é LGBTQIAPN+, quando escolheu sua sexualidade?
    \item Se você "mudar de ideia”, poderá optar por outro apetite e atração sexual?
\end{itemize}

A essa altura, é fácil compreender que “opção” sexual não existe. A sexualidade faz parte de todo ser humano, está no nosso ser biológico, e vai simplesmente se manifestar. Chamamos essa manifestação específica da atração sexual de orientação sexual.

\textbf{Agora, outro teste:}

\begin{itemize}
    \item Se você é homem, quando ganhou suas primeiras peças de roupa azuis e carrinhos de brinquedo?
    \item Se você é mulher, quando ganhou o primeiro kit de panelinhas, e bonecas para cuidar como se fossem filhas?
\end{itemize}

É uma raridade infâncias não serem afetadas por nenhum tipo de estereotipação, infelizmente.

Tem coisas que não têm a ver com a sexualidade, mas que a sociedade associa. Por exemplo, nenhum santo disse, nem faz sentido acreditar que "naturalmente”, por conta de seu gênero, uma mulher não possa ter interesse em carros e se tornar engenheira. Do mesmo jeito, ninguém disse que homens não devam se preparar para cuidar de seus filhos ou cozinhar. Mas, conforme vamos crescendo, vamos nos acostumando com o nosso entorno e incorporando como "natural" que homem tem um tipo de comportamento pré-estabelecido e mulher tem outro, igualmente pré-estabelecido.

O mesmo se aplica às roupas e sapatos que vestimos, o tamanho do cabelo, ou se alguém gosta ou não de usar maquiagem. Nada disso está no seu DNA, tudo é criado pelos seres humanos. Quando nascemos e olhamos em volta, procuramos nos espelhar.

E isso é violento: se eu mexer com "coisas de outro gênero”, serei discriminade ou violentade?

Veja como isso não é "natural”, mas construído socialmente: na Escócia, uma outra sociedade, é comum que homens usem uma saia chamada kilt. Claro que só veste o kilt quem gosta dessa tradição. Se você respeita um escocês de saias e entende que isso não faz dele "menos homem”, também pode respeitar um brasileiro que veste saias porque gosta - e entender que isso não tem a ver com a sexualidade dele.

O fato do gênero ser socialmente construído não faz dele menos real. Alguns adeptes a extremismos afirmam que gênero é uma "ideologia" sendo "passada para nossas crianças”. Uma "ameaça às famílias” pois "convence homens a serem mulherzinhas” e mulheres a "virar macho”. Pura ignorância: as pessoas têm experiências e convivências muito reais e importantes com essa construção social, conforme verá adiante.

\subsection{Falando de identidade de gênero}

Durante um tempão, o protagonismo da sigla LGBTQIAPN+ (Lésbicas, Gays, Bissexuais, Travestis, transexuais e Transgêneros, Queers e Questionando, Intersexuais, Assexuais, Pansexuais e Não Binaries e todos + que tiverem comportamento diferente dessas poucas letras) ficou muito restrito ao discurso da sexualidade, então uma parcela significante da sigla foi deixada de lado, enquanto se avançava em direitos e em consciência acerca de outros, chegando ao ponto em que os dois assunto – identidade de gênero e sexualidade – misturavam-se na mente das pessoas. É muito importante diferenciar esses dois assuntos, para podermos entender essas pessoas e respeitá-las como se deve.

Enquanto a sexualidade se limita às suas relações com as outras pessoas, a identidade de gênero diz respeito a como você se enxerga no mundo, e expressa suas individualidades. Aí que entra nossa letra T da sigla, que designa pessoas transexuais, transgêneros ou travestis; nos três casos, estamos nos referindo a pessoas que não se identificam com o gênero que os foi imposto ao nascerem, sendo a única diferença entre esses termos quem os fala: o termo transexual se refere a um tempo em que sexo e gênero eram vistos como a mesma coisa, mas ele se refere às mesmas pessoas que "transgênero”, já travesti se refere a pessoas trans marcadas por classe e marginalização, normalmente está associada a pessoas periféricas de maior segregação social.

Existem pessoas trans de diversos tipos e que se expressam de formas variadas, algumas delas estão dentro da binaridade homem/mulher, outras transitam entre essas duas identidades ou não se identificam com nenhuma delas, como as pessoas não-binárias. Em todos os casos, essas pessoas precisam enfrentar um gama de preconceitos e segregação inimaginável. Segundo o IBGE, o Brasil é o país que mais mata pessoas trans no mundo e cerca de 90\% dessas pessoas acabam recorrendo à prostituição, tanto devido à vida familiar que, na maioria dos casos, expulsam essas pessoas de casa quando ainda jovens, quanto devido a um mercado de trabalho trans excludente - isso explica o número ínfimo de pessoas trans em cargos de destaque na sociedade. Em todos os casos, a transfobia internalizada em nossa sociedade simplesmente não permite que essas pessoas vivam; por isso que precisamos dar voz a essas pessoas, ouvindo-as sempre e tentando sempre entender suas dores, evitando preconceitos e conversas que você não teria com pessoas cisgênero. E \textbf{sempre respeitando a forma como a pessoas quer ser tratada} - como seus pronomes e nome social. Não sabe os pronomes de alguém? Pergunte!

\begin{figure}[htbp!]
    \centering
    \includegraphics[width=0.5\linewidth]{placeholder.jpg}
    % \caption{Lina Pereira dos Santos, mais conhecida como Linn da Quebrada. Cantora, compositora, atriz, ativista social e travesti. Utiliza os pronomes ela/dela. • Elliot Page. Atore, diretore, produtore canadense e transgênero não-binárie. Indicade ao Globo de Ouro e ao Oscar por sue papel como protagonista do filme Juno. Utiliza os pronomes ele/dele e elu/delu.}
    \label{fig:enter-label}
\end{figure}

\subsection{O que ocorre na prática}

Apesar de não se tratar de uma escolha, algumas orientações sexuais e identidades de gênero são privilegiadas em relação a outras na sociedade: é o caso dos heterossexuais (pessoas atraídas pelo gênero oposto) e dos cisgêneros. A sigla LGBTQIAPNPN+ representa a todes que não são heterossexuais ou cisgêneros e que, por não estarem de acordo com o padrão social, geralmente são oprimides e desprivilegiades, alvos de preconceito.
Lado a lado com o machismo, a LGBTfobia é passada institucionalmente de geração em geração e, juntos, eles enfraquecem todes quanto à sua constituição sexual – criando vários problemas sociais. Há uma disparidade gigantesca na condição de emprego, com salários menores, maus tratos e assédios em local de trabalho, entre outras situações.

O caso de transexuais é ainda mais preocupante: os piores empregos são os que "sobram”, quando existem, levando muites a recorrer ao arriscado trabalho sexual como último recurso para não cair na miséria. Essas pessoas também sofrem mais violência verbal e violência física: o Brasil é o país que mais mata trans e travestis no mundo. Foram pelo menos 140 assassinatos em 2021, 33\% das mortes mundiais desses grupos [1] – um número que sabemos ser subnotificado, pois 98,8\% das ocorrências não possuem informações sobre a identidade de gênero das pessoas e a redução do número de casos nesse ano em alguns estados ocorreu sem ações específicas do Estado [2].\\
{
\footnotesize
\color{gray}
\textit{
    \\
    \texttt{[1]} Fonte: Trans Murder Monitoring \\
    \texttt{[2]} Fonte: Dossiê da Associação Nacional de Travestis e Transsexuais sobre Assassinatos de pessoas trans em 2021.\\
}
}
\par Portanto, exercer a sua liberdade sexual pode não ser tão fácil se você não estiver no grupo hétero/cis. Felizmente, na Unicamp, há espaços mais seguros para você ser o que você é. A história de luta LGBTQIAPN+ fez do ambiente universitário, hoje, um espaço em que você pode ter voz e proteção independentemente da sua orientação sexual ou identidade de gênero. Dentro da universidade, há alguns coletivos LGBTQIAPN+ (e mais alguns em formação!) dedicados a manter e ampliar as conquistas, defendendo pessoas LGBTQIAPN+ em situações de opressão. Se você é uma pessoa LGBTQIAPN+ e se sentiu constrangide por isso em qualquer situação, não hesite em procurar seu centro acadêmico ou um desses coletivos para assegurar sua proteção e de demais colegas.

\subsection{Virando o jogo}

E se você é uma pessoa hetero e cis, você pode agir de forma respeitosa com sues colegues LGBTQIAPN+, defendendo-es quando perceber uma situação opressora, usando sua voz e seu privilégio para ajudar e não fazendo ”””piadas””” que constrangem alguém. É muito importante tomar um lado e proteger pessoas que sofrem preconceitos. Quem se "isenta” diante de uma opressão, já tomou o lado do opressor.

Talvez você tenha dúvidas sobre a sua orientação sexual, porque se sente mais atraíde por um gênero que pelo outro embora se atraia por ambos, ou porque seus interesses afetivos são por um gênero mas seu interesse sexual é maior com outro, ou até mesmo porque seu desempenho sexual com cada gênero é diferente. Lembre-se que não é necessário colocar sua sexualidade em uma caixa! Apenas algumas situações vividas não definem quem você é e a sexualidade é fluida. Permita-se se descobrir, com respeito e consentimento SEMPRE.

Pra fechar essa seção, é importante você entender que refletir sobre esses assuntos é essencial para evoluirmos na nossa área e na sociedade! A tecnologia é para todes, para avançarmos como sociedade, ultrapassarmos os limites que temos hoje e nos tornarmos cada vez mais conscientes, livres e capazes de realizar as coisas no mundo que vivemos. Pensando nisso, precisamos nos perguntar para que pessoas essa nossa sociedade está sendo construída: para todes mesmo ou só para algumas pessoas? Essa pergunta pode nos guiar, e nos fazer refletir!

Sempre bom olhar para o lado e vermos quem são as pessoas compartilham os espaços conosco. Por que só elas, e não outras pessoas, estão ali? Que pessoas tem acesso ao que existe e as inovações que vamos criar? E, se são sempre os mesmos grupos, com as mesmas características, pensarmos "por quê?".

Lembrando também que as mudanças acontecem aos poucos, dentro de cada pessoa que reflete. Contamos com você, pra refletir e construir; para criar ambientes para todes!

Se quiser conversar mais sobre isso, não hesite, chame a gente do CACo! c:

\begin{figure}
    \centering
    \includegraphics[width=0.5\linewidth]{placeholder,jpg}
    % \caption{Foto: Reprodução  •  Uma pena que este manual (em sua versão impressa) seja em tons de cinza e você não possa ver quantas cores essa imagem tem! Esta foto retrata a Parada do orgulho LGBT de São Paulo, um evento que acontece desde 1997 na Avenida Paulista. Segundo os organizadores, a edição de 2011 apresentou o maior número de participantes de sua história, tendo presentes estimados: 4 milhões de pessoas.}
    \label{fig:enter-label}
\end{figure}