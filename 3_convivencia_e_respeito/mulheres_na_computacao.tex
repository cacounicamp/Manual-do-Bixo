\section{Mulheres na Computação}
As minas na comp são o primeiro sinal de algo não está muito certo! Você vai perceber que terá poucas mulheres em cada uma das salas da comp; não dá pra ser coincidência né?

Bom, a primeira coisa é que existe uma diferenciação de tratamento, oportunidades e situações vivenciadas quando se é mulher. Isso se dá no dia a dia, no trabalho, nas amizades e em todos os âmbitos sociais que possuímos. Dito isso, você deve imaginar que essas características também se replicam sobre quem entra na Unicamp, se reproduzem na Unicamp, e, infelizmente, sobre o curso no qual você está entrando, a Computação.

Infelizmente é uma realidade e, nesta seção, nós vamos te explicar um pouco sobre isso e sobre como nós vamos mudá-la!

Não é simples lidar com essa questão, porque a ideia do que as mulheres devem ou não fazer, do que elas são boas ou não, do lugar que pertence a elas ou não, é algo estabelecido e repassado há muito tempo para nós.

A comp é um curso majoritariamente composto por homens e, como é da área tecnológica, já vem vinculado com essa ideia do "ser masculino", que "é" inteligente, racional e lógico. Isto acaba refletindo nas minas de maneira negativa, como se criasse uma atmosfera onde elas não tem espaço, não se sentem parte do meio e acabam tendo experiências ruins.

A ideia de que isso é "comum" vai se enraizando sutilmente e estruturalmente em todas as pessoas. Por isso, viemos aqui para levantar alguns pontos e fazer você refletir sobre alguns "padrões" que já existem e que, com certeza, não são legais.

\subsection{O que conhecemos como mulher}

Desde criança já existe um “padrão” em que estamos muito inseridos: meninas devem ser mais comportadas, educadas, sempre organizadas e limpas. Meninas devem se arrumar, cuidar da beleza, etc. Mulheres estão destinadas a serem mães e a casar; correm perigo, devem sempre estar com amigos ou namorado que poderão ajudar. Para os homens é construído o imaginário inverso: de liberdade e do indiví-duo que é capaz de tudo.

Por muito tempo, reproduzimos o que a sociedade dita. Portanto, antes mesmo de termos até mesmo um senso crítico e uma visão mais ampla da própria vida, já fo-mos criades com tais pensamentos. \textbf{Precisamos mais e mais entender e abraçar a existência de uma pluralidade, despida de esteriótipos. Meninas, meninos e menines igualmente incentivades à autonomia, à confiança, à criticidade e à liberdade.}

\subsection{A universidade reflete a sociedade}

A Unicamp não é uma bolha, e tudo que acontece lá fora também acontece aqui dentro. Listamos aqui algumas das situações corriqueiras que podem ser evitadas para que você possa refletir, entender como funciona e identificar comportamentos preconceituosos que só reproduzem os problemas do machismo. Dá uma olhada!

Um grande mal é que esses comportamentos impostos sobre as meninas, por vezes, fazem com que elas de fato acreditem que são menos ou que estão com dificuldades anormais. Temos que quebrar essa impressão que foi criada de que as mulheres não têm a capacidade de adaptação no espaço acadêmico, no esporte ou em qualquer outro espaço. As mulheres estão preparadas, sim, para todos os desafios que estão por vir. Podem se adaptar à novas rotinas e se desenvolver como qualquer pessoa! A comp também é delas!

Essas situações podem ser reproduzidas por qualquer pessoa diante de um gru-po minoritário e entre pessoas de grupos minoritários também. Sempre importante entendermos que estamos diante de ideias muito enraizadas e que podemos nos tornar mais conscientes disso ao refletirmos e ouvirmos o que elas têm a dizer.

\subsection{Pode isso, Marta?}
\todo{Adicionar imagem da marta}

Coisas corriqueiras que deveriam deixar de acontecer LOGO:

\begin{itemize}
	\item \textbf{Impressão de dificuldade.} É normal se sentir perdido numa turma quando não conhecemos ninguém ou viemos de outra dade. Mas isso é natural e não deve ser explicado por "ser menina". Portanto, não tem nada a ver achar que as minas tem dificuldades além do normal, e não é legal os colegas se sentirem "os salvadores" por ajudarem em algo.
	\item \textbf{Explicanismo.} Loucos para se sentirem os "salvadores", alguns caras tentam explicar coisas para as mulheres o tempo inteiro – até quando elas sabem mais do que eles sobre o assunto. Famoso palestrinha. Claro que se ajudar dentro da área é algo normal, e você deve fazer, mas devemos refletir se não estamos fazendo suposições e reforçando estereótipos.
	\item \textbf{Assédio sexual.} Quando qualquer pessoa diz não, ela não tá falando grego, sueco, sânscrito. É não. Se ela não estiver em condições de dizer nada, seja um ser humano e ajude-a. Se você queria ouvir um "sim", querer não é poder. Bola pra frente, você não vai morrer por isso e insistir não vai mudar nada. Respeite.
	\item \textbf{Objetificação.} Diferentemente do que mostram as TVs, revistas, sites porno-gráficos… mulheres não são objetos. Essa ideia cria "justificativas" para assédios.
	\item \textbf{Julgamento pela aparência.} É sem noção, mas existem comentários como: "menina da comp é feia", "menina de exatas é lésbica", "x cursos tem garotas bonitas". Nenhum estereótipo é saudável e reproduzir esse tipo de ideia é completamente babaca.
	\item \textbf{Silenciamento.} Quando uma mulher está falando e magicamente brotam conversas paralelas, tosses, olhares para o teto, assobios e até gente que corta uma frase no meio (!) só pra dizer em voz alta: "concordo!"... Todo mundo deve ter a oportunidade de passar suas ideias e ser entendide de forma completa; não há motivos para não deixar as mulheres fazerem isso.
	\item \textbf{A persistência do mito da "friendzone".} As relações com as mulheres não são batalhas campais em que se ganha território até captar uma bandeira de namoro ou sexo. A "zona da amizade" é um mito. Relacione-se com respeito.
\end{itemize}

\subsection{Política de Segurança Pública}

Outra consequência desse ciclo de comportamentos é que gera insegurança física para as meninas. Infelizmente, no caminho para casa e até dentro do campus, frequentemente ocorrem assédios. E isso tem um histórico: o programa de segurança interna Campus Tranquilo foi desestruturado com a demissão generalizada dos guardas da Funcamp e re-terceirização do serviço - o que pode acontecer também com es funcionáries dos bandejões, majoritariamente mulheres negras. Os cortes de verba fizeram com que algumas áreas tenham metade dos postes de luz desligados à noite, transformando a penumbra no "normal”.

Em festas e repúblicas também há casos de assédios. Para evitar essas situações e garantir a segurança das mulheres na universidade, a ARU (Associação de Repúblicas da Unicamp) e uma grande parte das festas têm \textbf{Comissões Acolhedoras e equipes de segurança} que tratam desses casos.

Além disso, você caloura, não deixe de contar conosco! Sua veteranas estão aqui para lhe apoiar e ajudar no que for preciso. Estamos, a cada ano que passa, fortalecendo mais e mais nossa própria rede de apoio às meninas do curso. Não queremos nos sentir caladas ou sozinhas! Não deslegitime seus próprios sentimentos se alguma situação lhe trouxer desconforto. Estamos aqui pra você e enfrentaremos isso juntas.

\begin{tcolorbox}
\textbf{EI!!!} Se você é homem e está pensando que é impossível se comportar diferente, ou
que isso tudo é "mimimi” ou um "exagero”, respire fundo! Comece ouvindo quem passa
todos os dias por essas situações. Exercite sua empatia e observe seu comportamento -
\textbf{ninguém está isente de reproduzir opressões estruturais} como o machismo. Queremos
um ambiente que seja confortável, onde todes se desenvolvam! Se comprometa com
isso! Toda ajuda é bem-vinda pra combater o machismo.
\end{tcolorbox}

\subsection{\small Algumas formas de evitar e combater o machismo no dia a dia da universidade}

\begin{itemize}
    \item \textbf{Não coloque as mulheres em caixinhas.} Cada pessoa tem seu jeito de ser e estudar, e as meninas tão cansadas de serem taxadas de "organizadas" mas não exatamente "inteligentes”. Conheça as pessoas de verdade, sem criar definições antes
    
    \item \textbf{Troque conhecimentos com meninas.} Será que você considera suas colegas de sala inteligentes? Ou acha um amigo mais esperto e rápido de pegar as coisas? Será que é isso mesmo ou na verdade é uma impressão antecipada? Cada pessoa tem suas facilidades e dificuldades dependendo da disciplina.

    \item \textbf{Troque ideia sobre computação.} Você consegue ter amizade com as meninas da sua sala? Consegue falar de tecnologia e trocar ideias sobre o curso? Todas as meninas da comp, as deste ano e as veteranas, escolheram fazer computação. Elas sabem porque estão aqui, e com certeza podem te contar. Vamos sair deste imaginário de que mina tá perdida quando entra no curso.

    \item \textbf{Suponha sempre que as pessoas sabem o mesmo que você.} Não seja aquela pessoa que acha que sabe tudo! Quando for conversar com uma menina ou qualquer pessoa, não baseie a conversa em tentar explicar coisas que você sabe e ela não. Todo mundo tem uma bagagem, uma história pra estar aqui e fazer comp. Trocar ideia, conhecer as pessoas de verdade é bem mais legal! Procure falar o que você sabe e também estar disposto a ouvir o que as pessoas sabem. Não subestime as mulheres!

    \item \textbf{Participe de eventos sobre o tema.} Todo ano há diversos eventos, de todos os tipos e gostos, para fortalecimento da pauta feminista e combate às diversas atitudes machistas que ainda existem. Todes podem e devem participar!
\end{itemize}