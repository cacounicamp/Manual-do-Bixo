\section{Pessoas Negras na Computação}

\subsection{Muito o que ser feito}
O Brasil, após 300 anos de escravidão, foi o último país da América a abolir a escra-
vidão negra formalmente, em 1888. Atualmente, depois de mais de um século, temos
enraizado no inconsciente coletivo da sociedade um pensamento que marginaliza
pessoas negras e as impede de se constituírem como cidadãs plenas. E o racismo está
presente nessa naturalização de ações, hábitos, situações, falas e pensamentos que
promovem, direta ou indiretamente, a segregação e/ou o preconceito racial. Um processo
que atinge tão duramente e diariamente a população negra.

De forma macro, está confirmado e demonstrado que pessoas negras possuem sua
vivência e acesso fragilizados, refletindo em dados negativos: a taxa de homicídio é
maior para os homens negros\footnote{De acordo com a denominação do IBGE entendidos como pretos e pardos}. A população negra também tem o menor acesso a
condições salariais melhores. Nas universidades, incluindo nossa querida Unicamp, é
visível que alunes e professores negres são a minoria nos diversos cursos, embora
componham mais da metade da população brasileira.

\subsection{Um projeto político}

O racismo presente na sociedade brasileira de hoje não é apenas um resquício do
passado, uma mera coincidência ou descuido. O racismo é um projeto político e inclui,
entre outras coisas, o apagamento da cultura negra, a perpetuação da desigualdade
racial e o extermínio da população negra nas periferias. Esse projeto talvez nunca tenha
sido tão bem representado por um governo como no governo de Jair Bolsonaro.

Ele construiu sua campanha a partir de um discurso de ódio que incentivou ataques
racistas por todo o país e governou realizando o desmonte de políticas sociais que
atingiam principalmente pessoas negras, além de defender a impunidade dos policiais
que praticam o extermínio dessa população. Não poderíamos deixar de mencionar que o
ex-presidente já se declarou diversas vezes contrário às cotas, ignorando toda a
discussão e os demonstrados avanços, com desempenho de cotistas igual ou superior ao
restante de estudantes, que as cotas étnico-raciais trouxeram\footnote{Políticas de inclusão têm resultados positivos nas universidades. Exame, 2017.}.

Por esses e por outros motivos, devemos entender o racismo como um problema
estrutural e não podemos nos deixar enganar pelo ex-vice-presidente Mourão que, ao
ser questionado sobre o caso do homem negro que foi espancado até a morte em um
supermercado Carrefour, afirmou que não existe racismo no Brasil. Afinal, uma das
táticas tradicionais do racismo é negar a sua própria existência.

\subsection{Luta é o início de um reconhecimento}

Percebida há décadas, essa situação tem gerado diversas mobilizações pautadas nas
diferentes problemáticas.

Movimentos como o "Vidas Negras Importam” expõem a violência policial e as altas
taxas de mortalidade de pessoas negras vitimadas pelo homicídio. Diversos coletivos,
grupos e páginas exaltam a beleza de pessoas negras em seus tons, traços e texturas de
cabelo - uma movimentação lindamente herdada do Partido dos Panteras Negras. Essas
iniciativas têm trazido maior representatividade e identidade ao povo negro, que sempre
teve nas suas características físicas um alvo de preconceito - quantos já não ouviram que
cabelo black é cabelo "ruim”?

As lutas e mobilizações cresceram tanto que ganharam até espaço nas mídias
mainstream, que historicamente contribuíram (e contribuem) para invisibilizar o povo
negro. Uma vitória parcial conquistada com muito suor! A maior abertura e reconheci-
mento vem com mais âncoras de jornal, influencers negres e histórias protagonizadas,
atuadas, dirigidas e criadas por negres.

Esse maior reconhecimento também pode ser percebido nas premiações: Pantera
Negra quebrou recordes, recebendo 3 oscars; a revolucionária animação Homem-Aranha
no Aranhaverso recebeu premiações; a luta anti-racista baseada em história real de
Infiltrado na Klan (com uma denúncia pertinente e corajosa ao ex-governo
estadunidense de Donald Trump) finalmente rendeu o reconhecimento merecido ao
diretor negro Spike Lee; recentemente, o curta Hair Love, que mostra a relação de afeto
entre um pai e sua filha negra sendo pautada pelo seu cabelo crespo, recebeu um Oscar.

\begin{figure}[htbp!]
    \centering
    \includegraphics[width=0.75\linewidth]{placeholder.jpg}
    % \caption{Na imagem algumas referências de pessoas pretas na cultura pop: 1. Filme "Pantera Negra"; 2. Animação "Homem-Aranha no Aranhaverso; 3. Filme "Infiltrados na Klan"; 4. Curta-metragem "Hair Love"}
    \label{fig:enter-label}
\end{figure}

\subsection{Nossa Universidade}
A Unicamp tem tido avanços, muito recentes, com a implementação de cotas. De fato,
foi uma das últimas do país a aderir à reparação histórica. Em 2016, dado um corte de
R\$ 40 milhões que colocou em risco a própria existência de alguns cursos já bastante
sucateados, uma assembleia de 1200 estudantes deliberou greve, ocupando a reitoria no
mesmo dia.

Rapidamente, a maioria dos institutos aderiu à pauta: contra os cortes orçamentários, por cotas e pela ampliação da moradia estudantil (veja a seção Permanência).

O fruto dessa luta, que ainda teve que enfrentar uma grande resistência institucional na votação do CONSU, culminou na aprovação de cotas étnico-raciais para negres e na criação do vestibular indígena. A USP, isolada entre as 3 estaduais paulistas e sob uma grande luta des estudantes, seguiu o exemplo e também aderiu às cotas logo depois. Nós, da Computação, participamos deste processo com 2 paralisações e votamos, por ampla maioria, apoio às cotas!

Cá estamos, em 2025, com as cotas implementadas, mas até hoje temos governos e uma crescente em grupos que agem ativamente para reverter os avanços. 

Além disso, deve-se ter a clareza de que os problemas des cotistas não terminam assim que entram na universidade. Muitas pessoas, por várias questões, têm a sua permanência comprometida e todes acabam tendo que enfrentar o racismo que ainda rasteja no ambiente universitário, através de piadas e comentários desrespeitosos, ou que se exibem descaradamente através de ataques diretos.

Um importante exemplo desses ataques ocorreu no ano passado, quando um professor da Faculdade de Engenharia Mecânica interrompeu agressivamente uma aula de Tópicos Especiais em Antropologia que exibia um RAP dos Racionais MC’s, chamou a música de barulho e silenciou a professora que ministrava a disciplina, uma mulher negra\footnote{Professor se “irrita” com Racionais MCs na UNICAMP e é acusado de racismo. Fórum, 2017}.
Nesse contexto devemos lembrar da frase de Angela Davis, filósofa, professora e uma das grandes militantes do Partido Comunista dos Estados Unidos (PCUSA) e do Partido dos Panteras Negras para Auto-Defesa: Numa sociedade racista, não basta não ser racista. É necessário ser antirracista! Façamos, então, nossa parte para sermos cada vez mais antirracistas. Que comentários e "piadas” de cunho racista sejam repreendidos. Que as características físicas de pessoas negras sejam respeitadas. Que qualquer situação vexatória seja reportada e tratada pela comunidade.

E vocês, nossos cares colegas negres, contem conosco!