% Este arquivo .tex será incluído no arquivo .tex principal. Não é preciso
% declarar nenhum cabeçalho

\section{Segurança}

Todos sabemos que o mundo tem estado muito perigoso, entretanto é importante ter
a atenção dobrada em Barão Geraldo. Por isso, expomos aqui algumas ferramentas
que você pode utilizar em situações de perigo ou para se previnir.

\subsection{Campus Tranquilo}

O serviço Campus Tranquilo é uma ronda de vigilância que circula dentro da
Unicamp. O objetivo deles inicialmente é proteger o patrimonio da universidade,
entretanto já se mostraram prestativos diversas vezes e podem te ajudar em uma
situação de risco. Além disso, eles possuem serviço de escolta no período
noturno.

\begin{itemize}
\item \textbf{Telefone:} (19) 3289-4453
\item \textbf{Site:} \url{bit.ly/campus-tranquilo}
\end{itemize}


\subsection{Botão do Pânico}

O Botão do Pânico é um app disponibilizado pela própria Unicamp, que tem como
objetivo registrar situações de pânico dentro do campus. O app está disponível
tanto na Play Store quanto na App Store, sendo que os credenciais para login
são: seu RA e a sua senha da DAC. O ideal é já fazer o download do aplicativo e
fazer o login, pois numa situação de risco não haverá tempo para isso.

O aplicativo utiliza sua localização para pedir emergencialmente a presença de
algum responsável da segurança da Unicamp.

\subsection{Grupo da Unicamp no Facebook}

O grupo da Unicamp no Facebook tem vários objetivos, como integralização e
diversão, entretanto também pode ser utilizado para espalhar informações. De vez
em quando rola no grupo algumas publicações sobre uma situação de risco que
ocorreu em Barão Geraldo, explicando o que aconteceu e onde especificamente
aconteceu. Por isso, mantenha-se atenta!
