% Este arquivo .tex será incluído no arquivo .tex principal. Não é preciso
% declarar nenhum cabeçalho

\section{Segurança}

Todos sabemos que o mundo tem estado muito perigoso, entretanto é importante ter a atenção dobrada em Barão Geraldo. Por isso, expomos aqui de algumas ferramentas que você pode utilizar em situações de perigo ou para se previnir.

\subsection{Segurança no Campus}

O Serviço de Segurança no Campus é uma ronda que circula dentro da Unicamp. O objetivo deles inicialmente é proteger o patrimonio da universidade, entretanto já se mostraram prestativos diversas vezes e podem te ajudar em uma situação de risco.

\subsection{Botão do Pânico}

O Botão do Pânico é um app disponibilizado pela própria Unicamp, que tem como objetivo registrar situações de pânico dentro do campus. O app está disponível tanto na Play Store quanto na App Store, sendo que o login é seu RA, e a senha é sua senha da DAC. O ideal é já fazer o download do aplicativo e fazer o login, pois numa situação de risco não haverá tempo para isso.
Link para Android: \url{https://play.google.com/store/apps/details?id=br.unicamp.ccuec.botaopanico}
Link para Apple: \url{https://itunes.apple.com/br/app/bot%C3%A3o-do-p%C3%A2nico/id1037064894?mt=8}

\subsection{Grupo da Unicamp no Facebook}

O grupo da Unicamp no Facebook tem vários objetivos, como integralização e diversão, entretanto também pode ser utilizado para espalhar informações. De vez em quando rola no grupo alguns posts sobre uma situação de risco que ocorreu em Barão Geraldo, explicando o que aconteceu e onde especificamente aconteceu. Por isso, mantenha-se atenta.
