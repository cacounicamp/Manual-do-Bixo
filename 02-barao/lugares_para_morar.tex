% Este arquivo .tex será incluído no arquivo .tex principal. Não é preciso
% declarar nenhum cabeçalho

\section{Lugares para morar}

Iremos te preparar para encontrar o lugar que mais combina com você ou, pelo
menos, evitar alguns erros como pagar o suficiente para uma casa para dividir
um quarto em república.

O custo de moradia em Barão Geraldo depende principalmente de três fatores:
proximidade da Unicamp, tamanho do imóvel e qualidade da casa (acabamento,
número de banheiros, presença de piscina etc.). Quanto à distância, os entornos
da \textbf{Avenida 1} (Avenida Doutor Romeu Tórtima) e da \textbf{Avenida 2}
(Avenida Professor Atílio Martini) costumam ser mais caros de se morar, por
serem próximos da Unicamp. A região que vai do centro de Barão Geraldo até a
Moradia Estudantil, um pouco mais distante da universidade (cerca de 10 minutos
de bicicleta), é em geral mais barata e concentra muito mais serviços, como
supermercados, restaurantes e bancos.

Uma boa dica para se informar a respeito de lugares para morar (repúblicas,
quitinetes, pensionatos) é o site \textbf{Morar Unicamp}
(\url{morarunicamp.com.br}), criado por estudantes. Lá você encontra
informações como endereço, preço, contato e detalhamento do lugar.

E, bixete, você não terá aulas na FEEC antes do segundo ano (se for engenheira
e seguir a proposta de cumprimento de currículo) e poucas aulas serão no IC, a
maioria delas serão ministradas no PB e no CB -- você logo saberá o que
significam essas siglas, continue lendo --, então não se preocupe procurar onde
morar perto da FEEC ou do IC.

Nós realizamos, em 2017, uma pesquisa para trazer informações atualizadas para
você sobre os gastos com alimentação, entretenimento, aluguel e com a vida de
modo geral para ajudá-la a verificar quais valores são cobrados em Barão
Geraldo. É importante ressaltar que não trabalhamos estatisticamente com esses
dados, apenas mostramos qual o valor mínimo e máximo, qual a média, e os
valores mais prováveis. Obtivemos mais de 200 respostas para uma dúzia de
perguntas, podemos afirmar que pelo menos 7 valores fazem a média que verá
aqui.

\subsection{Moradia Estudantil}

A \textbf{Moradia} é um exemplo de conquista de estudantes. O processo de
reivindicação de uma moradia estudantil para a Unicamp começou com o movimento
Taba. Durante dois anos, alunos e alunas ficaram acampados no CB (Ciclo Básico)
até que as obras começassem. Hoje em dia, graças à Moradia, várias pessoas que
não teriam condições de se manter em Campinas pagando aluguel podem estudar na
Unicamp.

\begin{figure}[h!]
  \centering
  \includegraphics[width=.45\textwidth]{img/barao/moradia.jpg}
\end{figure}

A Moradia existe desde 1989. De lá para cá, o número de vagas em cursos da
Unicamp aumentou muito e as vagas da Moradia tornaram-se insuficientes para
acomodar todos que precisam. A reivindicação de mais vagas para a Moradia é uma
das principais bandeiras do DCE e um desejo de muitas estudantes.

Cada casa da Moradia, normalmente dividida por quatro pessoas, constitui-se de
um quarto, uma cozinha, um banheiro e uma sala. Há ainda o Ônibus da Moradia,
um circular da Unicamp que transporta pessoas durante o dia todo da Moradia até
a Unicamp e vice-versa.

A Moradia está localizada na Avenida Santa Isabel, 1125, a cerca de 3 km do
campus.

Para saber mais sobre a Moradia e o processo seletivo, entre no site
\url{www.pme.unicamp.br}.

\subsection{Repúblicas}

A melhor escolha se você tiver condições de pagar por uma moradia e gosta de
companhia.

\begin{figure}[h!]
  \centering
  \includegraphics[width=.45\textwidth]{img/barao/republica.jpg}
\end{figure}

Você pode levar quem quiser para sua casa, chegar no horário que bem entender,
além do que você conhecerá muita gente nova. Se possível, more em uma república
de cursos mistos, pois assim você terá contatos diversos. Cerca de 41\% das
pessoas que responderam a pesquisa sobre custo de vida em Barão Geraldo moram
em repúblicas.

O custo de uma vaga em república é bastante variável, depende principalmente do
número de pessoas com quem dividirá o quarto (ou até se é um quarto individual)
e da localização. Um quarto compartilhado tem um custo entre 500 e 800 reais,
sendo a média R\$ 606,36. Já um quarto individual, entre 500 e 1.000 reais,
com várias ocorrências de vários valores, ou seja, muito bem distribuído dentro
deste intervalo, com o custo médio R\$ 726,47. Esses valores já incluem as
despesas como água, luz, gás, internet etc.

Escolha bem as pessoas com quem você vai morar para não ter problemas com
diferentes estilos de vida. Tem gente que gosta de lavar louça a cada 5 minutos
e tem gente que usa o chão como lata de lixo tranquilamente. Veja com quem você
se dá melhor.

\subsection{Quitinetes}

Cuidado! A especulação imobiliária em Barão Geraldo chega a ser imbecil. As
quitinetes mobiliadas são normalmente compostas de um banheiro e um cômodo que
é quarto, sala, cozinha e área de serviço. Os valores de aluguel ficam entre
600 e 2.300 reais (bem distribuídos principalmente entre 800 e 1500 reais),
sendo a média R\$ 1.105,07. Sim, mil e cem reais por um microespaço. Só porque
é perto da Unicamp. Fique de olho, tome cuidado com os contratos. Cerca de 34\%
das pessoas que responderam a pesquisa sobre custo de vida em Barão Geraldo
moram em quitinetes. Ou seja, três quartos de quem participou da pesquisa moram
ou em quitinete, ou em república.

\begin{figure}[h!]
  \centering
  \includegraphics[width=.45\textwidth]{img/barao/quitinete.jpg}
\end{figure}

As melhores relações custo-benefício de quitinetes são as daquelas próximas ao
centro de Barão Geraldo ou no espaço entre as avenidas. E lembre-se de que não
é só de Unicamp que se vive. Não adianta pagar mais caro para estar do lado da
universidade se você fica muito longe dos supermercados, farmácias, padarias,
restaurantes etc.

\subsection{Pensionatos}

Pensionatos são como repúblicas, mas podem vir com regras. Muitas regras.
Dependendo do pensionato que você conseguir, pode tornar-se uma grande roubada.
Alguns não deixam você levar pessoas para casa, reclamam se você chegar tarde e
não liberam festas; outros, não, então procure bem.

O preço varia de acordo com as comodidades disponíveis. Um quarto compartilhado
custa entre 520 e 1.050 reais, sendo a média R\$ 848,00, enquanto um quarto
individual, entre 800 e 1.600 reais, o custo médio é R\$ 1.088,18. Pode ser uma
opção muito cômoda se você procura um conjunto de casa, comida e roupa lavada.
Cerca de 11\% das pessoas que responderam pesquisa sobre custo de vida em Barão
Geraldo moram em pensionatos.

É muito importante que você saiba que contratos de um ano (ou qualquer período)
em pensionatos são ilegais e você não precisa cumpri-los.

\subsection{Casa ou apartamento}

Cerca de 14\% das pessoas que participaram da pesquisa moram em casa ou
apartamento, o custo médio é R\$ 1.129,66, variando entre 650 e 3.600 reais.
No entanto, isso provavelmente não irá atrair muito as alunas e alunos de
graduação, mas de pós graduação. Também houve vários casos em que o aluguel era
dito como zero (que não entraram na média), significando que moravam com
parentes ou que haviam isenção de alguma espécie.

\subsection{Segurança}

Por ter muitas casas de famílias abastadas e de estudantes (em geral
desatentos), Barão Geraldo é grande alvo de assaltos a residências e, além
disso, o distrito peca pela falta de segurança.

Não é raro ouvir que alguém foi assaltado enquanto voltava para casa à noite
sem companhia ou que teve a casa saqueada durante um feriado prolongado. Mais
chocantes ainda são os casos de estupro que ocasionalmente são divulgados em
grupos de e-mail e redes sociais. Cuidado! É importante zelar pela sua
integridade e pela de seus pertences -- assim como seus pais fazem em sua casa,
não importa onde eles morem.

\begin{figure}[h!]
  \centering
  \includegraphics[width=.45\textwidth]{img/barao/seguranca.jpg}
\end{figure}

Evite andar sem companhia à noite, especialmente nos fins de semana. Se o seu
pensionato ou a sua república paga o segurança da rua, o que é altamente
recomendável, use \emph{sempre} de seus serviços, seja ligando para pedir
escolta ao chegar em casa ou para avisar caso ouça algum barulho suspeito.

Ao voltar para sua cidade em feriados prolongados, deixando a casa vazia, não
se esqueça de trancar todas as portas e janelas de casa, verificar se não há
nada no quintal que possa ser levado facilmente (colocar as bicicletas e
aparelhos de som na sala é uma boa ideia) e trancar os objetos de valor
(computadores, televisões) dentro dos quartos.

\subsection{Imobiliárias}

\begin{itemize}
\item \textbf{Imobiliária Barão Housing}
  \\Telefone: (19) 3289-4113
  \\Endereço: Rua Tranquilo Prosperi, 383
  \\E-mail: \email{atendimento@baraohousing.com.br}
  \\Site: \url{baraohousing.com.br}

\item \textbf{Imobiliária Lanza}
  \\Endereço: Rua Benedito Alves Aranha, 104
  \\Telefone: (19) 3289-1717
  \\WhatsApp: (19) 99805-2853
  \\E-mail: \email{lanza@lanzaimoveis.com.br}
  \\Site: \url{lanzaimoveis.com.br}

\item \textbf{Imobiliária Professor Sebastião}
  \\Endereço: Av. Dr. Romeu Tortima, 344
  \\Telefone: (19) 3289-2317
  \\E-mail: \email{ipsimoveis@ipsimoveis.com.br}
  \\Site: \url{ipsimoveis.com.br}

\item \textbf{Imobiliária Barão}
  \\Endereço: Rua Maria Tereza Dias da Silva, 224
  \\Telefone: (19) 4141-1010
  \\E-mail: \email{roberto@imobiliariabarao.com}
  \\Site: \url{imobiliariabarao.com}

\item \textbf{Amaral Imóveis}
  \\Endereço: Av. Dr. Luiz de Tella, 864
  \\Telefone: (19) 3254-4755
  \\WhatsApp: (19) 99960-4755
  \\E-mail: \email{amaral@amaralimoveis.net}
  \\Site: \url{amaralimoveis.net}

\item \textbf{Zaine Conquista Imóveis}
  \\Endereço: Av. Santa Isabel, 84
  \\Telefone: (19) 3289-4050 / (19) 3307-0433
  \\E-mail: \email{zaine@correionet.com.br}
  \\Site: \url{zaineconquista.com.br}

\item \textbf{Ismê Assessoria Imobiliária}
  \\Endereço: Rua Christina G. Miguel, 250
  \\Telefone: (19) 3289-4325
  \\E-mail: \email{isme@isme.com.br}
  \\Site: \url{isme.com.br}

\item \textbf{Rute Svartman Imóveis}
  \\Endereço: Rua Engenheiro Edward de Vita Godoy, 850
  \\Telefone: (19) 3368-0881
  \\E-mail: \email{imoveis@rutesvartman.com.br}
  \\Site: \url{rutesvartman.com.br}

\item \textbf{Imobiliária Ávila \& Ferraris}
  \\Endereço: Av. Dr. Romeu Tortima, 714
  \\Telefone: (19) 3289-3522
  \\E-mail: \email{dcaavila@terra.com.br}
  \\Site: \url{avilaeferrarisimoveis.com.br}

\item \textbf{Denilson Imóveis}
  \\Endereço: Av. Professor Atílio Martini, 55
  \\Telefone: (19) 3289-1444
  \\E-mail: \email{contato@denilsonimoveis.com.br}
  \\Site: \url{denilsonimoveis.com.br}

\item \textbf{Imobiliária Cidade Universitária}
  \\Endereço: Av. Dr. Romeu Tortima, 1101
  \\Telefone: (19) 3289-3322
  \\E-mail:\\
  \begin{small}
  \email{contato@cidadeuniversitariaimoveis.com.br}
  \end{small}
  \\Site: \url{cidadeuniversitariaimoveis.com.br}

\item \textbf{Mega Barão Imóveis}
  \\Endereço: Rua Ângelo Vicentim, 622
  \\Telefone: (19) 3289-7101 / (19) 3386-4141
  \\E-mail: \email{megabarao@megabaraoimoveis.com.br}
  \\Site: \url{megabaraoimoveis.com.br}

\item \textbf{Libano Imóveis}
  \\Endereço: Rua Benedito Alves Aranha, 270
  \\Telefone: (19) 3789-9999
  \\E-mail: \email{contato@libanoimoveis.com.br}
  \\Site: \url{libanoimoveis.com.br}

\item \textbf{Marco Imóveis}
  \\Endereço: Rua José Pugliesi Filho, 420
  \\Telefone: (19) 3287-8083
  \\Site: \url{marcoimovel.com.br}

\item \textbf{Lokal Imóveis}
  \\Endereço: Rua José Próspero Jacobucci, 290
  \\Telefone: (19) 3256-4616
  \\Site: \url{lokalimoveis.com.br}

\item \textbf{Carpe Diem Imóveis}
  \\Endereço: Av. Dr. Romeu Tortima, 184
  \\Telefone: (19) 3579-5655 / (19) 3304-9323
  \\Site: \url{carpediemimoveis.com.br}

\item \textbf{Cássio Carvalho Imóveis}
  \\Endereço: Av. Santa Isabel, 750
  \\Telefone: (19) 3288-0143
  \\E-mail: \email{cassio@cassioimoveis.com.br}
  \\Site: \url{cassioimoveis.com.br}

\item \textbf{Delphos Empreendimentos Imobiliários}
  \\Endereço: Av. Albino J. B. de Oliveira, 830
  \\Telefone: (19) 3289-5353

\item \textbf{Valter Imóveis}
  \\Endereço: Rua Maria Ferreira Antunes, 122
  \\Telefone: (19) 3289-6088
\end{itemize}

E pela quantidade de imobiliárias vistas, dá para ter uma ideia de como a
especulação imobiliária come solta em Barão. Boa sorte, bixete.
