% Este arquivo .tex será incluído no arquivo .tex principal. Não é preciso
% declarar nenhum cabeçalho

%
% Este arquivo foi remodelado para TENTAR deixar o mais organizado possível.
% Note que os estabelecimentos em Barão Geraldo são uma zona, então é bem
% difícil tornar isto bonito. Para isso, acho que vale a pena definir algumas
% instruções:
%
% * Manter poucas categorias para simplicidade, não colocar estabelecimentos
% como 'subsection'/'subsubsection' mas como itens de um 'itemize', com o nome
% em negrito,
% * Tentar explicar brevemente o que o local serve, usando uma linguagem
% universal, sem muitas gírias ou abreviações.
%

\section{O que comer dentro da Unicamp}
\subsection{Os restaurantes universitários da Unicamp}

Um dos momentos de glória do dia de uma futura engenheira, cientista ou
bacharel é o Bandejão. É a hora de intensas e indiscutíveis emoções. Caso sua
salada corra sobre a mesa, mantenha-se calma. Evite discussões, jamais tente
descobrir o sabor do suco pelo paladar (caju ou manga?). É mais cômodo ler no
cardápio do dia. Uma dica: para cortar o bife faça muita força e, quando
começar a amolecer, pare, você chegou na bandeja.

Falando sério agora: o RU (Restaurante Universitário), ou Bandejão, ou ainda
Bandeco, fica ao lado da Biblioteca Central, bem em frente ao PB (Prédio
Básico, ou Ciclo Básico II) e, a menos que você não queira economizar uma boa
grana com comida, vai ser o lugar onde você vai estar na maioria dos seus
horários de almoço. Com o tempo, você vai ver que o Bandeco (ou qualquer um dos
restaurantes universitários) é o ``coração da Unicamp''. É o local de você se
encontrar com amigas e amigos (combinando ou não antes), contar os micos nas
aulas, jogar conversa fora e falar mal da comida, que nem é tão ruim assim como
muitos dizem. Sem dúvida, é o melhor custo-benefício da Unicamp. Por R\$2,00,
você tem direito a arroz, feijão, pão, salada, proteína de soja, suco e café à
vontade. A carne e a sobremesa tem que dar uma choradinha para a tia ou o tio
para poder repetir, mas geralmente dá certo.

\begin{figure}[h!]
    \centering
    \includegraphics[width=.45\textwidth]{img/barao/bandeco.jpg}
\end{figure}

Existe também o RA (Restaurante Administrativo, não confundir com registro
acadêmico), também conhecido como Prateco, pelo fato de a comida ser servida
em pratos e não em bandejas, e o RS (Restaurante da Saturnino), o já não tão
novo restaurante universitário, que tem este nome por estar na rua Saturnino de
Brito.

O RA fica atrás da Faculdade de Engenharia Elétrica e de Computação (FEEC),
perto do prédio da Engenharia Básica. Em comparação com o Bandeco, o espaço
físico é bem menor, mas você mesmo se serve, apesar da carne ser servida pela
tia ou tio que trabalha lá, assim como no RS, que, por sua vez, fica perto do
IC-3 e, sem dúvida, é a melhor opção se você não se importar em fazer uma
caminhada, já que o ambiente é bem menos claustrofóbico que o RU e RA e é bem
longe.

Dependendo de onde você vai ter aula antes ou depois do almoço, é melhor
almoçar no RU, RS ou RA, mas, para isso, você precisa ter seu Cartão
Universitário (também chamado de RA) carregado, explicaremos como isso funciona
em breve.

\subsubsection{Horário de funcionamento dos restaurantes universitários}

Os restaurantes funcionam de segunda a sexta apenas, nos seguintes horários:

\begin{itemize}
\item RU, das 7h às 8h30 (café da manhã), das 10h30 às 14h (almoço) e das 17h30
 às 19h45 (jantar).
\item RA, das 11h30 às 14h (almoço) e das 17h30 às 19h (jantar).
\item RS, das 11h às 14h (almoço) e das 17h30 às 19h (jantar).
\end{itemize}

Em períodos especiais, como fim de ano, os restaurantes podem funcionar em
horários reduzidos ou não abrem, fique de olho nos e-mails e informes que você
recebe. Uma recomendação é curtir a página da Prefeitura Universitária da
Unicamp no Facebook, pois sempre postam algumas mudanças em serviços.
Geralmente o RU é o único que permanece aberto.

Para saber previamente o cardápio do Bandejão, acesse o site da Prefeitura do
Campus (\url{prefeitura.unicamp.br}), o GDE (\url{gde.ir}) ou o aplicativo de
Serviços da Unicamp, feito pela CCUEC, você poderá ver seu saldo no cartão, o
cardápio, suas notas após o fechamento do semestre etc.

\subsubsection{Como funciona o esquema de carregar o cartão?}

Simples. Você vai a uma das máquinas que existem na entrada dos restaurantes,
insere seu RA e coloca cédulas na máquina, ela irá recusá-las, como faria
qualquer outra máquina de refrigerante que você vê em filmes ou já utilizou
quando era menor, e você tenta novamente com a cédula virada de ponta cabeça
até funcionar. Alguns dizem até que é um processo aleatório. Infelizmente as
máquinas só aceitam notas e não existe troco.

Outra maneira de colocar créditos é fazer um depósito na conta do Bandeco no
Santander (Ag.: 207 / Conta: 43.010.009-2) ou no Banco do Brasil (Ag.: 4203-X /
Conta: 66.315-8) e, para carregar, dirigir-se ao Posto de Atendimento situado
no RU, das 9h às 14h e das 15h às 17h, apresentando o comprovante de depósito.
Não são aceitos comprovantes de pagamento de entrega de envelope ou via
internet.

\subsubsection{Cardápio vegetariano}

Um cardápio ovo-lacto-vegetariano está sendo oferecido desde o fim de 2013. Ele
está disponível apenas no RS, numa fila separada. Agora, é possível saborear
pratos tais como abobrinha agridoce, hambúrguer de soja, legumes com molho
branco, torta de aveia, entre outros, mas não se anime muito, há várias
críticas com relação a variedade.

No período de férias, normalmente o RS e RA estarão fechados e o cardápio
vegetariano passa a ser servido no RU. Não há fila separada, você precisa pedir
por ele em vez da carne.

\subsubsection{Café da manhã}

Em março de 2016, houve o início das operações do RU no período da manhã,
atendendo a uma demanda de estudantes. De acordo com o ex-reitor Tadeu Jorge,
a Unicamp investiu cerca de R\$ 70 mil para equipar os restaurantes para
oferecer o café da manhã. O período é das 7h às 8h30.

O preço é apenas R\$ 1,00. O cardápio consiste de café, leite, pão, manteiga,
cada dia há uma geleia e fruta diferente, então divirta-se com suas amigas e
amigos tentando advinhar qual será a geleia do dia.

% Fim da seção dos restaurantes universitários

\subsection{Outros lugares na Unicamp}

\subsubsection{Cantinas}

Há várias cantinas espalhadas pela Unicamp, pode ser bem cara às vezes, mas
sempre tem aquele combo razoável de algum salgado com um suco feito na hora ou
pão de queijo com café. Quase todas possuem salgados prontos, lanches naturais
e alguns doces. Algumas até fazem X-salada e pão na chapa.

\begin{itemize}
\item \textbf{Cantina da Física}
  \\Próximo ao CB, possui self-service.
\item \textbf{Cantina da Biologia}
  \\Próximo ao CB, possui self-service e marmita.
\item \textbf{Cantina da Química}
  \\Próximo ao CB, possui prato-feito, abre também aos sábados.
  \\O melhor é o pão de queijo em dobro às quartas e quintas!
\item \textbf{Banca de sucos do CB}
  \\Como no nome, fica próximo ao CB, possui diversas opções de sucos, vende
  frutas e salgados. Se quiser almoçar rápido, um salgado e uma vitamina é a
  nossa recomendação.
  \\Todo dia há um sabor de suco em oferta, ótima oportunidade para sair do
  tradicional suco de laranja.
\item \textbf{Gatti}
  \\Fica do lado do IC-2, na Cênicas/Dança.
  \\Se você é vegetariana, é uma boa dica.
\item \textbf{Cantina da Educação}
  \\Também ao lado do IC-2, tem self-service.
\item \textbf{Cantina da Economia}
  \\Ao outro lado do IC-2, tem bons lanches.
\item \textbf{Cantina da Mecânica}
  \\Fica próximo à FEEC, abre bem cedo, serve o bom pingado (café com leite)
  com pão na chapa matinal.
\item \textbf{Cantina da FEA/FEM}
  \\Próxima à FEEC, possui self-service no almoço e bons lanches.
  \\Pra quem não gosta de café, fica a dica da latinha de Red Bull a R\$ 5,99.
\end{itemize}

Sobre açaí: se você tiver de ir até as salas de aula da medicina, estiver
cansada e morrendo para tomar um açaí, o da cantina de lá é caro e ruim.
Prefira os da feirinha, falaremos mais dela abaixo.

\subsubsection{Feirinha}

Nas quartas e quintas há uma feira no centro da praça do CB, na qual há opções
bem variadas, desde pastéis a comida japonesa, embora geralmente mais caras que
as cantinas. Uma opção interessante é o Porqueta, que serve costelinhas de
porco assadas e sanduiches muito bem feitos, é um pouco caro. Outra é batata suíça que, por R\$12, mata sua fome até 18h30.

\section{O que comer fora da Unicamp}

Nos finais de semana, os restaurantes universitários não abrem, assim como a
maioria das cantinas (que, se abrem, o fazem apenas no sábado). Ou seja, você
terá que se virar fora da Unicamp, seja em restaurantes, seja aprendendo a
cozinhar. Aqui um resumo de lugares que se destacam:

Na Av. 1 e proximidades, há o Jhonny Grill, o Bardana, a Padaria Alemã e vários
restaurantes próximos à padaria.

Na Av. 2 e proximidades, há o Aulus (mais caro no final de semana, sobretudo no
domingo, porém costuma ter camarão à milanesa para fazer valer a pena; a
marmita não possui tantas opções de carne e acompanhamentos, no entanto é
grande e custa R\$13,75, bem mais em conta que o self-service) e, mais para
cima na avenida, há o Yaki-Ten (comida chinesa por quilo, japonesa por pessoa).
% "Logo mais abaixo há o Ilha do Barão" não achei nenhuma informação sobre esse
% lugar.. ele existe?

No centro de Barão, não faltam opções. No sentido da entrada de Barão pela
Estrada da Rhodia/Av. Albino A. B. de Oliveira, tem o Estância Grill, o Barão
da Picanha, o Gordão Lanches, o Solar dos Pampas, o Estância d'Oliveira, o Vila
Santo Antônio, o Ki-Pizza, o restaurante Baroneza, o Salsinha e Cebolinha, o
Alabama, o Pão de Açúcar, o McDonald's, o Burger King e o Subway no Tilli
Center. Em frente ao Pague Menos, tem o Lótus: vegetariano (não vegano), barato
e gostoso.

Ainda no centro, na Av. Santa Isabel e adjacências, tem o Cronópio, o
Frangonete (próximo ao Santander), o HotDog Central, as pizzarias Sapore Pizza
e Pizza Fiori. Próximo à Moradia, tem a Tonha (Centro do Acarajé), o Kalunga
Lanches e o famoso dogão da moradia.

Próximo à entrada da Cidade Universitária II, à Estrada da Rhodia, há a
Panetteria Di Capri, a Pizzaria Gregória, o TBONE (possui marmitex), o Greg
Burgers (hambúrguer e milkshake são excelentes), o Tabuá dos Mares e o
Morena-flor, também há uns restaurantes mais caros, como a Romana (parecido com
a Di Capri, mas muito mais cara).\\

\begin{figure}[h!]
    \centering
    \includegraphics[width=.45\textwidth]{img/barao/pizza.jpg}
\end{figure}

% Pulamos uma linha do "resumo" de estabelecimentos
Em Barão Geraldo, muitos estabelecimentos fazem muita coisa, o que torna bem
trabalhoso uma formatação razoável do material. Fizemos o seguinte: pegamos as
categorias que consideramos ``mais procuradas'' e colocamos os estabelecimentos
que abrangem mais categorias primeiro, como a Padaria Alemã que faz muito mais
que uma padaria.

Dividimos os estabelecimentos entre as categorias:

% 'noitemsep' deixa os itens compactos, sem tanto espaço entre um e outro
\begin{itemize}[noitemsep]
% duas colunas de itens
\begin{multicols}{2}
\item padaria,
\item pizzaria,
\item salgados,
\item lanchonete,
\item bar,
\item restaurante,
\item churrascaria,
\item marmita,
\item sorveteria.
\end{multicols}
\end{itemize}

Aqui estão algumas informações mais detalhadas, contatos de cada
estabelecimento:

\subsection{Padaria, pizzaria, salgados, restaurante}

\begin{itemize}
% Dividi em 'subitems' espaçados para facilitar a leitura. Podemos tentar usar
% o package 'easylist', mas achei desnecessário trocar por enquanto.
% Espaçamento resolve o problema da leitura.
\item \textbf{Panetteria Di Capri}
  \\Endereço: R. Maria Tereza Dias da Silva, 530
  \\E-mail: \email{panetteriadcapri@uol.com.br}
  \\Telefone: (19) 3289-4446, (19) 3289-3338
  \\
  % Apresentação
  \\Na Estrada da Rhodia, próximo à entrada da Cidade Universitária II, servem
  café da manhã, no almoço possui alguns pratos, já no jantar servem pizzas e
  grelhados que são substituídos por um buffet de sopas durante o inverno.
  \\
  % Breve descrição das categorias servidas
  \\O pão francês é muito bom por um preço legal, possui grande variedade de
  salgados, incluindo até tortas e lanches. Tem um bom cardápio para o café da
  manhã sendo que, de sexta a domingo, servem um buffet com várias opções a um
  preço fixo (aproximadamente R\$12).
  \\
  \\No almoço há alguns pratos para comer no local ou levar. Se pedir um
  grelhado, tem acesso livre a um balcão com saladas e outras coisas como
  petiscos. A noite, há pizzas e o mesmo esquema do grelhado, que é substituído
  por um buffet de sopas no inverno.
\end{itemize}

\subsection{Padaria, pizzaria, salgados}

\begin{itemize}
\item \textbf{Padaria Alemã}
  \\Endereço: Av. Dr. Romeu Tórtima, 285
  \\E-mail para encomendas:\\\email{encomendas@padariaalema.com.br}
  \\E-mail para sugestões, reclamações:\\\email{gerencia@padariaalema.com.br}
  \\Site: \url{www.padariaalema.com.br}
  \\Telefone: (19) 3289-2581
  \\
  \\Na Av. 1, próxima ao supermercado Dalben, servem café da manhã, lanches e
  pizzas.
  \\
  \\Os lanches são gigantescos e possuem várias opções de recheio por um preço
  relativamente barato, já que um pode substituir um almoço.
  \\
  \\O café da manhã é servido das 7 às 13h, embora a padaria feche apenas às
  22h. Serve um grande combo de café da manhã: suco, café com leite, chocolate
  quente, croissant, mamão, bolo, pão francês, torradas, manteiga e geleia.
  Você ainda pode fazer trocas, como suco por chocolate, croissant por 2 pães
  na chapa, mamão por banana, pergunte o que oferecem!
  \\
  \\Também servem pizza que, dependendo do recheio, é muito barata.
  Infelizmente não fazem entrega.
  \\
  \\É um bom lugar para tomar café num sábado ou domingo valendo pelo almoço ou
  para levar seus pais se eles puderem te visitar.
\end{itemize}

\begin{figure}[h!]
    \centering
    \includegraphics[width=.45\textwidth]{img/barao/padaria.jpg}
\end{figure}
