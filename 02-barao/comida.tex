% Este arquivo .tex será incluído no arquivo .tex principal. Não é preciso
% declarar nenhum cabeçalho

%
% Este arquivo foi remodelado para TENTAR deixar o mais organizado possível.
% Note que os estabelecimentos em Barão Geraldo são uma zona, então é bem
% difícil tornar isto bonito. Para isso, acho que vale a pena definir algumas
% instruções:
%
% * Manter poucas categorias para simplicidade, não colocar estabelecimentos
% como 'subsection'/'subsubsection' mas como itens de um 'itemize', com o nome
% em negrito,
% * Tentar explicar brevemente o que o local serve, usando uma linguagem
% universal, sem muitas gírias ou abreviações.
%

\section{O que comer dentro da Unicamp}
\subsection{Os restaurantes universitários da Unicamp}

Um dos momentos de glória do dia de uma futura engenheira, cientista ou
bacharel é o Bandejão. É a hora de intensas e indiscutíveis emoções. Caso sua
salada corra sobre a mesa, mantenha-se calma. Evite discussões, jamais tente
descobrir o sabor do suco pelo paladar (caju ou manga?). É mais cômodo ler no
cardápio do dia. Uma dica: para cortar o bife faça muita força e, quando
começar a amolecer, pare, você chegou na bandeja.

Falando sério agora: o RU (Restaurante Universitário), ou Bandejão, ou ainda
Bandeco, fica ao lado da Biblioteca Central, bem em frente ao PB (Prédio
Básico, ou Ciclo Básico II) e, a menos que você não queira economizar uma boa
grana com comida, vai ser o lugar onde você vai estar na maioria dos seus
horários de almoço. Com o tempo, você vai ver que o Bandeco (ou qualquer um dos
restaurantes universitários) é o ``coração da Unicamp''. É o local de você se
encontrar com amigas e amigos (combinando ou não antes), contar os micos nas
aulas, jogar conversa fora e falar mal da comida, que nem é tão ruim assim como
muitos dizem.

% Ao atualizar o preço do almoço e jantar do restaurante universitário,
% corrigir referência em transporte.tex que o compara o custo da passagem de
% ônibus municipal.

Sem dúvida, os restaurantes universitários oferecem o melhor custo-benefício da
Unicamp para almoço e jantar. Por R\$ 3,00, você tem direito a arroz, feijão,
pão, salada, proteína de soja, suco e café à vontade. A carne e a sobremesa tem
que dar uma choradinha para a tia ou o tio para poder repetir, mas geralmente
dá certo.

\begin{figure}[h!]
  \centering
  \includegraphics[width=.45\textwidth]{img/barao/bandeco.jpg}
\end{figure}

Existe também o RA (Restaurante Administrativo, não confundir com registro
acadêmico), também conhecido como Prateco, pelo fato de a comida ser servida
em pratos e não em bandejas, e o RS (Restaurante da Saturnino), o já não tão
novo restaurante universitário, que tem este nome por estar na rua Saturnino de
Brito.

O RA fica atrás da Faculdade de Engenharia Elétrica e de Computação (FEEC),
perto do prédio da Engenharia Básica. Em comparação com o Bandeco, o espaço
físico é bem menor, mas você mesmo se serve, apesar da carne ser servida pela
tia ou tio que trabalha lá, assim como no RS, que, por sua vez, fica perto do
IC-3 e, sem dúvida, é a melhor opção se você não se importar em fazer uma
caminhada, já que o ambiente é bem menos claustrofóbico que o RU e RA e é bem
longe.

Dependendo de onde você vai ter aula antes ou depois do almoço, é melhor
almoçar no RU, RS ou RA, mas, para isso, você precisa ter seu Cartão
Universitário (também chamado de RA) carregado, explicaremos como isso funciona
em breve.

De acordo com a pesquisa de custo de vida em Barão Geraldo realizada em 2017,
no almoço, apenas 17\% das pessoas participantes não usam os restaurantes
universitários, um pouco menos da metade os utilizam todos os dias. Já no
jantar, 28\% nunca usam os restaurantes e menos de um terço o fazem todos os
dias.

Nos almoços fora da Unicamp, 10\% disseram gastar menos que 10 reais, 32\%
gastam entre 10 e 17 reais, 37\%, entre 17 e 25 reais. Nos jantares, 17\%
disseram gastar menos de 10 reais, 26\% gastam entre 10 e 17 reais, 35\%,
entre 17 e 25 reais e uma parcela considerável de 14\% gastam entre 25 e 32
reais.

Então a maioria das pessoas não utilizam muito os serviços do RU, RS ou RA para
jantar e tendem a gastar mais, apesar do custo ser melhor distribuído entre os
intervalos de valores. No almoço, poucas não usam os restaurantes
universitários e gastam entre 10 e 25 reais quando não o fazem.

\subsubsection{Horário de funcionamento dos restaurantes universitários}

Os restaurantes funcionam de segunda a sexta apenas, nos seguintes horários:

\begin{itemize}
\item RU, das 7h às 8h30 (café da manhã), das 10h30 às 14h (almoço) e das 17h30
  às 19h45 (jantar).
\item RA, das 11h30 às 14h (almoço) e das 17h30 às 19h (jantar).
\item RS, das 11h às 14h (almoço) e das 17h30 às 19h (jantar).
\end{itemize}

Em períodos especiais, como fim de ano, os restaurantes podem funcionar em
horários reduzidos ou não abrem, fique de olho nos e-mails e informes que você
recebe. Uma recomendação é curtir a página da Prefeitura Universitária da
Unicamp no Facebook, pois sempre postam algumas mudanças em serviços.
Geralmente o RU é o único que permanece aberto.

Para saber previamente o cardápio do Bandejão, acesse o site da Prefeitura do
Campus (\url{prefeitura.unicamp.br}), o GDE (\url{gde.ir}) ou o aplicativo de
Serviços da Unicamp, feito pela CCUEC, você poderá ver seu saldo no cartão, o
cardápio, suas notas após o fechamento do semestre etc.

\subsubsection{Como funciona o esquema de carregar o cartão?}

Simples. Você vai a uma das máquinas que existem na entrada dos restaurantes,
insere seu RA e coloca cédulas na máquina, ela irá recusá-las, como faria
qualquer outra máquina de refrigerante que você vê em filmes ou já utilizou
quando era menor, e você tenta novamente com a cédula virada de ponta cabeça
até funcionar. Alguns dizem até que é um processo aleatório. Infelizmente as
máquinas só aceitam notas e não existe troco.

Outra maneira de colocar créditos é fazer um depósito na conta do Bandeco no
Santander (Ag.: 207 / Conta: 43.010.009-2) ou no Banco do Brasil (Ag.: 4203-X /
Conta: 66.315-8) e, para carregar, dirigir-se ao Posto de Atendimento situado
no RU, das 9h às 14h e das 15h às 17h, apresentando o comprovante de depósito.
Não são aceitos comprovantes de pagamento de entrega de envelope ou via
internet.

\subsubsection{Cardápio vegetariano}

Um cardápio ovo-lacto-vegetariano está sendo oferecido desde o fim de 2013. Ele
está disponível apenas no RS, numa fila separada. Agora, é possível saborear
pratos tais como abobrinha agridoce, hambúrguer de soja, legumes com molho
branco, torta de aveia, entre outros, mas não se anime muito, há várias
críticas com relação a variedade.

No período de férias, normalmente o RS e RA estarão fechados e o cardápio
vegetariano passa a ser servido no RU. Não há fila separada, você precisa pedir
por ele em vez da carne.

\subsubsection{Café da manhã}

Em março de 2016, houve o início das operações do RU no período da manhã,
atendendo a uma demanda de estudantes. De acordo com o ex-reitor Tadeu Jorge,
a Unicamp investiu cerca de R\$ 70 mil para equipar os restaurantes para
oferecer o café da manhã. O período é das 7h às 8h30.

O preço é R\$ 2,00 depois do aumento de 2017 (era apenas 1 real), ou seja,
dependendo do que você come, não vale mais a pena tomar café da manhã no RU. O
cardápio consiste de café, leite, pão, manteiga, cada dia há uma geleia e fruta
diferente, então divirta-se com suas amigas e amigos tentando advinhar qual
será a geleia do dia.

% Fim da seção dos restaurantes universitários

\subsection{Outros lugares na Unicamp}

\subsubsection{Cantinas}

Há várias cantinas espalhadas pela Unicamp, pode ser bem cara às vezes, mas
sempre tem aquele combo razoável de algum salgado com um suco feito na hora ou
pão de queijo com café. Quase todas possuem salgados prontos, lanches naturais
e alguns doces. Algumas até fazem X-salada e pão na chapa.

\begin{itemize}
\item \textbf{Cantina da Física}
  \\Próximo ao CB, possui self-service.

\item \textbf{Cantina da Biologia}
  \\Próximo ao CB, possui self-service e marmita.

\item \textbf{Cantina da Química}
  \\Próximo ao CB, possui prato-feito, abre também aos sábados.
  \\O melhor é o pão de queijo em dobro às quartas e quintas!

\item \textbf{Banca de sucos do CB}
  \\Como no nome, fica próximo ao CB, possui diversas opções de sucos, vende
  frutas e salgados. Se quiser almoçar rápido, um salgado e uma vitamina é a
  nossa recomendação.
  \\Todo dia há um sabor de suco em oferta, ótima oportunidade para sair do
  tradicional suco de laranja.

\item \textbf{Gatti}
  \\Fica do lado do IC-2, na Cênicas/Dança.
  \\Se você é vegetariana, é uma boa dica.

\item \textbf{Cantina da Educação}
  \\Também ao lado do IC-2, tem self-service.

\item \textbf{Cantina da Economia}
  \\Ao outro lado do IC-2, tem bons lanches.

\item \textbf{Cantina da Mecânica}
  \\Fica próximo à FEEC, abre bem cedo, serve o bom pingado (café com leite)
  com pão na chapa matinal.

\item \textbf{Cantina da FEA/FEM}
  \\Próxima à FEEC, possui self-service no almoço e bons lanches.
  \\Pra quem não gosta de café, fica a dica da latinha de Red Bull a R\$ 5,99.
\end{itemize}

Sobre açaí: se você tiver de ir até as salas de aula da medicina, estiver
cansada e morrendo para tomar um açaí, o da cantina de lá é caro e ruim.
Prefira os da feirinha, falaremos mais dela abaixo.

\subsubsection{Feirinha}

Nas quartas e quintas há uma feira no centro da praça do CB, na qual há opções
bem variadas, desde pastéis a comida japonesa, embora geralmente mais caras que
as cantinas. Em semanas com feriados em algum desses dias, a feirinha irá abrir
ou na terça-feira ou na sexta-feira, é quase garantido que ela estará aqui 2
vezes por semana.

Uma opção interessante é o Porqueta, que serve costelinhas de porco assadas e
sanduiches muito bem feitos, é um pouco caro. Outra é batata suíça que, por
R\$ 12, mata sua fome até 18h30.

\section{O que comer fora da Unicamp}

Nos finais de semana, os restaurantes universitários não abrem, assim como a
maioria das cantinas (que, se abrem, o fazem apenas no sábado). Ou seja, você
terá que se virar fora da Unicamp, seja em restaurantes, seja aprendendo a
cozinhar. Aqui um resumo de lugares que se destacam:

Na Av. 1 e proximidades, há o Jhonny Grill, o Bardana, a Padaria Alemã e vários
restaurantes próximos à padaria.

Na Av. 2 e proximidades, há o Aulus (mais caro no final de semana, sobretudo no
domingo, porém costuma ter camarão à milanesa para fazer valer a pena; a
marmita não possui tantas opções de carne e acompanhamentos, no entanto é
grande e custa R\$ 15,75, bem mais em conta que o self-service) e, mais para
cima na avenida, há o Yaki-Ten (comida chinesa por quilo, japonesa por pessoa).
% "Logo mais abaixo há o Ilha do Barão" não achei nenhuma informação sobre esse
% lugar.. ele existe?

No centro de Barão, não faltam opções. No sentido da entrada de Barão pela
Estrada da Rhodia/Av. Albino A. B. de Oliveira, tem o Estância Grill, o Barão
da Picanha, o Gordão Lanches, o Solar dos Pampas, o Estância d'Oliveira, o Vila
Santo Antônio, o Ki-Pizza, o restaurante Baroneza, o Salsinha e Cebolinha, o
Alabama, o Pão de Açúcar, o McDonald's, o Burger King e o Subway no Tilli
Center. Em frente ao Pague Menos, tem o Lótus: vegetariano (não vegano), barato
e gostoso.

Ainda no centro, na Av. Santa Isabel e adjacências, tem o Cronópio, o
Frangonete (próximo ao Santander), o HotDog Central, as pizzarias Sapore Pizza
e Pizza Fiori. Próximo à Moradia, tem a Tonha (Centro do Acarajé), o Kalunga
Lanches e o famoso dogão da moradia.

Próximo à entrada da Cidade Universitária II, à Estrada da Rhodia, há a
Panetteria Di Capri, a Pizzaria Gregória, o Greg Burgers (hambúrguer e
milkshake são excelentes), o Tabuá dos Mares e o Morena-flor, também há uns
restaurantes mais caros, como a Romana (parecido com a Di Capri, mas muito mais
cara).\\

\begin{figure}[h!]
  \centering
  \includegraphics[width=.45\textwidth]{img/barao/pizza.jpg}
\end{figure}

% Pulamos uma linha do "resumo" de estabelecimentos
Em Barão Geraldo, muitos estabelecimentos fazem muita coisa, o que torna bem
trabalhoso uma formatação razoável do material. Fizemos o seguinte: pegamos as
categorias que consideramos ``mais procuradas'' e colocamos os estabelecimentos
que abrangem mais categorias primeiro, como a Padaria Alemã que faz muito mais
que uma padaria.

Dividimos os estabelecimentos entre as categorias:

\begin{compactitemize}
% duas colunas de itens
\begin{multicols}{2}
\item padaria,
\item pizzaria,
\item salgados,
\item lanchonete,
\item bar,
\item restaurante,
\item churrascaria,
\item marmita,
\item sorveteria.
\end{multicols}
\end{compactitemize}

Aqui estão algumas informações mais detalhadas, contatos de cada
estabelecimento:

\subsection{Bar, restaurante, pizzaria, churrascaria, marmita}

\begin{itemize}
\item \textbf{Aulus}
  \\Endereço: Av. Professor Atílio Martini, 939
  \\Telefone: (19) 3289-4453
  \\E-mail: \email{aulus01@terra.com}
  \\Site: \url{www.aulus.com.br}
  \\
  \\Localizado na Av. 2, próximo ao balão, é o mais caro dos citados aqui, mas
  é muito bom (e bonito). O cardápio geralmente inclui peixes e frutos do mar e
  churrasco todo dia, além de feijoada toda quarta-feira e pizza à noite.
\end{itemize}

\subsection{Padaria, pizzaria, salgados, restaurante}

\begin{itemize}
% Dividi em 'subitems' espaçados para facilitar a leitura. Podemos tentar usar
% o package 'easylist', mas achei desnecessário trocar por enquanto.
% Espaçamento resolve o problema da leitura.
\item \textbf{Panetteria Di Capri}
  \\Endereço: R. Maria Tereza Dias da Silva, 530
  \\Telefone: (19) 3289-4446, (19) 3289-3338
  \\E-mail: \email{panetteriadcapri@uol.com.br}
  \\
  % Apresentação
  \\Na Estrada da Rhodia, próximo à entrada da Cidade Universitária II, servem
  café da manhã, no almoço possui alguns pratos, já no jantar servem pizzas e
  grelhados que são substituídos por um buffet de sopas durante o inverno.
  \\
  % Breve descrição das categorias servidas
  \\O pão francês é muito bom por um preço legal, possui grande variedade de
  salgados, incluindo até tortas e lanches. Tem um bom cardápio para o café da
  manhã sendo que, de sexta a domingo, servem um buffet com várias opções a um
  preço fixo (aproximadamente R\$ 12).
  \\
  \\No almoço há alguns pratos para comer no local ou levar. Se pedir um
  grelhado, tem acesso livre a um balcão com saladas e outras coisas como
  petiscos. A noite, há pizzas e o mesmo esquema do grelhado, que é substituído
  por um buffet de sopas no inverno.
\end{itemize}

\subsection{Restaurante, lanchonete, sorveteria}

\begin{itemize}
\item \textbf{Pepe Loco}
  \\Endereço: Av. Romeo Tórtima, 1550
  \\Telefone: (19) 3305-4146
  \\E-mail: \email{contatopepeloco@gmail.com}
  \\Site: \url{www.pepeloco.com.br}
  \\
  \\Serve comida mexicana no estilo fast-food, porém costuma ser bem caro pela
  qualidade e quantidade que oferece. Tem um bom milkshake.
\end{itemize}

\subsection{Restaurante, pizzaria, marmita}

\begin{itemize}
\item \textbf{Alabama Restaurante e Pizzaria}
  \\Endereço: Av. Albino J. B. de Oliveira, 1170
  \\Telefone: (19) 3289-0692

\item \textbf{Raízes Zen}
  \\Endereço: R. Antonio Pierozzi, 94
  \\Telefone: (19) 3305-2667, (19) 3288-0531
  \\E-mail: \email{barao@raizeszen.com.br}
  \\Site: \url{raizeszen.com.br}
  \\
  \\Vegetariano com bom preço.

\item \textbf{Sapore Pizza}
  \\Endereço: Av. Santa Isabel, 326
  \\Telefone: (19) 3289-0228
  \\
  \\Para quando você estiver com pelo menos mais um amigo para rachar a pizza,
  acaba sendo uma boa pedida. Geralmente as pizzas de mussarela e de calabresa
  estão com preços bem acessíveis. Além de pizzas, fazem esfihas e batata
  rechada. Entregam até 23h.
  \\
  \\A Sapore também tem self-service no almoço, R\$ 15,50 por pessoa durante a
  semana e um pouco mais aos fins de semana, e marmita pra retirar no local,
  R\$ 36 o quilo pra você montar sua marmita com as coisas do self-service, ou
  R\$ 11 a marmita pronta, há várias opções de carnes.
\end{itemize}

\subsection{Padaria, pizzaria, salgados}

\begin{itemize}
\item \textbf{Padaria Alemã}
  \\Endereço: Av. Dr. Romeu Tórtima, 285
  \\Telefone: (19) 3289-2581
  \\E-mail para encomendas:\\\email{encomendas@padariaalema.com.br}
  \\E-mail para sugestões, reclamações:\\\email{gerencia@padariaalema.com.br}
  \\Site: \url{www.padariaalema.com.br}
  \\
  \\Na Av. 1, próxima ao supermercado Dalben, servem café da manhã, lanches e
  pizzas.
  \\
  \\Os lanches são gigantescos e possuem várias opções de recheio por um preço
  relativamente barato, já que um pode substituir um almoço.
  \\
  \\O café da manhã é servido das 7 às 13h, embora a padaria feche apenas às
  22h. Serve um grande combo de café da manhã: suco, café com leite, chocolate
  quente, croissant, mamão, bolo, pão francês, torradas, manteiga e geleia.
  Você ainda pode fazer trocas, como suco por chocolate, croissant por 2 pães
  na chapa, mamão por banana, pergunte o que oferecem!
  \\
  \\Também servem pizza que, dependendo do recheio, é muito barata.
  Infelizmente não fazem entrega.
  \\
  \\É um bom lugar para tomar café num sábado ou domingo valendo pelo almoço ou
  para levar seus pais se eles puderem te visitar.
\end{itemize}

\begin{figure}[h!]
  \centering
  \includegraphics[width=.45\textwidth]{img/barao/padaria.jpg}
\end{figure}

\subsection{Restaurante, marmita, salgados}

\begin{itemize}
\item \textbf{Pastelaria Oba Oba}
  \\Endereço: R. Benedito Alves Aranha, 115
  \\Telefone: (19) 3249-1908
\end{itemize}

\subsection{Restaurante, salgados, bar}

\begin{itemize}
\item \textbf{Empório do Nono}
  \\Endereço: Av. Albino J. B. Oliveira, 1128
  \\Telefone: (19) 3289-0041
  \\Site: \url{emporiodonono.com.br}
  \\
  \\Caro, tem um chopp muito bem tirado e petiscos maravilhosos. Próximo ao
  terminal de Barão Geraldo.
\end{itemize}

\subsection{Restaurante, churrascaria, marmita}

\begin{itemize}
\item \textbf{Jhonny Grill}
  \\Endereço: R. Roxo Moreira, 1344
  \\Telefone: (19) 3289-7920
  \\
  \\Muito próximo da Unicamp, em frente à entrada II (Av. 1), possui
  self-service por quilo ou a vontade a um bom preço, churrasco às terças,
  quintas e sábados, pratos feitos e entregam marmita.

\item \textbf{Moriá}
  \\Endereço: R. Roxo Moreira, 1728
  \\Telefone: (19) 3289-2343
  \\
  \\Próximo à reitoria, serve self-service por quilo com preços bons.
\end{itemize}

\subsection{Bar, lanchonete, sorveteria}

\begin{itemize}
\item \textbf{Greg Burguers}
  \\Endereço: R. Maria Tereza Dias da Silva, 664
  \\Telefone: (19) 3289-6400
  \\Site: \url{gregburgers.com.br}
  \\
  \\Uma lanchonete muito boa, mas também muito cara. Uma das especialidades lá
  é o milkshake (realmente muito bom). Fica na esquina da Panetteria Di Capri.
\end{itemize}

\subsection{Restaurante, churrascaria}

\begin{itemize}
\item \textbf{Campus Grill}
  \\Endereço: R. Roxo Moreira, 1830
  \\Telefone: (19) 3289-0084
  \\
  \\Em frente à guarita do HC, serve comida boa a um preço um tanto alto.

\item \textbf{Estância Grill}
  \\Endereço: Av. Albino J. B. de Oliveira, 271
  \\Telefone: (19) 3289-8697, (19) 3289-6055, (19) 3289-1511
  \\E-mail: \email{estanciareservas@gmail.com}
  \\Site: \url{www.estanciacampinas.com.br}
  \\
  \\Logo na entrada de Barão Geraldo, possui rodízios de carne e pizza à noite.

\item \textbf{Solar dos Pampas}
  \\Endereço: Av. Dr. Romeu Tortima, 165
  \\Telefone: (19) 3289-1484, (19) 3289-7869
  \\
  \\Buffet excelente. Custa R\$ 24,00 apenas a comida e R\$ 34,00 com
  refrigerante e suco incluídos (o tanto que você conseguir beber). Fazem um
  esquema no aniversário das pessoas que sai por R\$ 18,00 com rodízio,
  cerveja, refrigerante, buffet, sorvete e pinga à vontade. Ao lado do Estância
  d'Oliveira.
\end{itemize}

\subsection{Restaurante, marmita}

\begin{itemize}
\item \textbf{Restaurante Baronesa}
  \\Endereço: R. Benedito Alves Aranha, 44
  \\Telefone: (19) 3289-9087
  \\
  \\No centro de Barão Geraldo, é self-service e entrega marmita.
\end{itemize}

\subsection{Bar, lanchonete}

\begin{itemize}
\item \textbf{Bar do Coxinha}
  \\Endereço: R. Antônio Pierozzi, 230
  \\Telefone: (19) 3289-2117
  \\
  \\Famoso pela coxinha (realmente boa), vale a pena ir, porém é caro. É
  próximo da Sapore Pizza.

\item \textbf{Buteco do Jair}
  \\Endereço: R. Eduardo Modesto, 212
  \\Telefone: (19) 3326-2903, (19) 3308-4825
  \\E-mail: \email{contato@butecodojair.com.br}
  \\Site: \url{www.butecodojair.com.br/}
  \\
  \\Outro lugar famoso pela coxinha, só que é de carne seca. Relativamente
  perto da Moradia da Unicamp. Fazem entrega.

\item \textbf{Marambar}
  \\Endereço: R. Euríco Vanderlei Morais Carva, 8
  \\Telefone: (19) 3289-2007
  \\
  \\Possui bebidas, lanches, sucos e porções a preços razoáveis. Ambiente
  agradável, ao ar livre, muito próximo da Unicamp, fica na entrada II, da Av.
  1. Bastante frequentado por alunas e alunos. Funciona de segunda a sábado, a
  partir das 14h.

\item \textbf{Ponto 1 Bar}
  \\Endereço: R. Eduardo Modesto, 54
  \\Telefone: (19) 3289-2378
  \\Site: \url{ponto1bar.com}
\end{itemize}

\subsection{Bar, pizzaria}

\begin{itemize}
\item \textbf{Star Clean}
  \\Endereço: Av. Atílio Martini, 940
  \\Telefone: (19) 3289-5454
  \\
  \\É o bar mais próximo à Unicamp e, por isso, está sempre cheio. Principal
  ponto de encontro depois da aula e tem um bom preço.
\end{itemize}

\subsection{Pizzaria, lanchonete}

\begin{itemize}
\item \textbf{Pizza Fiori}
  \\Endereço: Av. Santa Isabel, 405
  \\Telefone: (19) 3289-3514
  \\Site: \url{http://www.pizzafiori.com.br}
\end{itemize}

\subsection{Restaurante}

\begin{itemize}
\item \textbf{Bardana}
  \\Endereço: Av. Dr. Romeu Tórtima, 1500
  \\Telefone: (19) 3289-9073
  \\Facebook: \url{fb.com/restaurantebardanabarao}
  \\
  \\Um pouco acima da entrada II (Av. 1), mesma faixa de preço do Jhonny Grill,
  muitos falam que é melhor. No jantar, possui pratos à la carte, veja o
  cardápio do jantar no Facebook, onde divulgam todos os dias durante a tarde.
  Bom lugar para passar a noite com amigas e amigos.

\item \textbf{Del Sol}
  \\Endereço: R. Roxo Moreira, 1648
  \\Telefone: (19) 3289-1446
  \\
  \\Localizado perto da reitoria, serve comida por quilo, sendo parecido (em
  preço e pratos) com o Bardana. O suco é grátis para pratos acima de R\$ 10.

\item \textbf{Ginza}
  \\Endereço: R. Roxo Moreira, 1768
  \\Telefone: (19) 3289-9281
  \\
  \\Serve à la carte com preços bons.

\item \textbf{China in Box}
  \\Endereço: R. Professor Gustavo Enge, 39
  \\Telefone: (19) 3254-5601
  \\
  \\Apesar de estar localizado no Cambuí, faz entregas em Barão Geraldo.

\item \textbf{Casa da Moqueca}
  \\Endereço: Maria Ferreira Antunes, 123
  \\Telefone: (19) 3289-3131
  \\
  \\Prato mais caro, mas serve duas pessoas.

\item \textbf{Gua.co}
  \\Endereço: Av. Albino J. B. de Oliveira, 1615
  \\Telefone: (19) 3365-2588
  \\Site: \url{guacamolecompany.com.br}
  \\
  \\Restaurante mexicano.

\item \textbf{Makis Place}
  \\Endereço: Av. Albino J. B. de Oliveira, 976
  \\Telefone: (19) 3367-3077
  \\Site: \url{makis.com.br}
  \\
  \\Temakeria próxima ao terminal Barão Geraldo.

\item \textbf{Kanu Sushi Beer}
  \\Endereço: Av. Dr. Romeu Tortima, 1259
  \\Telefone: (19) 3289-0073
  \\Horário de funcionamento: de segunda à sexta das 11h30 às 14h30min e das 17h30 às 22h45min. De sábado das 12h00 às 15h30min e das 18h00 até às 17h do domingo.
  De domingo das 17h30 até às 22h45min.
  \\Site: \url{https://www.facebook.com/kanusushibeer/}
  \\
  \\Restaurante de sushi aberto no lugar da antiga Temakeria Barão Geraldo.
  O preço é um pouco alto mas compatível com o preço de comida japonesa em
  Barão Geraldo. Possui rodízios por preço bom 

\item \textbf{Estância d'Oliveira}
  \\Endereço: Av. Albino J. B. de Oliveira, 576
  \\Telefone: (19) 3289-5369, 3249-1510
  \\Site: \url{estanciadoliveirabarao.com.br}
  \\
  \\Antigo Universo das Massas. Rodízio de massas perto do Terminal. Bom e não
  é caro. De domingo à noite é o horário mais barato e dá pra encher bem o
  bucho de massa. Depois de ir até lá, você não vai querer saber de comer
  massas por um bom tempo.
\end{itemize}

\subsection{Pizzaria}

Todas as pizzarias dessa seção fazem entregas.

\begin{itemize}
\item \textbf{Pizza Mais}
  \\Endereço: R. Ângela Signori Grigol, 330
  \\Telefone: (19) 3289-0320
  \\
  \\Faz entregas e vende pizza gigante por um preço amigável, bom para dividir
  com amigos.

\item \textbf{Barão das Pizzas}
  \\Endereço: R. Jerônimo Pattaro, 351
  \\Telefone: (19) 3249-1630
  \\Site: \url{baraodaspizzas.com.br}

\item \textbf{Ki-pizza}
  \\Endereço: R. Horácio Leonardi, 76
  \\Telefone: (19) 3289-0863

\item \textbf{Domino's Pizzaria}
  \\Endereço: Av. Albino J. B. de Oliveira, 1453
  \\Telefone: (19) 3368-7557
  \\Site: \url{www.dominos.com.br}
  \\
  \\Famosa rede de pizzarias, tem pizza em dobro às terças-feiras.

\item \textbf{Vila Ré - Pizza}
  \\Endereço: Av. Albino J. B. de Oliveira, 658
  \\Telefone: (19) 3289-0321, (19) 3289-0319
  \\
  \\Pizzaria próxima do terminal e do supermercado Dalben. Tem alguns sabores
  diferentes, as pizzas são boas e o preço não é alto.
\end{itemize}

\subsection{Lanchonete}

\begin{itemize}
\item \textbf{Nadog's Hot Dog}
  \\Endereço: Av. Santa Izabel, 359
  \\Telefone: (19) 3029-2270
  \\
  \\Faz entregas de cachorro-quente.

\item \textbf{Azedinho Doce}
  \\Endereço: Av. Santa Isabel, 518
  \\Telefone: (19) 3365-6555
  \\
  \\Serve um açaí muito bom e vários tipos de comidas mais leves, como lanches
  naturais, crepes e saladas, além de vários sucos. O preço não é caro e a
  comida é boa.

\item \textbf{Bagdá -- Cozinha Árabe}
  \\Endereço: R. Maria Ferreira Antunes, 116
  \\Telefone: (19) 3289-0311
  \\Site: \url{www.bagdacozinhaarabe.com.br}
  \\
  \\Esfihas boas, mas um pouco caras. Entregam em Barão (cardápio no site), mas
  em horários de pico costumam demorar um pouco. A música ambiente inclui
  música ao vivo e ritmos variados, desde a MPB ao Blues.
\end{itemize}

\begin{figure}[h!]
  \centering
  \includegraphics[width=.45\textwidth]{img/barao/bar.jpg}
\end{figure}

\begin{itemize}
\item \textbf{Battataria Suíça}
  \\Endereço: Av. Albino José B. de Oliveira, 2297
  \\Telefone: (19) 3201-1174
  \\Site: \url{www.battataria.com.br}
  \\
  \\Fazem entrega, possui batatas recheadas. É um pouco caro, mas vale a pena
  conferir. Terça-feira tem o bom ``terça em dobro'', onde você paga uma e leva
  duas batatas.

\item \textbf{Namaste Salad}
  \\Endereço: R. José Martins, 751
  \\Telefone: (19) 3289-4178
  \\
  \\Restaurante com comida caseira.

\item \textbf{De La Rua}
  \\Endereço: Av. Dr. Romeu Tórtima, 1331
  \\Telefone: (19) 99229-5218
  \\
  \\Em frente ao Wizard, vende comida mexicana.

\item \textbf{Bronco Burger}
  \\Endereço: R. Agostinho Pattaro, 199
  \\Site: \url{www.broncoburger.com.br}
  \\
  \\Lanchonete conhecida por colocar sua marca no pão dos sanduíches.

\item \textbf{Roots Burger}
  \\Endereço: Av. Santa Isabel, 369
  \\Telefone: (19) 98131-0660
  \\
  \\Food truck especializado em sanduíches de costela.
\end{itemize}

\begin{figure}[h!]
  \centering
  \includegraphics[width=.45\textwidth]{img/barao/burger.jpg}
\end{figure}

\subsection{Bar}

\begin{itemize}
\item \textbf{Casa São Jorge Bar}
  \\Endereço: Av. Santa Isabel, 655
  \\Telefone: (19) 3249-1588
  \\
  \\Aberto a partir das 18h de terça a domingo, possui música ao vivo com
  grande variedade. Mais ou menos perto da moradia.

\item \textbf{Echos Studio Bar}
  \\Endereço: R. Agostinho Pattaro, 56 e 64
  \\Telefone: (19) 3201-8900
  \\E-mail: \email{contato@echos.mus.br}
  \\Site: \url{echos.mus.br/studiobar}
  \\
  \\Um bar relativamente novo, possui apresentaçes ao vivo direto, que costumam
  ser de rock, blues ou jazz. Funciona de terça a sábado a partir das 19h.
  Conforme a semana passa, fecha cada vez mais tarde: 2h às terças, 4h às
  sextas e sábados.

\item \textbf{Rudá bar}
  \\Endereço: Av. Santa Isabel, 490
  \\Telefone: (19) 3249-3087
  \\E-mail: \email{rudabar@gmail.com}
  \\
  \\É um bar dançante, com apresentações de dança de salão e até aulas
  gratuitas, eventualmente.

\item \textbf{Bar do Zé}
  \\Endereço: Av. Albino J. B. de Oliveira, 1325
  \\Telefone: (19) 3289-3159
  \\
  \\Localizado bem em frente ao Pão de Açúcar, o pub tem apresentações ao vivo
  todas as semanas.
\end{itemize}

\subsection{Marmita}

\begin{itemize}
\item \textbf{Marmita da Tia Rita}
  \\Telefone: (19) 99477-3184

\item \textbf{Marmitex Copa e Cozinha}
  \\Telefone: (19) 3249-0153
  \\
  \\Não cobram taxa de entrega, R\$ 10 o marmitex e R\$ 15 a marmita grande.

\item \textbf{Hot-Dog e Lanches Tiozinho}
  \\Endereço: R. Angela Signori Grigol, 742
  \\Telefone: (19) 3365-3537
  \\
  \\Tem vários tipos de hot-dogs (com catupiry, com cheddar, com frango{\dots})
  O único problema é que cobram taxa de entrega para um lanche e fecham à
  meia-noite.

\item \textbf{Kalunga Lanches}
  \\Endereço: R. Sebastião Bonomi, 40
  \\Telefone: (19) 3289-5236
  \\
  \\Perto da moradia, não fazem entrega, mas ficam abertos até altas horas.
  Destaque para o caldinho de feijão. Obs: o lugar é limpo e bom.

\item \textbf{Lanchão \& Cia}
  \\Endereço: Av. Albino J. B. de Oliveira, 1214
  \\Telefone: (19) 3289-3665
  \\Site: \url{lanchao.com.br}
  \\
  \\Um dos melhores lanches de Campinas (quiçá o melhor). Os lanches geralmente
  são grandes e muito bons, e os preços são compatíveis com a qualidade e
  quantidade. Servem no carro se você preferir, com uma bandeja que fica presa
  no vidro. Fica no centro de Barão Geraldo, próximo ao Santander e Pão de
  Açúcar. Destaque para a batata frita, feita de uma forma muito diferente,
  extremamente crocante e quase cremosa por dentro.

\item \textbf{Burger King}
  \\Endereço: Av. Albino J. B. de Oliveira, 1000
  \\Site: \url{www.burgerking.com.br}
  \\Uma recomendação é o King em Dobro para dividir com amigas ou amigos, ou
  ainda comer por duas refeições: 2 Whoppers custam R\$ 15, por mais 10 reais
  você recebe a batata e refrigerante (refill), ou seja, R\$ 12,50 para cada
  pessoa se for dividir. Se preferir, pode pegar lanches menores mas ainda
  gostosos, como o Duplo Cheddar com Bacon.

\item \textbf{McDonald's}
  \\Endereço: Av. Albino J. B. de Oliveira, 1430
  \\Telefone: (19) 3289-0318
  \\
  \\Dispensa apresentações. Entregas das 11h às 23h. Costuma ficar aberto
  de madrugada, até as 4 da manhã.

\item \textbf{Barraquinhas}
  \\Há várias barraquinhas de hot-dog no centro de Barão e perto da Moradia.
  Destaque para o dog do terminal, o Hot Dog Central, o Pedrogue e o dogão da
  moradia. Se você quiser um lanche, uma boa pedida é o Star Tresh (Raimundão
  ou Guarujá, chame como você quiser), que fica perto do balão da Avenida 2 e
  costuma ficar aberto até altas horas. Perto da Unicamp, ao lado do posto
  Ipiranga que fica na avenida 1 também tem um dog prensado muito bom e barato.

\item \textbf{Mega Sandubão}
  \\Endereço: Av. Albino J. B. de Oliveira, 2287
  \\Telefone: (19) 3288-0204
  \\
  \\Antigo Ponto Final. Lanchonete localizada na estrada da Rhodia
  (continuação da avenida Albino José de Oliveira) que entrega lanches até
  a meia noite. Muitos gostam bastante dessa lanchonete pela famosa maionese
  temperada que servem, não se esqueça de pedir quando for comprar lanches. À
  noite serve cerveja a um bom preço. Localiza-se na estrada da Rhodia.

\item \textbf{Subway}
  \\Endereço: Av. Albino J. B. de Oliveira, 1556
  \\Telefone: (19) 3201-8411, (19) 3201-8410
  \\Site: \url{www.subdelivery.com.br}
  \\
  \\Vende dos mais variados tipos de lanches, muito bons e não tão caros. Do
  lado do Subway tem um caixa 24 horas que trabalha com os principais bancos.
  O Subway faz entregas em algumas regiões de Barão.
\end{itemize}

\subsection{Sorveteria}

\begin{itemize}
\item \textbf{Sorvete em Camadas}
  \\Endereço: R. Antônio Pierozzi, 16
  \\Telefone: (19) 3289-0387
  \\
  \\Sorveteria com sorvete artesanal, alguns sabores são 100\% vegetais,
  outros, ao leite. O açaí é bem gostoso. Possuem pouca variedade e é bem
  caro.
\item \textbf{Caramelle Doceria}
  \\Endereço: Av. Dr. Romeu Tórtima, 1140
  \\Telefone: (19) 3395-3348
\end{itemize}
