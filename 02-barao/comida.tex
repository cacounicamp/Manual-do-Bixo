% Este arquivo .tex será incluído no arquivo .tex principal. Não é preciso
% declarar nenhum cabeçalho

\section{Comida dentro da Unicamp}
\subsection{Os restaurantes universitários da Unicamp}

Um dos momentos de glória do dia de uma futura engenheira, cientista ou
bacharel é o Bandejão. É a hora de intensas e indiscutíveis emoções. Caso sua
salada corra sobre a mesa, mantenha-se calma. Evite discussões, jamais tente
descobrir o sabor do suco pelo paladar (caju ou manga?). É mais cômodo ler no
cardápio do dia. Uma dica: para cortar o bife faça muita força e, quando
começar a amolecer, pare, você chegou na bandeja.

Falando sério agora: o RU (Restaurante Universitário), ou Bandejão, ou ainda
Bandeco, fica ao lado da Biblioteca Central, bem em frente ao PB (Prédio
Básico, ou Ciclo Básico II) e, a menos que você não queira economizar uma boa
grana com comida, vai ser o lugar onde você vai estar na maioria dos seus
horários de almoço. Com o tempo, você vai ver que o Bandeco (ou qualquer um dos
restaurantes universitários) é o ``coração da Unicamp''. É o local de você se
encontrar com amigas e amigos (combinando ou não antes), contar os micos nas
aulas, jogar conversa fora e falar mal da comida, que nem é tão ruim assim como
muitos dizem. Sem dúvida, é o melhor custo-benefício da Unicamp. Por R\$2,00,
você tem direito a arroz, feijão, pão, salada, proteína de soja, suco e café à
vontade. A carne e a sobremesa tem que dar uma choradinha para a tia ou o tio
para poder repetir, mas geralmente dá certo.

\begin{figure}[h!]
    \centering
    \includegraphics[width=.45\textwidth]{img/barao/bandeco.jpg}
\end{figure}

Existe também o RA (Restaurante Administrativo, não confundir com registro
acadêmico), também conhecido como Prateco, pelo fato de a comida ser servida
em pratos e não em bandejas, e o RS (Restaurante da Saturnino), o já não tão
novo restaurante universitário, que tem este nome por estar na rua Saturnino de
Brito.

O RA fica atrás da Faculdade de Engenharia Elétrica e de Computação (FEEC),
perto do prédio da Engenharia Básica. Em comparação com o Bandeco, o espaço
físico é bem menor, mas você mesmo se serve, apesar da carne ser servida pela
tia ou tio que trabalha lá, assim como no RS, que, por sua vez, fica perto do
IC-3 e, sem dúvida, é a melhor opção se você não se importar em fazer uma
caminhada, já que o ambiente é bem menos claustrofóbico que o RU e RA e é bem
longe.

Dependendo de onde você vai ter aula antes ou depois do almoço, é melhor
almoçar no RU, RS ou RA, mas, para isso, você precisa ter seu Cartão
Universitário (também chamado de RA) carregado, explicaremos como isso funciona
em breve.

\subsubsection{Horário de funcionamento dos restaurantes universitários}

Os restaurantes funcionam de segunda a sexta, nos seguintes horários:

\begin{itemize}
\item RU, das 7h às 8h30 (café da manhã), das 10h30 às 14h (almoço) e das 17h30
 às 19h45 (jantar).
\item RA, das 11h30 às 14h (almoço) e das 17h30 às 19h (jantar).
\item RS, das 11h às 14h (almoço) e das 17h30 às 19h (jantar).
\end{itemize}

Em períodos especiais, como fim de ano, os restaurantes podem funcionar em
horários reduzidos ou não abrem, fique de olho nos emails e informes que você
recebe. Uma recomendação é curtir a página da Prefeitura Universitária da
Unicamp no Facebook, pois sempre postam algumas mudanças em serviços da
Unicamp. Geralmente o RU é o único que permanece aberto.

Para saber previamente o cardápio do Bandejão, acesse o site da Prefeitura do
Campus (\url{prefeitura.unicamp.br}), o GDE (\url{gde.ir}) ou o aplicativo de
Serviços da Unicamp, feito pela CCUEC, você poderá ver seu saldo no cartão, o
cardápio, suas notas após o fechamento do semestre etc.

\subsubsection{Como funciona o esquema de carregar o cartão?}

Simples. Você vai a uma das máquinas que existem na entrada dos restaurantes,
insere seu RA e coloca cédulas na máquina, ela irá recusá-las, como faria
qualquer outra máquina de refrigerante que você vê em filmes ou já utilizou
quando era menor, e você tenta novamente com a cédula virada de ponta cabeça
até funcionar. Alguns dizem até que é um processo aleatório. Infelizmente as
máquinas só aceitam notas e não existe troco.

Outra maneira de colocar créditos é fazer um depósito na conta do Bandeco no
Santander (Ag.: 207 / Conta: 43.010.009-2) ou no Banco do Brasil (Ag.: 4203-X /
Conta: 66.315-8) e, para carregar, dirigir-se ao Posto de Atendimento situado
no RU, das 9h às 14h e das 15h às 17h, apresentando o comprovante de depósito.
Não são aceitos comprovantes de pagamento de entrega de envelope ou via
internet.

\subsubsection{Cardápio vegetariano}

Um cardápio ovo-lacto-vegetariano está sendo oferecido desde o fim de 2013. Ele
está disponível apenas no RS, numa fila separada. Agora, é possível saborear
pratos tais como abobrinha agridoce, hambúrguer de soja, legumes com molho
branco, torta de aveia, entre outros, mas não se anime muito, há várias
críticas com relação a variedade.

No período de férias, normalmente o RS e RA estarão fechados e o cardápio
vegetariano passa a ser servido no RU. Não há fila separada, você precisa pedir
por ele em vez da carne.

\subsubsection{Café da manhã}

Em março de 2016, houve o início das operações do RU no período da manhã,
atendendo a uma demanda de estudantes. De acordo com o ex-reitor Tadeu Jorge,
a Unicamp investiu cerca de R\$ 70 mil para equipar os restaurantes para
oferecer o café da manhã. O período é das 7h às 8h30.

O preço é apenas R\$ 1,00. O cardápio consiste de café, leite, pão, manteiga,
cada dia há uma geleia e fruta diferente, então divirta-se com suas amigas e
amigos tentando advinhar qual será a geleia do dia.

% Fim da seção dos restaurantes universitários

\subsection{Outros lugares na Unicamp}



\section{Comida fora da Unicamp}
