% Este arquivo .tex será incluído no arquivo .tex principal. Não é preciso
% declarar nenhum cabeçalho

\section{Diversão}

Está com vontade de fazer algo para livrar a cabeça do stress da vida
acadêmica? Como toda universidade, existem inúmeras opções de entretenimento e
diversão próximas a Unicamp, desde festas e bares até cinemas e teatros.

Se você gosta ou se interessa por jogos de tabuleiro, a dica é a luderia que
existe na Av. 2, o Metropoly. Lá você paga 15 reais por pessoa e joga a
vontade. A comida e bebida são um pouco caras, mas mesmo assim compensa. Se
você não sabe o que jogar, peça sugestões aos garçons de jogos.

O cinema mais próximo e mais conveniente é o do Shopping Parque Dom Pedro. Nas
segundas feiras, há 50\% de desconto para todos os ingressos, então, agora que
você é estudante, pagará um quarto do preço normal, é uma ótima opção para quem
quer se divertir sem gastar muito dinheiro.

Se você curte festas e baladas, fique atenta na saída do bandeco, pois é lá que
elas são divulgadas. A maioria das festas acontecem às quintas feiras em
repúblicas ou no Campinas Hall. Existem também festas dentro da Unicamp que,
apesar de serem proibidas, acontecem com certa regularidade.

De acordo com a pesquisa sobre custo de vida em Barão Geraldo realizada em
2017, 7\% das pessoas que participaram não saíam, 13\% saiam uma vez ao mês,
26\%, duas vezes, 9\%, três vezes, 22\%, quatro vezes e um pouco menos de um
terço saíam mais de 4 vezes ao mês. Dessas pessoas, 3\% diziam gastar menos de
10 reais, 15\% gastam entre 10 e 20 reais, 39\%, entre 20 e 30 reais, 28\%,
entre 30 e 40 reais e, ainda, 15\% gastam mais que 40 reais em rolês.

\subsection{Espaços culturais}

\begin{itemize}
\item \textbf{Casa do Lago:} A Casa do Lago é um espaço cultural dentro da
  Unicamp que promove espetáculos artísticos, oficinas culturais, seminários e
  debates acadêmicos aproximando os universitários, professores, funcionários e
  comunidade local. Todos os dias tem uma sessão de cinema das 16h às 19h e são
  oferecidas oficinas culturais gratuitas como aulas de violão, teatro, yoga,
  ballet etc. Para mais informações, visite a página
  \url{www.casadolago.preac.unicamp.br}

\item \textbf{Semente:} Fica no fim da avenida Santa Isabel, depois da moradia.
  Sempre tem apresentações artísticas, como teatros e espetáculos musicais.
\end{itemize}

\subsection{Cinemas}

\begin{itemize}
\item \textbf{Kinoplex}
  \\Endereço: Shopping Parque D. Pedro (Rodovia Dom Pedro I, Km 137 -- Jd. Sta.
  Genebra)
  \\Telefone: (19) 3131-2800
  \\Site: \url{kinoplex.com.br}

\item \textbf{Cinemark Iguatemi}
  \\Endereço: Shopping Center Iguatemi (Av. Iguatemi, 777 -- Vila Brandina)
  \\Site: \url{cinemark.com.br}

\item \textbf{Box Cinépolis Campinas}
  \\Endereço: Campinas Shopping (Rua Jacy T. de Camargo, 940 -- Jardim do Lago)
  \\Telefone: (19) 3268-2288
  \\Site: \url{cinepolis.com.br}

\item \textbf{Cine Galleria}
  \\Endereço: Galleria Shopping (Rod. Dom Pedro I, Km, 131,5 -- Jd. Nilópolis)
  \\Telefone: (19) 3207-1333
  \\Site: \url{bit.ly/2jkvsgb}

\item \textbf{Cine Moviecom Unimart}
  \\Endereço: Shopping Unimart (Av. John Boyd Dunlop, 350 -- Chácara da
  República)
  \\Telefone: (19) 3744-5000
  \\Site: \url{moviecom.com.br}

\item \textbf{Multiplex Parque das Bandeiras}
  \\Endereço: Shopping Parque das Bandeiras (Av. John Boyd Dunlop, 3900)
  \\Telefone: (19) 3728-4321/ (19) 3728-4320
  \\Site: \url{www.cinearaujo.com.br}

\item \textbf{Topázio Cinemas}
  \\Endereço: Shopping Prado (Av. Washington Luís, 2480 -- Parque Prado)
  \\Telefone: (19) 3975-1857
  \\Site: \url{pradoboulevard.com.br/cinema.asp}
\end{itemize}

\subsection{Teatros}

\begin{itemize}
\item \textbf{Lume Teatro}
  \\Endereço:  Rua Carlos Diniz Leitão, 150 Vila Santa Isabel -- Barão Geraldo
  \\Telefone: (19) 3289-9869/3289-3155
  \\Site: \url{lumeteatro.com.br}

\item \textbf{Teatro Interno Luiz Otávio Burnier}
  \\Endereço: Centro de Convivência Cultural (Praça Imprensa Fluminense s/nº --
  Cambuí)
  \\Telefone: (19) 3232-5977

\item \textbf{Teatro de Arena}
  \\Endereço: Centro de Convivência Cultural (Praça Imprensa Fluminense s/nº --
  Cambuí)
  \\Telefone: (19) 3232-5977

\item \textbf{Teatro Carlos Maia}
  \\Endereço: Rua Cel. Quirino, 2 -- Bosque dos Jequitibás
  \\Telefone: (19) 3231-8795
  \\Site: \url{bit.ly/1muLXRz}

\item \textbf{Teatro José de Castro Mendes}
  \\Endereço: Praça Corrêa de Lemos, s/nº -- Vila Industrial
  \\Telefone: (19) 3272-9359
  \\Site: \url{bit.ly/1AHgZKw}

\item \textbf{Auditório Beethoven (Concha Acústica)}
  \\Endereço: Av. Heitor Penteado, s/nº -- Portão 2 -- Lagoa do Taquaral
  \\Site: \url{bit.ly/1tPxFyT}

\item \textbf{Teatro de Arte e Ofício}
  \\Endereço: Rua Conselheiro Antônio Prado, 529 -- Vila Nova
  \\Telefone: (19) 3241-7217 / (19) 98200-0149
  \\Site: \url{bit.ly/1xT55Lg}

\item \textbf{Teatro Dom Nery (Externato São João)}
  \\Endereço: Rua José de Alencar, 360  Centro
  \\Telefone: (19) 3231-2644
  \\Site: \url{bit.ly/1xH9Qsy}

\item \textbf{Teatro Teresa Aguiar (Conservatório)}
  \\Endereço: Rua José de Alencar, 701 -- Centro

\item \textbf{Teatro da Vila Padre Anchieta}
  \\Endereço: Av. Cardeal Dom Agnelo Rossi, s/nº -- Vila Padre Anchieta
  \\Telefone: (19) 3282-0024
  \\Site: \url{bit.ly/1DnEsTg}

\item \textbf{Centro Cultural Evolução}
  \\Endereço: Rua Regente Feijó, 1087 -- Centro

\item \textbf{Teatro Iguatemi}
  \\Endereço: Shopping Iguatemi (Av. Iguatemi, 777 -- Vila Brandina).
  \\Telefone: (19) 3294-3166
  \\Site: \url{www.teatroiguatemicampinas.com.br/}

\item \textbf{Teatro do SESC}
  \\Endereço: Rua Dom José I, 270 -- Bonfim (próximo à rodoviária nova)
  \\Telefone: (19) 3737-1500

\item \textbf{Teatro do SESI Amoreiras}
  \\Endereço: Av. das Amoreiras, 450 -- Parque Itália.
  \\Telefone: (19) 3772-4100

\item \textbf{Teatro SOTAC}
  \\Endereço: Rua Barão de Jaguara, 2 -- Bosque
  \\Telefone: (19) 3235-2266
  \\Site: \url{www.sotac.com.br}

\item \textbf{Teatro Sia Santa}
  \\Endereço: Rua Sebastião Paulino dos Santos, 20 -- Parque Santa Bárbara
  \\Telefone: (19) 3281-3174
  \\Site: \url{www.siasanta.art.br}
\end{itemize}

\subsection{Boates e baladas}

\begin{figure}[h!]
  \centering
  \includegraphics[width=.45\textwidth]{img/barao/boate.jpg}
\end{figure}

\begin{itemize}
\item \textbf{Cooperativa Brasil:} Para quem gosta de um bom forró, sempre com
  shows diversos. A galera gosta muito da quarta-universitária.

\item \textbf{Campinas Hall:} Muitas das festas mais legais da Unicamp
  acontecem lá (como a Festa Brega e a Festa do Contrário), perto da PUCC. É
  bem grande.
  \\Site: \url{campinashall.com.br}

\item \textbf{Barril da Máfia:} Programação musical bem variada e um clima bem
  legal.

\item \textbf{Kraft:} Localizada próxima ao Taquaral (na Avenida Imperatriz
  Leopoldina), toca musica psi a noite toda e fica aberta até quase o
  amanhecer. Mulher entra de graça até a meia-noite.

\item \textbf{Cambuí:} Neste bairro existem vários barzinhos, a maioria é
  temático, alguns são um pouco caros e cobram covert. É um ótima escolha para
  quem tiver carro pois fica um pouco longe de Barão.
\end{itemize}

\subsection{Shopping Centers}

\begin{figure}[h!]
  \centering
  \includegraphics[width=.45\textwidth]{img/barao/d_pedro.jpg}
\end{figure}

\begin{itemize}
\item \textbf{Shopping Parque D. Pedro:} Foi considerado o maior shopping da
  América Latina até pouco tempo atrás. Localiza-se na Rodovia Dom Pedro, km
  137 (razoavelmente próximo à Unicamp). O ônibus 3.38, que sai do Terminal
  Barão, vai para lá e para o Iguatemi. Outras opções de ônibus saindo do
  terminal são 2.10 e 3.00.
  \\Site: \url{parquedpedro.com.br}

\item \textbf{Shopping Iguatemi:} Shopping normal, o mais antigo e o segundo
  maior de Campinas. Localiza-se na Avenida Iguatemi, 777. O 3.38 demora uns 40
  minutos para chegar lá. Frequentado pela galera mais nova e pelo pessoal com
  um pouco mais de dinheiro.
  \\Site: \url{www.iguatemicampinas.com.br}

\item \textbf{Parque das Bandeiras Shopping:} Inaugurado no fim de 2012.
  Localiza-se na região do Campo Grande, região noroeste de Campinas (MUITO
  longe). Conta com telas de cinema de 300 m$^{2}$.
  \\Site: \url{shoppingparquedasbandeiras.com.br}

\item \textbf{Campinas Shopping:} Longe a dar com pau, mas as lojas não são
  muito caras. Localiza-se no Jardim do Lago, às margens das rodovias
  Anhanguera e Santos Dumont. Provavelmente você nunca irá lá.
  \\Site: \url{campinasshopping.com.br}

\item \textbf{Shopping Prado:} Shopping pequeno, porém as lojas são um pouco
  caras. É outro que fica muito longe e que você provavelmente nunca irá até
  lá. Localizado na Av. Washington Luís, 2480 -- Parque Prado.
  \\Site: \url{pradoboulevard.com.br}

\item \textbf{Galleria Shopping:} Muito bonito, mas lojas mui\-to caras. Também
  localizado na Rodovia Dom Pedro, mas no km 131,5. O ônibus 3.00 sai do
  terminal de Barão Geraldo e passa lá.
  \\Site: \url{www.galleria.com.br}

\item \textbf{Shopping Unimart:} Shopping pequeno, as lojas não são muito muito
  caras. Localiza-se na Avenida John Boyd Dunlop, 350. O ônibus 1.34 sai do
  terminal de Barão Geraldo e passa próximo.
  \\Site: \url{unimart.com.br}

\item \textbf{Shopping Spazio Ouro Verde:} Shopping peque\-no, inaugurado no
  fim de 2010. As lojas não são muito caras. É outro Shopping que também fica
  MUITO longe. Localiza-se na Av. Ruy Rodriguez, 3900, Parque Universitário
  (região do Ouro Verde).
  \\Site: \url{spazioouroverde.com.br}

\item \textbf{Ventura Mall:} Shopping pequeno, as lojas são muito caras. Fica
  localizado na Av. Moraes Salles, 2790 -- Nova Campinas.
  \\Site: \url{venturamall.com.br}
\end{itemize}
