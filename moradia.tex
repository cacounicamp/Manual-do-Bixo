% Este arquivo .tex será incluído no arquivo .tex principal. Não é preciso
% declarar nenhum cabeçalho

\section{Lugares para morar}

O preço de uma casa é a área do paralelepípedo formado pelos seguintes valores
nos eixos: proximidade da Unicamp, tamanho da casa (quantidade de quartos)
e qualidade da casa (acabamento, quantidade de banheiros, etc.) Quanto
à distância, a avenida 1 (avenida Dr. Romeu Tortima) e avenida 2 (avenida Prof.
Atílio Martini) são bem caras por serem próximas da Unicamp, as ruas entre elas
muitas vezes também são. A região que vai do centro de Barão até a moradia
geralmente é boa e barata para se morar: Tem bastantes serviços e ainda é perto
da Unicamp (mais ou menos 10 minutos de bicicleta).

Uma boa dica para se informar a respeito de lugares para morar (repúblicas,
kitnets, pensões) é o site Morar Unicamp
(\url{morarunicamp.com.br}), criado por alunos da Unicamp e que
contém informações como endereço, preço, contato e detalhamento do lugar. E mais
lugares podem ser adicionados ao site.

\subsection{Moradia Estudantil}

A moradia estudantil é um exemplo de conquista de todos os estudantes.
O processo de reivindicação de uma moradia estudantil para a Unicamp começou com
o movimento TABA. Durante muito tempo, alunos ficaram acampados no CB (Ciclo
Básico) para reivindicar seu direito a estudar.
\begin{figure}[h!]
    \centering
    \includegraphics[scale=0.55,keepaspectratio=true]{img/imgs/5-moradia/-029.jpg}
\end{figure}
Hoje em dia, graças à moradia,
várias pessoas que possivelmente não teriam condições de se manter em Campinas
pagando aluguel, tem como estudar na Unicamp. A moradia existe desde 1989
e durante esse período a Unicamp ampliou muito suas vagas. Isto fez as vagas na
moradias tornarem-se insuficientes para todos que querem. Alguns anos atrás
houve grandes movimentações pela ampliação e reforma da moradia, pois algumas
casas estavam caindo.

Cada "casa" (que normalmente é dividida entre quatro pessoas) se constitui de um
quarto, uma cozinha, um banheiro e uma sala. Ainda tem o Circular da Moradia, um
ônibus da Unicamp que transporta a galera durante o dia todo da moradia até
a Unicamp e vice-versa.

A moradia está localizada na Avenida Santa Isabel, 1125. Aproximadamente a uns
3 km do campus da Unicamp.

Para saber mais sobre o processo seletivo entre no site da Moradia Estudantil
(\url{www.pme.unicamp.br}). Muito provavelmente você
será abordado por um assistente social na matrícula.

\subsection{Repúblicas}

Geralmente a melhor escolha, se você tiver condições de pagar por uma moradia.
\begin{figure}[h!]
    \vspace{-10pt}
    \centering
    \includegraphics[scale=0.58,keepaspectratio=true]{img/imgs/5-moradia/-034.jpg}
    \vspace{-10pt}
\end{figure}
Você pode levar quem quiser para sua casa (dependendo da aprovação dos
moradores), chegar no horário que quiser, e conhecerá muita gente nova. Tente,
se possível, morar em uma república de cursos mistos, pois assim você terá
contatos diversos. O custo de uma vaga em uma república é muito variável
(depende do nível de conforto que você quer). Gira em torno de
R\$350,00 o aluguel para dividir quarto e R\$450,00 para pegar
um quarto sozinho, além das contas adicionais (água, luz, etc.), que costumam
totalizar R\$150,00, em média. Procure bem as pessoas com quem você vai morar para não ter
problemas com diferentes estilos de vida (tem gente que gosta de lavar louça
a cada 5 minutos e tem gente que gostaria de morar junto com porcos, veja com
quem você se dá melhor).

\subsection{Kitnets}

Tomem cuidado com elas, pois a especulação imobiliária em Barão Geraldo chega
a ser imbecil. E nos últimos anos ficou fora de controle. As kitnets mobiliadas,
normalmente um quarto-sala-cozinha-área de serviço e banheiro, estão com valores
da ordem de R\$ 1000,00 para mais. Sim, mil reais por um microespaço. Só porque
é perto da Unicamp. Fique de olho e tome cuidado com os contratos.
\begin{figure}[h!]
    \centering
    \includegraphics[scale=0.55,keepaspectratio=true]{img/imgs/5-moradia/-033.jpg}
\end{figure}

As melhores relações custo-benefício de kitnets são aquelas próximas ao centro
de Barão Geraldo ou no espaço entre as avenidas. E lembre-se não é só de Unicamp
que se vive. Não adianta pagar mais caro para estar do lado da Unicamp, se você
fica muito longe dos mercados, farmácias etc.

\subsection{Pensões}

Dependendo da pensão que você conseguir pode tornar-se uma grande roubada.
Algumas pensões não deixam você levar pessoas para sua casa, reclamam se você
chegar tarde e não liberam festas, já outras não; então procure bem. O preço
também não é muito bom, mas costuma ser mais barato que kitnets. É bom se você
quer um esquema casa-comida-roupa-lavada. É muito importante que você saiba que
contratos de um ano (ou qualquer período) em pensionatos são ilegais e você não
precisa cumpri-los.

\subsection{Dicas de Segurança}

Além de um novo ambiente de estudos, conhecerá novos lugares e pessoas. E,
provavelmente, em breve estará planejando sua mudança para Campinas.

Muitos dos estudantes moram em Barão Geraldo, por ser mais perto da Unicamp,
possibilitando que sua bicicleta se torne seu meio de transporte principal, por
ser o lugar com maior número de repúblicas e pensionatos e também por ser onde
a maior parte das festas acontece!

No entanto, por ser constituído em sua maioria por casas de famílias com muitas
posses e casas de estudantes (em geral desatentos), Barão Geraldo peca pela
falta de segurança.

Não é raro ouvir de alguém que foi assaltado enquanto voltava para casa à noite
sozinho ou que a casa foi saqueada durante um feriado prolongado. Portanto,
é importante zelar pela sua integridade e de seus pertences -- assim como seus
pais fazem em sua casa, não importa onde eles morem.
\begin{figure}[h!]
    \centering
    \includegraphics[scale=0.55,keepaspectratio=true]{img/imgs/5-moradia/seguranca.jpg}
\end{figure}

Se sua pensão ou república paga o segurança da rua, faça uso dele, seja pedindo
escolta ao chegar em casa ou telefonando caso ouça algum barulho suspeito. Ao
voltar para sua cidade em feriados prolongados, deixando a casa vazia, não se
esqueça de trancar todas as portas e janelas de casa, verificar se não há nada
no quintal que possa ser levado facilmente (colocar as bicicletas e aparelhos de
som na sala pode ser uma boa ideia), e trancar os objetos de valor
(computadores, televisões) nos quartos.

\subsection{Imobiliárias}

\begin{itemize}
\item   \textbf{Imobiliária Barão Housing}
        \\Telefone: (19) 3289-4113
        \\Endereço: Rua Tranquilo Prosperi, 383
        \\E-mail: \email{atendimento@baraohousing.com.br}
        \\Site: \url{baraohousing.com.br}

\item   \textbf{Imobiliária Lanza}
		\\Endereço: Rua Benedito Alves Aranha, 104
		\\Telefone: (19) 3289-1717 / (19) 3307-7155
		\\E-mail: \email{lanza@lanzaimoveis.com.br}
		\\Site: \url{lanzaimoveis.com.br}

\item   \textbf{Imobiliária Professor Sebastião}
		\\Endereço: Av. Dr. Romeu Tortima, 344
		\\Telefone: (19) 3289-2317
		\\E-mail: \email{ipsimoveis@ipsimoveis.com.br}
		\\Site: \url{ipsimoveis.com.br}

\item   \textbf{Amaral Imóveis}
		\\Endereço: Av. Dr. Luiz de Tella, 864
		\\Telefone: (19) 3287-0655 / (19) 4141-1010
		\\E-mail: \email{amaral@amaralimoveis.net}
		\\Site: \url{amaralimoveis.net}

\item   \textbf{Zaine Conquista Imóveis}
		\\Endereço: Av. Santa Isabel, 84
		\\Telefone: (19) 3289-4050 / (19) 3289-2761
		\\E-mail: \email{zaine@correionet.com.br}
		\\Site: \url{zaineconquista.com.br}

\item   \textbf{W. Post Imóveis}
		\\Endereço: Av. Dr. Romeu Tortima, 1140
		\\Telefone: (19) 3289-2155
		\\E-mail: \email{w.post@ig.com.br}
        % Link Quebrado; aparentemente alguem não pagou o servidor... Talvez
        % volte a funcionar no futuro.
        %\\Site: \url{wpostimoveis.com.br}  

\item   \textbf{Ismê Assessoria Imobiliária}
		\\Endereço: Rua Christina G. Miguel, 250
		\\Telefone: (19) 3289-4325
		\\E-mail: \email{isme@isme.com.br}
		\\Site: \url{isme.com.br}

\item   \textbf{Rute Svartman Imóveis}
		\\Endereço: Rua Engenheiro Edward de Vita Godoy, 850
		\\Telefone: (19) 3368-0881
		\\E-mail: \email{imoveis@rutesvartman.com.br}
		\\Site: \url{rutesvartman.com.br}

\item   \textbf{Imobiliária Ávila \& Ferraris}
		\\Endereço: Av. Dr. Romeu Tortima, 714
		\\Telefone: (19) 3289-3522
		\\E-mail: \email{dcaavila@terra.com.br}
		\\Site: \url{avilaeferrarisimoveis.com.br}

\item   \textbf{Imobiliária Cidade Universitária}
		\\Endereço: Av. Dr. Romeu Tortima, 624
		\\Telefone: (19) 3289-3322
		\\E-mail: \email{contato@cidadeuniversitariaimoveis.com.br}
		\\Site: \url{cidadeuniversitariaimoveis.com.br}

\item   \textbf{Denilson Imóveis}
		\\Endereço: Av. Dr. Luís de Tella, 55
		\\Telefone: (19) 3289-1444
		\\E-mail: \email{contato@denilsonimoveis.com.br}
		\\Site: \url{denilsonimoveis.com.br}

\item   \textbf{Mega Barão Imóveis}
		\\Endereço: Rua Francisca Resende Merciai, 103 B
		\\Telefone: (19) 3289-7101 / (19) 3386-4141
		\\E-mail: \email{megabarao@megabaraoimoveis.com.br}
		\\Site: \url{megabaraoimoveis.com.br}

\item   \textbf{Libano Imóveis}
		\\Endereço: Rua Francisca Resende Merciai, 90
		\\Telefone: (19) 3789-9999
		\\E-mail: \email{contato@libanoimoveis.com.br}
		\\Site: \url{libanoimoveis.com.br}

\item   \textbf{Marco Imóveis}
		\\Endereço: Rua José Pugliesi Filho, 420
		\\Telefone: (19) 3287-8083
		\\Site: \url{marcoimovel.com.br}

\item   \textbf{Imobiliária Canal}
		\\Endereço: Av. Professora Ana Maria Silvestre Adade, 555
		\\Telefone: (19) 3256-2622

\item   \textbf{Lokal Imóveis}
		\\Endereço: Rua José Próspero Jacobucci, 290
		\\Telefone: (19) 3256-4616

\item   \textbf{Carpe Diem Imóveis}
		\\Endereço: Av. Dr. Romeu Tortima, 184
		\\Telefone: (19) 3579-5655 / (19) 3304-9323
		\\Site: \url{carpediemimoveis.com.br}

\item   \textbf{Cássio Carvalho Imóveis}
		\\Endereço: Av. Santa Isabel, 750
		\\Telefone: (19) 3288-0143
		\\E-mail: \email{cassio@cassioimoveis.com.br}
		\\Site: \url{cassioimoveis.com.br}

% não achei referencia dessas imobiliarias pelo google...
% esperando resultado das ligações do gagau pra decidir se vai remover

\item   \textbf{Esparta Empreendimentos Imobiliários}
        \\Endereço: Av. Albino J. B. de Oliveira, 830
        \\Telefone: (19) 3289-5353

\item   \textbf{Roma Imóveis}
        \\Endereço: Rua Agostinho Pattaro, 222
        \\Telefone: (19) 3289-0441

\item   \textbf{Valter Imóveis}
        \\Endereço: Rua Maria Ferreira Antunes, 22
        \\Telefone: (19) 3289-6088

\item   \textbf{Imobiliária Marco Antônio}
        \\Endereço: Av. Dr. Romeu Tortima, 1522
        \\Telefone: (19) 3289-2417

\item   \textbf{André Imóveis}
        \\Endereço: Rua Maria Nassif Mokarzel, 42
        \\Telefone: (19) 3289-6607

\end{itemize}
