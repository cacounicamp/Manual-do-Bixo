\section{ITCP (Incubadora Tecnológica de Cooperativas Populares)}

\begin{wrapfigure}{L}{0.35\linewidth}
\includegraphics[width=0.9\linewidth]{imagens/itcp.png}
\end{wrapfigure}
A Incubadora Tecnológica de Cooperativas Populares (ITCP) é uma rede nacional, que na Unicamp está vinculada à Pró-Reitoria de Extensão e Cultura (ProEC), enquanto um programa de extensão universitária. Há duas décadas, desenvolvemos ações orientadas pelos princípios da Educação Popular, da Autogestão e da Extensão Comunitária e Popular, buscando fortalecer a ação territorial e a geração de trabalho e renda a partir da formação e do apoio a Empreendimentos Econômicos Solidários (como cooperativas e associações), grupos informais de trabalho, movimentos sociais, espaços culturais e comunitários da Região Metropolitana de Campinas. Levando em conta as necessidades desses territórios, propõe-se uma parceria que atenda essas demandas, que fortaleça os caminhos a serem percorridos, que possibilite espaços democráticos, de autonomia, bem como a troca e a construção coletiva de saberes. Nesse rico processo, que é interdisciplinar, são produzidos novos conhecimentos e tecnologias sociais, necessários à transformação da realidade. Anualmente, realizamos nosso processo seletivo para estágio, além de ofertarmos algumas bolsas de extensão (para graduação e pós-graduação), bolsas BAS e BAEF.

\begin{tags}

    \instagram{itcp\_unicamp}
    \sep
    \url{https://www.proec.unicamp.br/itcp/}
\end{tags}