\section{GER - Grupo de Estudos em Robótica}

\begin{wrapfigure}{L}{0.35\linewidth}
\includegraphics[width=0.9\linewidth]{imagens/ger.png}
\end{wrapfigure} 
O GER, Grupo de Estudos em Robótica, é uma extra-curricular com foco em robôs autônomos. Somos mais de 20 alunos da Unicamp de diferentes cursos, como das diversas Engenharias, de Física e de Letras, com o objetivo comum de aprender sobre robótica, programação, eletrônica e afins. 

Buscamos, além de colocar os conhecimentos adquiridos na graduação em prática, também compartilhar um pouco do que aprendemos com a comunidade externa à universidade. Para isso, fazemos oficinas em escolas públicas da região por meio do nosso projeto social afim de despertar interesse na área de tecnologia e ainda divulgar as oportunidades da universidade pública.

Atualmente, além das partes administrativas e de comunicação, temos 4 projetos:
\begin{enumerate}
    \item O VSSS, Very Small Size Soccer, ou futebol de robô;
    \item O Drone;
    \item O Projeto Social;
    \item O Humanoide.
\end{enumerate}
Se interessou por algum projeto? Quer participar de competições nacionais, exibições e eventos de divulgação de conhecimento conosco? Ou simplesmente não quer ser surpreendido por robôs autônomos tomando conta do nosso mundo no futuro? Vem com a gente e \#BORAFAZERROBÔ
\\ 

\begin{tags}
    \instagram{ger.unicamp}
    \sep
    \url{https://beacons.ai/ger_unicamp/home}
\end{tags}