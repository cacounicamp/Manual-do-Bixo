\section{LKCamp}

\begin{wrapfigure}{L}{0.35\linewidth}
\includegraphics[width=0.9\linewidth]{imagens/lkcamp.jpg}
\end{wrapfigure}
O LKCamp é um grupo de estudos de software livre focado no kernel Linux. Suas atividades são abertas para todas as pessoas interessadas em 
contribuir para este projeto.

O kernel Linux é a parte principal de muitos dos sistemas operacionais utilizados na atualidade, como as distribuições Linux (Ubuntu, Fedora, Mint, Arch, etc) para computadores e o Android para celulares.

O kernel é o núcleo do sistema operacional, sendo responsável por gerenciar o hardware do dispositivo, além de oferecer uma interface abstraída do hardware para os programas que estejam rodando, permitindo que eles utilizem os recursos da máquina (GPU, tela, etc) de forma mais simples.

Temos como objetivo principal ajudar pessoas a se tornarem contribuidoras do kernel 
Linux. O grupo organiza ao longo do ano oficinas e hackathons para toda 
a comunidade (seja ela da Unicamp ou não).

\begin{tags}
    \url{https://lkcamp.dev}
    \sep
    \telegram{lkcamp}
\end{tags}