\section{Olá, Mundos!}

\begin{wrapfigure}{L}{0.35\linewidth}
\includegraphics[width=0.9\linewidth]{imagens/olamundos.png}
\end{wrapfigure}
O Olá, Mundos! é um projeto de extensão do Instituto de Computação que reúne estudantes de Ciência e Engenharia da Computação com um propósito transformador: Democratizar o ensino de computação para jovens de baixa renda.

Nosso principal objetivo é promover aulas semanais de introdução a algoritmos e programação usando a linguagem de programação Python em escolas públicas localizadas em Campinas, tornando o acesso desses conteúdos mais acessíveis e importante para o futuro dos estudantes.

Juntos, consolidamos nossos conhecimentos em algoritmos, marketing e metodologias de ensino. Mais do que isso, crescemos como indivíduos, desenvolvendo maior consciência, empatia e senso de responsabilidade social.

O projeto é dividido em três principais áreas:

Educacional: Cria e planeja os tópicos e materiais que são usados em sala de aula, além de implementar novos projetos e iniciativas educacionais.

Marketing: Gerencia as redes sociais, criando conteúdo e interagindo com o público, além de buscar novas parcerias e oportunidades de colaboração.

Executivo: Supervisiona a saúde, bem-estar e integração dos membros, gerencia as finanças e registra as horas de extensão.

Mas não se preocupem! Todos os membros do projeto têm a oportunidade de dar aulas, independentemente da área escolhida.

Venham fazer parte Olá, Mundos! para viver experiências ótimas, conhecer novas pessoas e aprender bastante! Fiquem atentos as datas do nosso processo seletivo =)

\begin{tags}
    \instagram{olamundos}
    \sep
    \url{https://www.linkedin.com/company/ola-mundos}
\end{tags}