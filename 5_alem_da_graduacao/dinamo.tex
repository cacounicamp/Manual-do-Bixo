\section{Coletivo Dínamo de Engenharia Popular}

\begin{wrapfigure}{L}{0.35\linewidth}
\includegraphics[width=0.9\linewidth]{imagens/dinamo.png}
\end{wrapfigure}
O Coletivo Dínamo de Engenharia Popular foi criado por estudantes da UNICAMP em 2019 com o objetivo de disputar uma Extensão que seja popular, democrática e que possibilite a criação de Tecnologias Sociais em conjunto com comunidades aos arredores de Campinas.

Para nós, muito além de ser uma atividade assistencialista e de voluntariado, a Extensão deve cumprir um papel triplo: permitir a aplicação prática das ciências por estudantes, avançar as lutas das comunidades e movimentos sociais a partir da criação conjunta de tecnologias, as Tecnologias Sociais, que construam a autonomia dessas comunidades e criar um espaço onde as engenharias e outras ciências exatas se comprometam politicamente com o avanço da luta dos movimentos sociais.

Da nossa fundação até hoje, atuamos em diferentes comunidades em Campinas e região, começando pela Ocupação Nelson Mandela, próxima a Viracopos, onde contribuímos com um trabalho de concepção e construção de uma nova sede para a comunidade, em conjunto com o Escritório Modelo de Arquitetura e Urbanismo da UNICAMP (Móbile).

Atualmente nosso trabalho se concentra em dois locais distintos de atuação: o Acampamento do MST Marielle Vive!, em Valinhos, e na Casa de Cultura Tainã, em Campinas.

No Marielle Vive! colaboramos em diferentes projetos nas frentes de infraestrutura, produção, saneamento, etc. Na Casa de Cultura Tainã, estamos participando de um projeto na área da computação e programação, com o objetivo de conectar várias comunidades tradicionais à Rede Mocambos.

O Dínamo também se propõe ser e promover um espaço de discussão política e social para as ciências exatas, propondo ações menores e eventos de caráter formativo sobre os mais diversos assuntos de interesse social, como a relação das Engenharias e as pautas de raça e gênero, o papel da tecnologia no mundo moderno, a soberania nacional e o internacionalismo.

Venha fazer parte do Dínamo e construir uma Extensão e uma Engenharia preocupadas com a luta e a transformação social!

Sigam o Dínamo no Instagram: @dinamo.engenhariapopular

Lutar! Criar! Engenharia Popular!

\begin{tags}
    \instagram{dinamo.engenhariapopular}
\end{tags}