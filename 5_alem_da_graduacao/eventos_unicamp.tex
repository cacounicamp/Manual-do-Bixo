\section{Eventos da Unicamp}

\subsection{Semanas da Unicamp}

Alguns cursos da Unicamp realizam anualmente um evento chamado de “Semana” em que es alunes de graduação têm contato com o mercado de trabalho, com pesquisas, tendências e novidades dos cursos e demais assuntos relacionados. Para isso, participam desses eventos ex-alunes e profissionais, realizam-se palestras e minicursos, são feitas visitas a empresas e debates. O evento da computação é a SECOMP. Recomendamos conhecer participar das semanas de outros cursos também!

\subsection{UPA - Unicamp de Portas Abertas}

\begin{wrapfigure}{L}{0.35\linewidth}
\includegraphics[width=0.9\linewidth]{placeholder.jpg}
\end{wrapfigure}

A UPA é um evento anual, em que, durante o dia, a Unicamp é apresentada para estudantes dos ensinos fundamental e médio de todo o país – muitos dos quais pré-vestibulandes. A apresentação da universidade é feita por professores e alunes que mostram as salas de aula e as pesquisas realizadas. O IC costuma realizar uma recepção com forte participação de alunes. E essa é uma oportunidade muito massa para você! Além de ser divertido,  o IC geralmente oferece uma pequena remuneração pelo trabalho. Participe! Para saber mais sobre o evento, acesse \textbf{www.upa.unicamp.br}.

\subsection{Feira do Livro}

As feiras do livro da Editora da Unicamp têm o objetivo de incrementar a biblioteca dos leitores e de estimular o acesso ao conhecimento que os livros da Editora proporcionam. Ao longo do evento, geralmente rolam lançamentos, promoções, venda de kits temáticos, rodas de conversa sobre a cultura do livro e sobre a situação da leitura no Brasil e produções especiais.
Em 2024, a Feira teve sua quinta edição. O evento durou três dias e contou com uma programação diversa incluindo programação cultural, lançamentos de novos livros publicados pela Editora da Unicamp, sessão de autógrafos e roda de conversa. Comumente acontece no pátio do PB (Ciclo Básico II). Para saber mais sobre o evento:
\textbf{festadolivroeditoraunicamp.com.br/} 