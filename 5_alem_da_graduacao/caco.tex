\section{CACo, o Centro Acadêmico da Computação}

\begin{wrapfigure}{L}{0.35\linewidth}
\includegraphics[width=0.9\linewidth]{imagens/caco.png}
\end{wrapfigure}

Desde o final do século passado, com a consolidação do capitalismo em sua forma neoliberal como a principal forma de gestão do mundo, o mundo assistiu a classe trabalhadora ter seus direitos atacados e suas formas de organização desmanchadas, produzindo um sentimento de desesperança generalizado, absorvido e promovido através das artes, no audiovisual, na literatura, na música, e nas artes plásticas. Com o fim do século das revoluções sociais, é decretado o “fim da história”, e somos constantemente levados a crer que não há alternativa ao sistema em que vivemos.

A partir de então, as classes dominantes conquistam seu grande triunfo: a completa responsabilização do indivíduo por problemas sociais que são reproduzidos de forma estrutural. A fome, o desemprego, a falta de tempo para viver além do trabalho, a distância quilométrica do “sucesso”, o adoecimento psíquico e a deterioração das relações interpessoais são colocadas sob a responsabilidade do indivíduo, o responsável por “escrever sua própria história”. Assim, como forma de contornar o sentimento constante de desesperança, a ideologia dominante oferece alternativas fáceis e que não colocam em risco a ordem social que nos oprime, adoece e mata: o discurso meritocrático e de empreendedorismo promovido por coachs e startups, a promoção da educação financeira como solução dos problemas materiais, o endeusamento das iniciativas individuais na solução de problemas tão complexos quanto as questões socioambientais, assim como outros exemplos que podem ter vindo à sua mente. Diante desse cenário de terra arrasada, o que nos resta?

Ao terrorismo da individualidade, nós, no CACo, opomos a esperança na coletividade como forma de construção do sonho, através da organização coletiva e da construção de espaços democráticos e seguros para todes, mas em especial es mais afetadies pelo sistema de exploração em que vivemos. A partir da escuta e do diálogo, buscamos construir a luta política em torno dos interesses des estudantes e da classe trabalhadora, sem nunca perder de vista o horizonte mais amplo de superação desta ordem social predatória. Para nós, a política não se resume a acordos fechados sem participação popular, mas é, pelo contrário, o tecido que envolve todas as nossas vidas.

Justamente por crermos nisso é que atuamos mobilizando nossa comunidade em diversas frentes. Nossa principal tarefa enquanto Centro Acadêmico é observar as demandas des estudantes dos nossos Institutos - o IC e a FEEC - atuando na representação destes nos órgão dos institutos, como a congregação e as comissões de graduação, e da própria universidade, como a Comissão Central de Graduação.  Atuamos também levando reclamações e sugestões às avaliações de curso semestrais, e quando necessário, levando denúncias do corpo estudantil contra professores, funcionários, e mesmo colegas de curso ou faculdade. E, é claro, representamos es estudantes da computação dentro do Movimento Estudantil, através do Conselho de Representantes de Unidade (CRU), e nas assembleias estudantis.

Fora isso, atuamos ainda oferecendo assistência aos estudantes dentro das nossas capacidades, tirando dúvidas, trazendo a tona debates políticos relevantes dentro da comunidade, promovendo eventos educativos e formativos, realizando eventos culturais, artísticos e de integração, e nos articulando com outras entidades do instituto e da universidade.  Algumas das ações promovidas por nós são:

\begin{itemize}
    \item O LariCACo, uma cantina onde revendemos doces, sucos, bolachas, salgadinhos, e outras comidas do tipo por valores acessíveis e próximos do preço de custo. É pautado pela ideia da economia solidária, e os pagamentos são feitos a partir de um sistema de confiança, onde você paga o que comprou usando pix ou dinheiro.
    
    \item A escrita, edição, impressão e distribuição deste Manual de Ingressante de forma 100\% gratuita, processo no qual temos a possibilidade de desenvolver nossas habilidades de escrita, design e diagramação, além de auxiliar es ingressantes a sobreviverem na universidade.
    
    \item Dentre os eventos lúdicos que realizamos, os principais são: o CACoGames, onde fazemos sessões de board games e videogames; o CineCACo, cineclube onde exibimos e conversamos sobre filmes, sejam relacionados à computação ou outras questões relevantes para nossa comunidade; e o já tradicional CACoPalooza, nosso evento cultural que conta com apresentações de diversas formas de arte, e oferece  um momento de descontração e integração, aberto para todes da universidade.
    
    \item Estamos presentes em órgãos representativos do IC, da FEEC, e na Comissão Central de Graduação da universidade. A atuação des nosses representantes é essencial para pensarmos nossa atuação e levarmos as demandas da comunidade, garantindo que nossa voz se faça ouvir nesses espaços que infelizmente restringem nossa participação.
    
    \item Estamos presentes na Unicamp de Portas Abertas, conversando com es vestibulandes e apresentando um pouco dos cursos de CC e EC junto das outras entidades. Além disso, no começo do ano organizamos a Palestra de Entidades e a Banca de Entidades, para que quem ingressa possa conhecer um pouco mais sobre as atividades extracurriculares e de extensão universitária.
    
    \item Organizamos anualmente a palestra AA/AB, que ajuda ingressantes da EC a escolherem entre uma das duas modalidades do curso. Ainda organizamos, por vezes, eventos como o curso de GIT, que serviu para familiarizar estudantes com essa ferramenta essencial na universidade e no mercado de trabalho.

    \item Conforme o avanço da luta estudantil, realizamos assembleias, reuniões abertas e grupos de estudo. O mais recente exemplo foi a criação do GT (Grupo de Trabalho) Antifascista, após os ataques fascistas na universidade, e que culminou na realização da 1ª Plenária Antifascista da Computação.

    \item Realizamos eventos, rodas de conversa e campanhas públicas a respeito de lutas anti-opressão, apoiando a luta antirracista, feminista, da população LGBTQIAPN+, e promovendo a consciência de classe nes futures trabalhadores da computação, buscando debater o papel da tecnologia na dominação de classe, e como ela pode melhorar a qualidade de vida da classe trabalhadora.
\end{itemize}

Depois de ler tanta coisa sobre o CACo, algumas perguntas podem surgir: Onde fica o CACo? Como se organiza? Como faz pra entrar? Por que o símbolo é um sapo? Bom, para começar, o CACo possui uma sala que fica no Instituto de Computação, ao lado do Jardim de Inverno: a salinha do CACo (o nome não é muito criativo, sabemos)! Lá temos televisão, jogos de carta, videogame, o LariCACo e um sofá bem confortável para tirar uma soneca. É importante ressaltar que o espaço é de uso comum para todes, e portanto é importante que todos tomemos cuidado com ela.

As nossas reuniões ordinárias (ou seja, periódicas) acontecem uma vez por semana de forma híbrida, e são abertas para es estudantes, quem quiser participar só precisa conversar com algum membro do CA. Lá debatemos nossas ações, planejamentos, e é onde o CA de fato acontece. Além disso, temos diversos Grupos de Trabalho que organizam eventos e ações, e com certeza vai ter algum pelo qual se interesse. 

Lembrando que, para participar do CA, não é necessário processo seletivo algum, apenas o compromisso de manter o espaço seguro para todes - não toleramos qualquer tipo de atitude de opressão!

Por fim, nosso símbolo é um sapo por conta do Caco, personagem dos Muppets que compartilha o seu nome conosco, e também porque os sapos são animais bonitinhos.

Acreditamos que depois desta apresentação, você conhece melhor o CACo e as ações promovidas por nós. Pode parecer bastante coisa, e de fato é! Só conseguimos realizar tantas atividades assim graças aos esforços de muitas pessoas com vontade de mudar a realidade ao seu redor e de plantar hoje a esperança de um futuro melhor. E se você também sente essa necessidade, gostaríamos de te convidar a conhecer melhor o nosso Centro Acadêmico, pois precisamos de gente como você!

Você pode entrar em contato conosco falando com es coordenadores da gestão e com es membres do CA. Nos colocamos à disposição, e estaremos sempre aqui por você! Todos os nossos links mais importantes estão no nosso linktr.ee: \textbf{linktr.ee/cacounicamp}. Nos siga também nas redes sociais: \textbf{@cacounicamp} e fique à vontade para mandar mensagem/e-mail para \textbf{caco.gestao@gmail.com} ou no Instagram.

\section{A Gestão SynCACo \(2024-2025\)}
\lipsum[1]

\section{Os cargos e coordenadores da gestão}

\begin{table}[hbtp!]
    \centering
    \rowcolors{2}{gray!10}{white}
    \bgroup
    \def\arraystretch{1.5}
    \color{black!75}
    \begin{tabular}{|m{0.3\linewidth} m{0.1\linewidth} m{0.5\linewidth}|}
        \hline
        \rowcolor{gray!10}
        José Victor & CC023 & Coordenador Presidente \\
        João Augusto & CC022 & Coordenador Administrativo \\
        Gabriel Soares & EC024 & Coordenador Financeiro \\
        Juca Magalhães & CC024 & Coordenador de Ensino e Graduação\\
        Michael Andrew & CC024 & Coordenador de Cultura\\
        \hline
    \end{tabular}
    \egroup
    \label{tab:my_label}
\end{table}