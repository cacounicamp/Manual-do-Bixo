\section{Um mundo de Entidades}

Todes na sociedade podem conformar entidades auto organizadas para representar seus interesses e realizar atividades desejadas por todes. São as entidades representativas.

As primeiras entidades assim eram as Sociedades de Socorro Mútuo, auto organizadas para gerir doações de trabalhadores para que os associados pudessem se aposentar e ter uma velhice digna ou se afastarem quando sofriam alguma doença, quando a aposentadoria e licença médica ainda não eram direitos trabalhistas (pois é!). Essas associações cresceram e passaram a representar interesses mais amplos dos trabalhadores.

A partir dessa experiência, es estudantes passaram também a se organizar em entidades que defendem seus interesses. Grêmios estudantis se formaram a partir daí nos colégios, e as universidades têm suas próprias entidades representativas também. O CACo é uma delas!

Todas elas são sempre auto organizadas e com gestões (equipes de estudantes) eleitas, que concorrem às eleições todos os anos. Abaixo, listamos as entidades representativas que você vai ouvir falar por aí. Elas estão organizadas por ordem de abrangência, de menor para maior. Não há uma hierarquia além da abrangência, ou seja, cada uma atua (ou deve atuar) com certo nível de independência que represente os interesses de seus próprios estudantes, não de outra entidade maior.

\begin{itemize}
    \item \textbf{UNE:} Entidade representativa de todos os estudantes de graduação do país. A pós-graduação tem sua própria, a ANPG, e a galera até o Ensino Médio, a UBES. As eleições de gestão são indiretas, por delegados votados nas universidades, a cada 2 anos. Bastante criticada por não aparecer muito aqui em baixo, nas universidades, onde tudo acontece.
    
    \item \textbf{DCE Unicamp:} Diretório Central dos Estudantes. Entidade que deve representar todos os estudantes de graduação na universidade.
    
    \item \textbf{CA - Centro Acadêmico:} (Alou!) Representa estudantes de determinados cursos ou institutos.
\end{itemize}

Não necessariamente vinculadas às entidades citadas, temos outras entidades igualmente auto organizadas e que podem se unir entre si – como as Ligas de Atléticas e Federações de Empresas Juniores. Não necessariamente representativas, estão as entidades e coletivos temáticos e de projetos.

As entidades temáticas que atuam na comp estão listadas nas próximas páginas. Visite as reuniões, veja qual te interessa mais e tenha uma graduação muito além da graduação! <3