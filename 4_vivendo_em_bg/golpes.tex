\section{Cuidado com os golpes}

Todo ano, milhares de estudantes entram em Barão Geraldo; geralmente, inexperientes com a vida adulta e que nunca tendo assinado contratos de nada. Essa situação é propícia para picaretas que queiram aplicar golpes. Claro, nem todo mundo numa república de 10 pessoas vai assinar o contrato de locação, mas se você estiver nessa situação, fique de olho e evite dor de cabeça! Abaixo, damos alguns toques pra você fazer antes de assinar qualquer coisa.

\begin{itemize}
    \item \textbf{Não pague ou assine nada antecipadamente.} Nem cheque, promissória... nada. As "taxas de reservas” e "taxas de contrato” falsas (e caras) e sua assinatura é sua alma.
    
    \item \textbf{Busque assistência jurídica.} O Serviço de Apoio ao Estudante (SAE) oferece apoio jurídico gratuito. Não deixe de conferir com eles os documentos da sua locação. Adultos da sua confiança e veteranes mais experientes, que já lidaram com contratos de aluguel, podem te ajudar - mas nada substitui profissionais.
    
    \item \textbf{Verifique se o corretor é real.} Corretores de imóveis têm um registro profissional da categoria chamado CRECI. Você sempre pode solicitar o número e conferir no site do CRECI se está ativo.
    
    \item \textbf{Verifique a propriedade do imóvel.} Já houveram casos de até a imobiliária "se enganar” e alugar casas sem o conhecimento do dono real do imóvel, devido a declarações falsas de propriedade e coisas desse tipo. Por isso, certifique-se de ler a original do certificado de propriedade do imóvel para você ver a autenticidade do documento e, se possível, vá ao cartório onde está registrada a casa para conferir as informações. São documentos que toda imobiliária deve (ou deveria) fornecer acesso aos seus clientes.
    
    \item \textbf{Conheça os valores}. Existem taxas – condomínio, IPTU, seguros obrigatórios. Ao alugar uma casa, informe-se em profundidade sobre quando e quais dessas taxas você deverá pagar pra saber o custo real de morar ali.
\end{itemize}
