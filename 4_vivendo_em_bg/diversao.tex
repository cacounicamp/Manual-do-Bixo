\section{Diversão}

\subsection{Divirta-se com responsabilidade e segurança}

Quanto a diversão e entretenimento na cidade, infelizmente as coisas estão meio irregulares por causa dos efeitos da pandemia; mas normalmente há muito o que fazer, e você não precisa ir tão longe! A Unicamp é um polo de produções e estudos culturais com uma agenda extensa. O site ou o Instagram de cada local costumam informar o que tá rolando. Além disso, Barão Geraldo tem barzinhos, festinhas e eventos em praças. O cinema mais próximo e mais conveniente é o do Shopping Parque Dom Pedro.

\subsection{Sebos e livrarias}

Em Barão há três sebos: O Curupira, o Cronópio e o Galpão. Geralmente você não encontra muita coisa boa de computação neles, mas tem bastannnte coisa legal. O Curupira fica na rua do Terminal, bem em frente a ele; não é difícil achar. O Galpão fica perto do Terminal; saindo do terminal pela avenida marginal à Albino de Oliveira, vire a primeira à esquerda e a segunda à direita. Funciona de segunda à sexta até às 18h. O Cronópio fica na mesma rua do Galpão, só que bem longe, próximo à padaria Fiori; seguindo a Santa Isabel, vindo do Centro de Barão, vire à esquerda na esquina que tem uma pizzaria (antes da Fiori); vire na primeira à esquerda e você está no Cronópio.

Na Unicamp, há a livraria Toledo que fica na Faculdade de Educação (Pedago). Também há a livraria da Unicamp (no IEL), tem preços bons mas não tem quase nada de exatas. Já a Livraria da Química tem livros de exatas, e geralmente eles conseguem importar a um preço bom.

Mais uma dica: antes de comprar um livro acadêmico, veja se você vai usar muito ele, se não tem bastante nas bibliotecas, se alguém não pode te emprestar, te vender ou tirar xerox. Na Unicamp não há muita necessidade em comprar livros, mas algumas pessoas preferem ou querem. Caso vá comprar, atente para livrarias virtuais, que podem ter preços muito menores. Em alguns casos, \textbf{a livraria da Editora} (localizada no piso térreo da BC) pode trazê-lo por um preço menor ainda.

\subsection{Exercícios}
\todo{Adicionar textinho sobre preço das academias}
Existem várias academias no centro de Barão e algumas perto da Unicamp; mas, se você prefere outros tipos de atividades físicas, os cursos de extensão (como pilates, natação, judô, tênis etc.) da FEF são oferecidos a cada semestre e com um preço mais em conta do que encontrado em outros lugares. Fique atente às datas e horários de inscrições! Alguns cursos ficam sem vagas em minutos! Página dos cursos de extensão: \textbf{www.fef.unicamp.br/fef/extensao/atividadesfisicas}.

\begin{figure}
    \centering
    \includegraphics[width=\linewidth]{placeholder.jpg}
    \label{fig:Locais de Lazer}
\end{figure}