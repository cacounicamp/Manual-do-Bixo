\section{Lugares para morar}

Separado do resto da cidade pela rodovia Dom Pedro I, Barão Geraldo é um distrito de Campinas onde fica a Unicamp e a Moradia Estudantil. Como, pra quem não tem carro, praticamente só se consegue acessar as outras partes da cidade através dos ônibus que saem do Terminal Barão, a vida universitária acaba se concentrando na região e muites estudantes acabam se tornando habitantes exclusivos de Barão Geraldo. Digamos que o mercado imobiliário e os mercados locais em geral se aproveitam um pouco da situação para inflar um tanto os seus preços. Por isso é bem importante ficar ligade nas dicas dessa seção que vão te ajudar a sobreviver por aqui!

Para aquelus que são de Campinas ou de outras cidades da região, morar com a família ainda é uma possibilidade. Mas talvez isso não proporcione a mesma experiência de estar a alguns passos do campus e, certamente, não é uma opção para todes, levando em conta que a Unicamp recebe ingressantes de todo o Brasil. Então, vamos te contar algumas alternativas para morar perto do campus, além do IC3, é claro – nosso instituto funciona 24h por dia e é bem possível passar a noite por lá. 

O custo de moradia em Barão Geraldo depende principalmente de três fatores: proximidade da Unicamp, tamanho do imóvel e qualidade da casa (acabamento, número de banheiros, presença de piscina etc.). Quanto à distância, os entornos \textbf{Avenida 1} (Avenida Doutor Romeu Tórtima) e da \textbf{Avenida 2} (Avenida Professor Atílio Martini) costumam ser mais caros de se morar, por serem próximos da Unicamp. A região que vai do centro de Barão Geraldo até a Moradia Estudantil, um pouco mais distante da universidade (cerca de 10 minutos de bicicleta), é em geral mais barata e concentra muito mais serviços, como supermercados, restaurantes e bancos.

Uma boa dica para se informar a respeito de lugares para morar (repúblicas, quitinetes, pensionatos) é o site \textbf{morandoembarao.com}. Lá, você encontra informações como endereço, preço, contato e detalhamento do lugar, com boa visualização da distância, além de poder filtrar a pesquisa com o tipo de vaga buscada. 

E vale lembrar que você não terá aulas na FEEC antes do segundo ano (se for engenheire e seguir a proposta de cumprimento de currículo) e poucas aulas serão no IC; a maioria delas serão ministradas no PB e no CB. 

Então não se preocupe procurar onde morar perto da FEEC ou do IC. Se você é da CC ou é da EC mas optou por pegar matérias à noite (sim, você pode fazer isso!), lembramos também que Barão não é um lugar tão seguro para ficar circulando a pé, principalmente sozinhe e no escuro. Consulte a seção Mantenha-se em segurança, na página \pageref{seguranca}.

\begin{figure}[htbp!]
    \centering
    \includegraphics[width=\linewidth]{placeholder.jpg}
    \label{fig:enter-label}
\end{figure}

\subsection{Moradia estudantil gratuita}

Trata-se de um programa de moradia destinado aos alunos com dificuldades em manter residência/moradia com recursos próprios, especialmente aqueles que residem fora da Região Metropolitana de Campinas (RMC) ou em grandes distâncias do campus.

A moradia - carinhosamente chamada de Moras - tem uma história incrível e é fruto de uma grande batalha dos estudantes. Saiba mais na seção Permanência Estudantil.

Cada casa da Moradia, dividida por estudantes, constitui-se de um quarto, uma cozinha, um banheiro e uma sala. Há ainda o Ônibus da Moradia, um circular da Unicamp que transporta pessoas durante o dia todo da Moradia até a Unicamp e vice-versa.

A Moradia está localizada na Avenida Santa Isabel, 1125 - a cerca de 3km do campus. E as inscrições para a bolsa moradia e outras bolsas abrem no mês de março para calouros. Não perca datas! Para saber mais sobre a Moradia e o processo seletivo, entre no site www.prg.unicamp.br/moradia/.

\subsection{Repúblicas}

Quando se fala numa república estudantil, talvez a primeira imagem que vem à cabeça seja de festas, bagunça e diversão. Mas não é só isso: as repúblicas oferecem grandes oportunidades de convivência e integração na vida universitária, assim como crescimento pessoal e profissional. A conexão com pessoas de outros cursos também é muito facilitada, com trocas de cultura e experiências, que dão a chance de fazer e consolidar amizades que podem durar por toda a vida. Junto do “baixo” valor de aluguel, são esses os motivos que tornam as reps a opção mais atrativa para estudantes.

Infelizmente, há uma história pouco documentada da especulação imobiliária em Barão Geraldo. Os valores dos aluguéis têm aumentado bastante ao longo de décadas, transformando a vida nas repúblicas. Imóveis que foram feitos originalmente para 5 ou 7 pessoas passaram por reformas e ampliações - às vezes com alguns problemas - para receber mais pessoas; quartos individuais se tornaram duplos e, no fim, até 15 pessoas vivem na mesma casa. É interessante observar como a convivência e o padrão dos imóveis mudam para dar conta de compartilhar entre mais pessoas os custos cada vez maiores de viver em Barão Geraldo. O custo de uma vaga em república é bem variável, dependendo principalmente do número de pessoas com quem dividirá quarto (ou se é um quarto individual) e da localização.

Aqui em Barão, temos a ARU, Associação de Repúblicas da Unicamp. Ela é a entidade que representa as Repúblicas Associadas da Unicamp e você pode saber mais sobre ela na seção Além da Graduação!

\subsection{Quitinetes}

Cuidado! A especulação imobiliária em Barão Geraldo chega a ser imbecil. As quitinetes mobiliadas são normalmente compostas de um banheiro e um cômodo que é quarto, sala, cozinha e área de serviço, tudo ao mesmo tempo. Os valores de aluguel ficam, em média, acima dos mil reais. Sim, mil reais por um espaço pequenininho. Só porque é perto da Unicamp. Fique de olho, tome cuidado com os contratos!

As melhores relações custo-benefício de quitinetes são as daquelas próximas ao centro de Barão Geraldo ou no espaço entre as avenidas. E lembre-se de que não adianta pagar mais caro para estar do lado da universidade se você fica muito longe dos supermercados, farmácias, padarias, restaurantes, etc.

\subsection{Pensionatos}

Pensionatos são como repúblicas, mas podem vir com regras. Dependendo do pensionato que você conseguir, pode tornar-se uma grande roubada. Alguns não deixam você levar pessoas para casa, reclamam se você chegar tarde e não liberam festas; outros, não, então procure bem.

O preço varia de acordo com as comodidades disponíveis. Pode ser uma opção muito cômoda se você procura um conjunto de casa, comida e roupa lavada.