\section{Mensagem da FEEC}

Prezados ingressantes do curso de Engenharia de Computação,

Parabéns pela conquista! É com imensa alegria que os recebemos na Unicamp e, em especial, na Faculdade de Engenharia Elétrica e de Computação (FEEC).

A vida universitária é uma fase singular e transformadora: poucos anos, intensos e breves, que moldam não apenas nossas escolhas profissionais, mas também nosso modo de pensar e enxergar o mundo. Nesta carta de boas-vindas, desejo compartilhar com vocês alguns pontos fundamentais: apresentar um pouco da FEEC e sua trajetória, destacar a excelência do curso que agora iniciam e refletir sobre a responsabilidade que temos, como membros de uma instituição pública de referência, perante a sociedade.

A FEEC iniciou oficialmente suas atividades acadêmicas em 1967, com a primeira turma de Engenharia Elétrica da Unicamp. Desde então, cresceu em recursos, prestígio e reconhecimento, consolidando-se como uma instituição líder no ensino e na pesquisa. Contamos com um corpo docente altamente qualificado, que equilibra a experiência dos professores que ajudaram a construir nossa história com o dinamismo de jovens pesquisadores recém-contratados. Nossa infraestrutura, embora em constante aprimoramento, proporciona condições adequadas para o aprendizado teórico e prático, contando ainda com o apoio de uma equipe técnica e administrativa comprometida com a qualidade do ensino.

O curso de Engenharia de Computação teve início em 1990, refletindo a expertise já estabelecida na área e a crescente demanda por profissionais capacitados para atuar no setor de tecnologia. Desde sua criação, manteve-se fiel às exigências da engenharia e à busca pela excelência acadêmica. Esse curso é compartilhado com o Instituto de Computação (IC), uma unidade de ensino e pesquisa de grande prestígio, onde vocês também encontrarão um corpo docente altamente qualificado. Assim como outras engenharias, a base do curso é sólida em matemática e física, exigindo dedicação especial nos primeiros semestres. É natural enfrentar desafios ao longo da jornada, mas é essencial manter o entusiasmo e a confiança. Para isso, vocês contarão com professores, coordenadores e a direção da FEEC e do IC, sempre dispostos a oferecer suporte. Na FEEC, há ainda o “Espaço de Acolhimento” (EA-FEEC), um recurso valioso para apoio acadêmico e pessoal.

A Engenharia de Computação é um curso desafiador, mas intelectualmente enriquecedor e bem estruturado. Invistam seu esforço, busquem ajuda quando necessário e aproveitem essa experiência de forma plena. Além dos estudos, a universidade é um ambiente de cultura, debates e crescimento pessoal. Aqui, vocês terão oportunidades únicas de aprendizado, networking e desenvolvimento de amizades duradouras.

Participem ativamente da vida universitária, das entidades estudantis e das diversas iniciativas que a Unicamp oferece. Mas lembrem-se de manter o foco na formação acadêmica, sem deixar de lado os ideais e objetivos que os trouxeram até aqui. Mais do que uma conquista pessoal, estudar em uma universidade pública de excelência carrega um compromisso com a sociedade. O ensino gratuito e de qualidade que recebemos é viabilizado pelo trabalho e pelo esforço de milhões de brasileiros. Por isso, é nossa responsabilidade retribuir esse investimento com dedicação, ética e compromisso em utilizar nosso conhecimento para o progresso social, científico e tecnológico do país.

Encerrando, compartilho uma citação de Alan Turing, um dos pioneiros da computação, que afirmou: "Às vezes, é a pessoa de quem ninguém espera nada que faz coisas que ninguém pode imaginar." Essa frase nos lembra que o potencial de cada um vai além do que os outros – e até nós mesmos – podemos prever. Hoje, vocês iniciam uma jornada repleta de desafios, descobertas e conquistas. Suas realizações serão motivo de orgulho para suas famílias e, agora, também para nós. Que essa caminhada seja marcada pela curiosidade, pela inovação e pelo desejo constante de aprender. Sejam bem-vindos, sejam bem-vindas à FEEC e aproveitem ao máximo essa experiência transformadora!

Hugo Enrique Hernandez Figueroa - Diretor da FEEC