\section{Mensagem do IC}

O Instituto de Computação da Unicamp (IC) tem suas origens em 1969, com a criação do primeiro curso de Bacharelado em Ciência da Computação do Brasil. Este programa pioneiro serviu de modelo para diversos outros cursos no país, consolidando a liderança da Unicamp na área.

Reconhecido como referência acadêmica por sua excelência no ensino, pesquisa e extensão, o IC já formou milhares de profissionais em seus programas de graduação, pós-graduação e extensão, contribuindo significativamente para o avanço científico e tecnológico do Brasil.

A produção científica de ponta do Instituto reflete-se em inúmeras contribuições para a sociedade. Essas incluem publicações de alto impacto, colaborações com instituições públicas e privadas, parcerias com a indústria, registro de patentes e licenciamentos. Acima de tudo, destaca-se a formação de profissionais altamente qualificados, muitos dos quais ocupam posições de destaque em diversos setores.

O ambiente inovador do Instituto também é um terreno fértil para o empreendedorismo. Até hoje, mais de 200 empresas foram fundadas por alunos e ex-alunos do IC, abrangendo desde startups até empresas globais, incluindo as chamadas empresas-unicórnio. Esses empreendedores não apenas alcançam sucesso no mercado, mas também contribuem diretamente para a formação de novas gerações, fomentando uma cultura sólida de inovação e empreendedorismo.

Além de suas atividades acadêmicas e de pesquisa, o Instituto também desenvolve programas sociais e promove atividades extracurriculares que buscam ampliar o impacto positivo na comunidade, fortalecendo o compromisso com a democratização do conhecimento e a formação de cidadãos conscientes e engajados.

Atualmente, o Instituto conta com quase 50 professores que atuam em diversas áreas da Ciência da Computação. Possui 14 laboratórios de pesquisa dedicados a projetos inovadores e disruptivos, realizados em colaboração com a indústria e outras instituições. Esses projetos somam mais de R\$ 50 milhões em investimentos, cujos resultados geram novos conhecimentos, tecnologias aplicadas ao mercado, publicações de impacto, formação de recursos humanos e reconhecimento internacional para o IC.