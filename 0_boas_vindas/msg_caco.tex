\section{Os melhores anos da sua vida!}

Parabéns!

Entrar na Universidade e cursar o ensino superior é uma grande conquista e pode lhe trazer algumas das experiências mais interessantes da sua vida. Uma delas é a oportunidade de se especializar em uma área do conhecimento e aprender com cientistas e profissionais que sabem muito bem - ou até mesmo descobriram - os assuntos que estão ensinando.

Nunca é tarde demais para estudar, mas muites de nós, quando temos essa oportunidade, a agarramos ainda jovens. E assim é comum que muitas coisas se misturem aos desafios da vida acadêmica: a entrada no mundo adulto, relações marcantes entre colegas e amigues, dezenas de convites para festas por mês, responsabilidades crescentes e as famosas contas pra pagar.

Enfim, o peso das novas responsabilidades e dos estudos, assim como as deficiências e exigências da própria universidade podem impactar nossa saúde mental - um tema que está sendo cada vez mais discutido e que incluímos neste manual para você poder começar o curso mais tranquile.

Mas muito mais do que estudar, aqui você vai encontrar toda uma comunidade realizando uma serie de atividades extras incríveis. Atividades para todos os gostos: de grupos de estudo a redes de apoio, times de esportes e e-sports, atléticas, empresas juniores e nós o seu centro acadêmico. Essas experiências elevam o significado dos anos na universidade a muito mais do que aulas, provas, livros e exercícios. E, através delas, você pode encontrar amizades pra toda a vida!

Entrar numa universidade internacionalmente reconhecida, como a Unicamp, tem ainda mais desdobramentos. Suas atividades cotidianas poderão ter um impacto tremendamente grande lá fora!

Isso se deve, em grande parte, ao fato de que nossa universidade é pública. Isso significa que você é ume des pouques que a população brasileira conseguiu colocar pra dentro de um ensino superior de qualidade que não apenas ensina uma nova profissão, mas que também produz ciência. Não por acaso, a poucos quilômetros do campus da Unicamp fica o Sirius, um acelerador de partículas de padrão mundial! Por esses motivos, pelo nosso trabalho científico e de nossos colegas, podemos nos orgulhar do retorno que damos à sociedade, fazendo o peso do nome Unicamp valer muito mais do que um diploma na parede.

Infelizmente, no contexto mundial, a realidade é que a universidade pública precisa que você, além de adquirir e criar conhecimento, a defenda como parte de sua jornada por aqui. A ciência e a educação são grandes patrimônios da humanidade e somos nós os responsáveis por zelar por elas, defendendo-as contra os crescentes ataques que tentam desmerecer e distorcer nossa posição na sociedade. Não desperdice a sua chance de cumprir esse papel!

Por tudo isso, bem-vinde! Os melhores anos da sua vida serão o que você fizer que eles sejam. Você tem um longo caminho a trilhar agora e vai se deparar com muitos desafios. Então, não se esqueça: você sempre pode contar com a gente.