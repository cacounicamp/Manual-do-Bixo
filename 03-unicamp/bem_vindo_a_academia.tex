% Este arquivo .tex será incluído no arquivo .tex principal. Não é preciso
% declarar nenhum cabeçalho

\section{Bem-vindo à academia}

Academia é o nome que se dá à comunidade internacional de pesquisadores e
estudantes de ensino superior, atuando em todas as áreas do conhecimento.
Geralmente centrada ao redor de universidades, porém organizações públicas e
privadas e também empresas fazem parte. A Academia é dividida principalmente
entre os pilares de Pesquisa, Ensino e Extensão, sendo os dois primeiros mais
acessíveis para nós da graduação.

A Unicamp está muito bem situada no cenário acadêmico, sendo a responsável por
15\% da produção científica brasileira, e tendo inúmeros pesquisadores de
renome internacional, então não vão faltar oportunidades para você entrar nesse
mundo.

\subsection{Iniciação Científica}

A iniciação científica é um tempo para alunas e alunos de graduação (você, no
caso) ter uma experiência acadêmica mais séria, sentir um pouco como é o clima
de pesquisa. Interessou? O que fazer? Calma, você mal entrou na Universidade.
Geralmente, o que se faz é conversar com professoras e professores da área com
a qual você se identifica mais (criptografia, teoria da computação,
processamento de imagens, inteligência artificial, física, química etc.) e ver
se está desenvolvendo algum projeto interessante naquela área, ou propor alguma
ideia sua mesmo. Depois você começa a estudar para redigir um projeto e
encaminhá-lo para alguma instituição de fomento à pesquisa (CNPq ou FAPESP),
pedindo uma bolsa de iniciação científica. A FAPESP paga em torno de R\$ 640,00
e aceita pedidos de bolsa em qualquer período do ano. O CNPq paga
aproximadamente R\$ 400,00, e o período para inscrição de projetos é geralmente
em junho e novembro. No primeiro semestre geralmente é bem mais difícil achar
alguma professora ou professor da área que você se interessa, aliás, é bem
difícil saber a área com a qual você se identifica, pois você mal começou o
curso e não conhece muito do que se estuda em computação, muito menos os
professores. Mas tenha paciência, agora parece tudo muito complicado, mas com o
tempo as coisas vão ficando mais simples. Se você realmente tiver uma sede
insaciável de conhecer o meio da pesquisa, procure a professora ou professor
que te deu aula de MC102, ele pode te orientar a respeito.

\begin{figure}[h!]
  \centering
  \includegraphics[width=.45\textwidth]{img/unicamp/congresso.jpg}
  \caption*{Congresso de Iniciação Científica da Unicamp}
\end{figure}

Outra coisa interessante a respeito da iniciação científica é que, se você
conseguir bolsa, pode pegar a disciplina MC040 e posteriormente MC041 (2
semestres), cada uma com 12 créditos. São 24 créditos praticamente ``de
bandeja'' para ajudá-la a recuperar o CR, caso esteja no fundo do poço. Note
bem as aspas. Os trabalhos de iniciação científica geralmente consomem muito
tempo de estudo e dedicação, não vá pensando que é moleza, não.

Na FEEC, você pode conseguir as matérias de iniciação (EE015 e EE016) mesmo
sem bolsa, mas você ainda assim vai precisar de um orientador. Lá a iniciação
científica também substitui o estágio, mas não tem equivalência com a do IC.

Note que sua iniciação científica não precisa estar vinculada a computação.
Como falaremos ainda neste capítulo, a Unicamp permite que você faça
disciplinas de qualquer instituto com créditos eletivos, isso pode te estimular
a fazer matérias de áreas que gosta fora da computação e nada te impede de
fazer alguma iniciação científica nisso, aproveite!

\subsection{Monitoria}

Além da iniciação científica, monitoria é uma forma muito comum de participar
da Academia, mais focado no pilar de ensino. Não vai demorar muito para você
descobrir a importância da monitoria para as matérias. Logo no primeiro
semestre, boa parte das disciplinas possuem monitoras ou monitores, que são
dividos em dois tipos: PED (Programa de Estágio Docente) e PAD (Programa de
Apoio Didático).

\textbf{PED} é alguém da pós-graduação que é responsável por ajudar a
professora ou professor na disciplina, normalmente, criando materiais,
exercícios, laboratórios e ajudando na correção dos mesmos. Já \textbf{PAD} é
alguém da graduação e geralmente só é responsável por ajudar nos laboratórios e
tirar dúvidas, não deve incluir correção de trabalhos.

Por enquanto, o mais importante pra quem quer participar é o PAD. Todo mundo
pode se inscrever, a forma e momento das inscrições varia de acordo com o
instituto, mas costuma ser no final do semestre. No IC e FEEC, recebemos um
e-mail da Secretaria de Graduação com um formulário para preencher.

O programa de PAD costuma incluir uma bolsa que fica em torno de R\$ 520,00.
Os pré-requisitos pra conseguir são (1) ter cursado a disciplina ou alguma
equivalente, (2) não ter reprovado na disciplina e (3) ter disponibilidade para
participar das atividades, que para o nosso curso costuma ser estar livre
durante os laboratórios, (4) ter o CR acima da média da sua turma ou ter a
maior nota na matéria dentre os candidatos.

Algumas disciplinas são mais concorridas que outras para monitoria, como é o
caso das primeiras matérias da computação. Algo interessante é que, assim como
iniciação, monitoria também tem duas disciplinas, MC050 e MC051 -- ambas de 8
créditos. Mesmo sem bolsa, é possível exercer monitoria, conseguindo os
créditos.

Mas vá com calma, como é preciso ter cursado a disciplina e ter um CR para se
candidatar pra monitoria, só vai ser possível se candidatar no final do seu
segundo semestre, já que as inscrições acontecem no fim do semestre (geralmente
próximo da matrícula). É bom se atentar que monitoria exige um certo tempo de
dedicação, em especial porque você estará ajudando outras alunas e alunos na
disciplina. É uma grande responsabilidade e uma ótima forma de ter contato com
um dos lados da Academia que muitas vezes é esquecido: justamente o lado que
lhe trouxe para cá!

\subsection{Intercâmbio}

A Unicamp é uma das universidades brasileiras que têm maior prestígio fora do
país e a VRERI-Unicamp (Vice-Reitoria Executiva de Relações Internacionais,
antigo CORI), IC e FEEC têm vários acordos bilaterais de intercâmbio. Então,
para você que quer dar um salto em algum idioma, conhecer outras culturas e
sentir na pele a aventura de ser estrangeira, comece a se preparar desde já.

Muita gente costumava pegar intercâmbio pelo Ciência Sem Fronteiras, mas ele
foi congelado pelo Governo Federal em 2015, e passou por uma reformulação: não
haverá mais vagas para a graduação, apenas para a pós. Ainda existem boas
opções pra quem gostaria de pegar um intercâmbio, mas ficou muito mais difícil.

A França hoje recebe um número razoável de estudantes de computação, devido a
acordos que a Unicamp tem com os INSA e com as Écoles Centrales e também graças
a bolsas de estudo oferecidas pela Capes e pelo governo francês. Os
intercâmbios para a França são bem mais concorridos que o Ciência sem
Fronteiras, pois são na modalidade de duplo diploma (você se forma pela Unicamp
e pela instituição francesa), há um processo seletivo envolvendo uma
entrevista, e vai durar mais tempo, dois anos ou mais, numa instituição de
prestígio como a École Polytechnique, por exemplo.

No site da VRERI existem oportunidades para ir para Estados Unidos, Japão,
América Latina, Alemanha, Espanha etc., muitos com boas bolsas de estudo ou com
incentivos que valem a pena caso você tenha um pouco de grana para se sustentar
no início. Visite sempre o site da VRERI e participe das reuniões que ela faz,
pois ficar ligada é a chave para conseguir encontrar uma boa oportunidade.
Existem outras opções de intercâmbio, como a AIESEC, que promove um intercâmbio
para estágios no exterior. Se seu interesse é mais profissional, procure se
informar.

\begin{quote}
``Mas vale a pena? Poxa, vou atrasar meu curso, ficarei deslocada de turma, vou
ficar em um país estranho, para quê? Acho que não vale a pena{\dots}''
\end{quote}

Vamos começar pelos motivos profissionais: ter no currículo que você fala uma
língua estrangeira fluentemente devido à sua imersão no país é algo muito
valorizado pelas empresas, além do fato que o pessoal do RH vai ver que você
tem capacidade de se virar sozinha, uma vez que não é tão óbvio sair do país e
recomeçar sua vida fora. Você não atrasará tanto seu curso, pois a Unicamp
conta com um sistema de equivalências de matérias, e se você escolher bem pode
fazer matérias que serão convalidadas na Unicamp. Agora, o que realmente é
importante: você está na faculdade, está na hora de deixar o colo da mamãe e
partir para o mundo! A experiência de conhecer outras culturas, criar laços de
amizade internacionais, viajar por terras desconhecidas não tem preço! Pense
que é no seu tempo de facul que terá oportunidade de fazer uma aventura destas,
não desperdice.

Se você abriu um sorriso e pensa que está preparado para sair do país, comece a
estudar, bixete! Não que ter uma boa nota seja a única forma de conseguir uma
vaga em uma bolsa de estudos, mas com certeza é a mais fácil. Busque atirar em
todas as frentes, mantenha seu CR num bom nível, busque conhecer organismos
como AIESEC e procure grupos de trabalho (no IC existem vários) que podem levar
alunos ao exterior. Boa sorte!

Se você realmente se interessou, aí vão uns links com mais informações:

\begin{itemize}
\item VRERI:
  \begin{small}
  \url{www.internationaloffice.unicamp.br}
  \end{small}
\item AIESEC: \url{aiesec.org.br}
\end{itemize}

Fique atenta aos e-mails que você receberá do IC e da FEEC. Alguns deles são
sobre programas de intercâmbio.

\subsection{Acesso a artigos e revistas científicas}

Os resultados de pesquisas científicas, no Brasil e no mundo, costumam ser
publicados por meio de periódicos e conferências, os quais normalmente são
disponíveis pela internet.

No Brasil, quase todas as instituições públicas de ensino superior, como a
Unicamp, participam de um sistema conhecido como \textbf{Portal de Periódicos
da Capes} (\url{periodicos.capes.gov.br}), que garante acesso a grande parte
das publicações científicas das principais editoras do mundo sem necessidade de
pagar nada a mais por isso.

Nas áreas de engenharia e de computação, quase todas as publicações relevantes
são acessíveis através desse sistema. Mas é importante você saber que esse tipo
de acesso só é possível a partir de endereços IP da Universidade, então se você
quiser acessar algum artigo quando estiver em casa, o ideal é usar o sistema de
acesso VPN (Virtual Private Network) disponibilizado pela Unicamp, como pode
ser visto no site: \url{bit.ly/1xHdH8X}.

Através da \textbf{Comunidade Acadêmica Federada (CAFe)}, foi disponibilizado
recentemente um método de acesso remoto aos periódicos sem necessidade de usar
a VPN, mais informações aqui: \url{bit.ly/1w1twlz}

Na Unicamp, você ainda tem acesso a diversas outras publicações e e-books que
não são cobertos pelo sistema da Capes, além de alguns periódicos impressos,
que podem ser encontrados nas bibliotecas. Caso você queira buscar algo no
material que há disponível física ou virtualmente na Universidade, acesse o
site do \textbf{Sistema de Bibliotecas da Unicamp}: \url{www.sbu.unicamp.br}.

Para uma busca mais abrangente de artigos científicos na internet, você pode
usar o \textbf{Google Acadêmico} (\url{scholar.google.com}). Mas atenção! Você
pode encontrar artigos que não são cobertos pelo Portal da Capes nem pela
Unicamp e exigem pagamento.

Além do Portal de Periódicos, existe também um novo modelo de publicações
científicas de acesso gratuito, chamado \textbf{open access}. Esse modelo tem
origem muito próxima do movimento pelo software livre. Publicações feitas nesse
sistema são acessíveis a qualquer momento, de qualquer IP e sem qualquer custo.
Alguns exemplos de grandes repositórios e editoras open access são:

\begin{itemize}
\item \textbf{SciELO:} \url{www.scielo.org}
\item \textbf{PLOS:} \url{plos.org}
\item \textbf{arXiv:} \url{arxiv.org}
\item \textbf{PMC:} \url{ncbi.nlm.nih.gov/pmc}
\end{itemize}

A rede \textbf{SciELO} é onde a maior parte dos artigos em português é
publicada. O acervo \textbf{PMC} é de publicações da área biomédica.

Bixete, guarde bem esta seção do manual! Pode ser que você não vá usá-la logo
de cara, mas quando você fizer iniciação científica ou um trabalho de
disciplinas mais avançadas, você aproveitará bastante essas informações.
