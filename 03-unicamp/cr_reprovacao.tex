% Este arquivo .tex será incluído no arquivo .tex principal. Não é preciso
% declarar nenhum cabeçalho

\section{Cuidado com CR e Reprovação}

Durante o curso você vai ouvir que se preocupar com seu CR é bobagem, que
estudar para tirar nota não leva a lugar nenhum, que depois de formada não é
seu CR que te colocará no mercado de trabalho etc. Cuidado, trabalhar numa
empresa não é a sua única opção de vida após formada, e preste atenção, pois
``após formada'' não significa durante o curso.

Durante o curso você vai ter a possibilidade de participar de várias atividades
acadêmicas e algumas delas vão exigir bom aproveitamento acadêmico da aluna ou
aluno. Por exemplo, para se candidatar a monitor de uma disciplina você precisa
ter CR acima do da média da turma. Para pleitear uma bolsa de iniciação
científica, onde há concorrência entre alunas e alunos do país todo, também
será exigido bom aproveitamento, assim como para uma bolsa de mestrado. Caso
você não saiba, mestrado faz parte da pós-graduação, ou seja, o seu CR vai te
influenciar até após formada.

Cuidado também com a reprovação. Há instituições, como a FAPESP (Fundação de
Amparo à Pesquisa do Estado de São Paulo), que é a maior fomentadora de
pesquisas do estado de São Paulo e que paga os maiores valores de bolsas do
país, que te excluem de qualquer disputa só por ter uma reprovação no seu
histórico escolar da graduação. Não te exclui oficialmente, mas como é muito
concorrido por ser a melhor pagadora, seu nome vai para o final da lista.

Parece óbvio que quem estuda tira boas notas, mas até você aprender a estudar
como a universidade exige, pode demorar um pouco, e há pessoas que nunca
aprendem. Ah, e cuidado com o estudo exagerado, é um curso de 4 a 8 anos, não
dá para manter o rítmo de estudo para vestibular -- caso você estudou
fortemente no cursinho ou ensino médio -- por todo esse tempo.

Há certos períodos (semestres), quando você já estiver mais avançado no curso,
em que poderá sentir-se à vontade para desistir de uma disciplina em que esteja
matriculado, deixando para completá-la posteriormente. Quando você fizer isso
há a possibilidade de desistir da disciplina, desmatricu\-lan\-do\--se
oficialmente. Mas há pessoas que simplesmente deixam de cursar a disciplina,
reprovando por nota e falta e ficando com uma nota baixa em seu histórico.
Cuidado com isso, pode ser frustrante para você no futuro. Por isso, se for
desistir de cursar uma disciplina após matriculado, sempre peça a desistência e
tente não reprovar.

Lembre-se de que o período de graduação é muito grande, você pode mudar de
ideia a qualquer momento sobre o que pretende fazer no futuro.
