% Este arquivo .tex será incluído no arquivo .tex principal. Não é preciso
% declarar nenhum cabeçalho

\section{Para que eu estou estudando isso?}

O ensino médio acabou, você finalmente está livre de todas as ``inutilidades'',
como química orgânica e separação silábica de verbos parnasianos, só vai ver
coisas relevantes para a profissão, e{\dots}

Pimba! HZ291. Pode, Arnaldo?

Primeiro, você precisa saber que a Universidade não é um curso técnico. A ideia
não é só te dar capacitação profissional, mas sim formar pessoas melhores. Para
que uma computeira ou um computeiro precisa de contabilidade? Para nada, mas
uma pessoa precisa ter uma noção disso, sobretudo de exatas.

Outro problema: o que exatamente é ``relevante para a sua profissão''?  A
computação é uma área muito vasta, e a graduação é muito generalista, para te
dar base para escolher. Por exemplo, vai ter gente que nunca mais vai usar
GA/Algelin -- geometria analítica e álgebra linear --, mas quem for para a área
de computação gráfica vai comer matriz no café da manhã. Quem garante que, no
meio do curso, você não decida ir para essa área? Ou ainda, que no seu emprego
não te joguem um problema desse tipo?

Se você continuar na universidade, na pós-graduação, você só terá matérias da
sua área, já que você já sabe o suficiente pra dizer que área é essa. Mas ainda
falta muito chão até lá{\dots}

Para quem é da engenharia, para conseguir o CREA, existem algumas matérias
obrigatórias, como resistência dos materiais. A Unicamp pode até contrariar
essas orientações, até certo ponto -- e ela o faz: as coordenadoras e
coordenadores da engenharia tem lutado para diminuir créditos obrigatórios e
aumentando eletivos --, mas há matérias em que as professoras e professores
dificilmente concordariam em alterar. (Por outro lado, você poderá construir
prédios de até 2 andares. Recomendamos fortemente que você não faça isso.)

Tanto para a ciência quanto para a engenharia, o curso não é para formar
simples programadores. Vocês serão mais que isso, serão cientistas, engenheiras
e engenheiros, e isso envolve ver coisas além de computação.
