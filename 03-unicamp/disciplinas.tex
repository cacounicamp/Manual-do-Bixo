% Este arquivo .tex será incluído no arquivo .tex principal. Não é preciso
% declarar nenhum cabeçalho

\section{Disciplinas}

\subsection{Matrícula}

A Unicamp é muito diferente da sua escolinha onde a tia Gertrudes entregava o
seu horário impresso coloridinho para você colar na capa do seu fichário. À
exceção do primeiro semestre letivo, no qual você já entra matriculada em todas
as matérias obrigatórias, na Unicamp você vai ter que se virar. O GDE
(\url{gde.ir}) é uma ferramenta criada por um aluno da engenharia e adotada
pela DAC que facilita muito o planejamento do seu horário, além de servir como
uma rede social interna, permitindo avaliação de oferecimentos de matérias
pelas professoras e professores, que serve de feedback às outras pessoas.

Peça sempre ajuda a uma veterana ou veterano quando for montar seu horário.
Informe-se sobre todas as professoras e professores que oferecem as matérias,
se são coxas ou pegam no pé, se dão aula bem ou mal, se demoram para entregar
as notas{\dots} Você vai poupar muita dor de cabeça. O melhor lugar para essas
discussões são os grupos de e-mail da turma, grupos de Facebook, grupo com
veteranas e veteranos nos mensageiros etc. Pode ter certeza que haverá grupos
assim que você entrar na Unicamp.

Nem sempre você vai conseguir exatamente o que quer na matrícula. Nesses casos,
pode tentar de novo no período de alteração de matrícula, que acontece,
normalmente, próximo ao inícios das aulas. Nesse período é possível pegar uma
matéria que não conseguiu na matrícula normal, permutar a turma de uma matéria
que já se matriculou ou até mesmo desistir de alguma delas, sem nenhum tipo de
penalidade.

\subsection{Calendário da DAC}

Para não ser pega pelo arrependimento de ir a alguma aula em dia de folga ou de
não poder mais trancar uma matéria que considera já perdida, anote as datas
importantes todo início de semestre no seu calendário.

Você pode encontrar essas datas no calendário da DAC: o site foi remodelado em
2017 e agora possui um filtro que seleciona o que é de interesse das alunas e
alunos de graduação, da pós, das professoras e professores, das funcionárias e
funcionários, o que torna sua vida muito mais fácil.

Isso não serve apenas para saber quais são os feriados, serve também para
períodos de estudo para o exame -- assim como o período para a aplicação da
prova --, período para desistência de disciplinas, trancamento de matrícula,
férias e matrícula no próximo semestre. Fique atenta, ninguém gosta de chegar
na faculdade e descobrir que poderia ter dormido até mais tarde!

\subsection{Cancelamento, Trancamento e Desistência}

Embora praticamente todas as alunas e alunos da Unicamp usem esses três termos
indiscriminadamente, como se fossem sinônimos, para a DAC, esses três termos
têm significados bastante distintos. Aí vai o que cada termo significa:

\subsubsection{Desistência de matrícula em disciplinas} (capítulo III, seção V
de \url{bit.ly/2ArORDi}): Processo que é chamado pelas alunas e alunos de
``trancamento''. Você não mais cursa essa disciplina no semestre, tendo de
cursá-la em algum semestre posterior (se for obrigatória). Só é possível
desistir uma vez da disciplina e pode-se pedir desistência até que se tenha
passado metade do semestre.
\subsubsection{Cancelamento de matrícula} (capítulo III, seção VII de
\url{bit.ly/2ArORDi}): Processo em que você se desliga da Unicamp, por motivo
de jubilação, por ter faltado às duas primeiras semanas do ano de ingresso, por
ter sido reprovado em todas as disciplinas do primeiro ou do segundo semestre
de ingresso, por ter sido expulso, por ter sido aprovado em outra universidade
pública (não é permitido fazer mais do que um curso de universidade pública
simultaneamente), ou por vontade própria.
\subsubsection{Trancamento de matrícula} (capítulo III, seção VI de
\url{bit.ly/2ArORDi}): Processo em que você não cursa qualquer disciplina da
Unicamp durante o semestre. É possível fazer até dois trancamentos de
matrícula, em semestres seguidos ou não, não podendo trancar nenhum dos dois
semestres do ano de ingresso. Desistência de todas as disciplinas configura-se
como trancamento. O trancamento é pedido na DAC, e pode ser pedido até que se
tenha transcorrido 2/3 do semestre (veja o calendário da DAC). Para cada
trancamento, o prazo máximo de integralização é postergado, ou seja, o seu
tempo fora da Unicamp não contará para o progresso do curso -- isso pode ser
muito útil em situações críticas.

\subsection{Mudança de Catálogo}

Pra quem não sabe, ou seja, você mesmo, bixete, para cada ano existe um
catálogo correspondente com todas as disciplinas que os alunos devem fazer para
se formar. Entretanto, esse catálogo não é o mesmo desde o século passado. Como
a computação é uma área muito dinâmica, o curso não pode ficar congelado e as
disciplinas precisam refletir o que o mundo e o mercado atual demandam. Assim,
ao decorrer dos anos ocorrem várias pequenas mudanças, como tirar uma
disciplina, oferecer outra, aumentar créditos, diminuir créditos e assim por
diante, todas feitas com muita discussão entre professoras, professores, alunas
e alunos, sobretudo nas reuniões semestrais de avaliação de curso.

Avaliação de curso é uma reunião proposta pelas coordenadoras ou coorenadores
que visa discutir todos esses pontos. Participe, pois é a sua chance de tornar
o curso melhor falando diretamente (ou não) com as coordenadoras ou
coordenadores do curso! Se tiver vergonha, pode pedir ao seu centro acadêmico,
o CACo, para fazer isto por você.

Como essas mudanças ocorrem num catálogo específico de um ano e só valem a
partir desse, os alunos dos anos anteriores não acompanham essas mudanças e
muitas vezes ficam defasados. Para solucionar isso, é possível mudar de
catálogo. Como assim? Simples. Se você é um aluno do ano de 2015, você teria
que fazer a matéria Fenômenos de Transporte (EM524), abreviada por FETRANSP. Já
em 2017, o catálogo não pede mais essa disciplina como obrigatória. Então, os
alunos de 2015 podem mudar pro catálogo de 2017, não precisando mais fazer essa
disciplina.

Mas, existem alguns poréns nesse troca-troca: se você mudar de catálogo, pode
haver matérias a mais que você precise fazer para completar o curso. Há também
o risco de não conseguir equivalência de algumas matérias que são semelhantes
entre si nos catálogos. Além disso, novas matérias só começarão a ser
oferecidas quando as bixetes e bixos do ano de ingresso do catálogo chegarem
nessa disciplina no catálogo proposto pelas coordenadoras ou coordenadores do
curso. Para exemplificar, imagine uma matéria remodelada ou nova proposta para
o quarto semestre no catálogo 2019. Essa matéria será dada apenas quando as
bixetes e bixos de 2019 quando chegarem ao 4º semestre. Não é comum de
acontecer, pois não é provável a criação de novas matérias, mas cuidado com as
equivalências e matérias que foram remodeladas.

Salvo esses detalhes, a mudança para catálogos de anos a frente do seu é livre
e bem fácil de conseguir, é só preencher um formulário \url{bit.ly/1LHPdyG} e
entregar à DAC. Para mudar para catálogos de anos anteriores (não recomendado)
é um pouco mais difícil, mas não impossível. Basta consultar a DAC.

\subsection{Eletivas e Teste de Proficiência}

A Unicamp oferece a oportunidade de personalizar seu currículo de acordo com
seu interesse por meio das \textbf{disciplinas eletivas}. Ao contrário das
disciplinas obrigatórias, com as eletivas você pode escolher a matéria que vai
cursar. Alguns créditos podem ser cumpridos com qualquer disciplina oferecida
pela Universidade, outros estão restritos a um determinado conjunto. Para mais
detalhes, consulte seu catálogo em \url{bit.ly/2BRj6j4}.

Mas não se esqueça de que com um grande poder vem uma grande responsabilidade!
A Unicamp lhe dá liberdade para escolher o melhor jeito de se preparar para seu
futuro, e espera que você saiba o que fazer com essa liberdade. Você pode
socializar com outros cursos, aprender uma língua estrangeira, assistir a
seminários ou obter um certificado de estudos na FEEC ou no IC.

\textbf{Teste de proficiência} é uma prova que permite dispensa de cursar uma
disciplina (desde que você obtenha a nota mínima, é claro). Se você acha que
sabe o suficiente sobre eletromagnetismo, por exemplo, pode tentar a
proficiência de Física Geral III. Nem todas as disciplinas oferecem o teste, e
você só pode fazê-lo uma vez por disciplina -- e se você já se matriculou na
disciplina e não passou, não pode fazer. Além disso, fazer o teste de
proficiência também é obrigatório para se matricular nas disciplinas de língua
inglesa e japonesa, independentemente de conhecimento prévio na língua.

Fique ligada no calendário da DAC para não perder as datas de inscrição nos
testes de proficiência! As datas dos testes de línguas são sempre no começo do
ano, diferentes das demais, que são no fim de cada semestre.

Disciplinas eletivas e teste de proficiência estão relacionados porque muitas
pessoas, especialmente nos cursos de computação, fazem proficiência em
disciplinas de línguas, eliminando créditos de eletivas, em alguns casos para
evitar o jubilamento, outros para não ter que passar mais um semestre na
faculdade. Apesar de registrar a familiaridade da aluna ou aluno com uma outra
língua ou disciplina em sua integralização, essa prática não enriquece a
graduação de nenhum estudante que faça tal escolha. Além disso, muitas bixetes
e bixos arrependem-se de terem feito a prova e perdido preferência na hora de
pegar uma disciplina interessante, podendo até mesmo não conseguir se
matricular. Isso acontece porque, depois que seus créditos de eletivas se
esgotam, você começa a puxar matérias não-obrigatórias como
\textbf{extracurriculares} e tem prioridade menor na atribuição de vagas.

Converse com suas veteranas e veteranos para descobrir o melhor jeito de
usufruir dessa liberdade que poucas universidades oferecem! Dificilmente você
não encontrará algo com o qual se identifica ou que não ensine lições
interessantes.

Para mais informações sobre teste de proficiência, acesse:
\url{bit.ly/2BS5O6e}.

\subsection{CEL}

O CEL -- Centro de Ensino de Línguas -- é, como o nome já diz, o orgão
responsável por oferecer aulas de diferentes idiomas a alunas e alunos da
Unicamp. Seja como disciplinas obrigatórias (para a engenharia da computação,
Inglês Instrumental I é uma delas) ou como eletivas, o CEL possui turmas de
várias línguas.

Todas as línguas no CEL oferecem níveis diferentes; assim, se você souber um
pouco de Francês, não precisa começar do início. É só realizar o teste de
proficiência, oferecido no primeiro semestre de cada ano, e você pode avançar
algumas turmas (ou todas) em uma determinada língua. Também é possível fazer o
teste de proficiência pra eliminar disciplinas (o que a maioria das pessoas
acaba fazendo com Inglês Instrumental). \textbf{Todas as alunas e alunos} que
quiserem se inscrever em \textbf{Inglês I (LA112)} ou
\textbf{Japonês I (LA111)} devem \textbf{fazer o teste}, mesmo que não saibam
nada da língua, mas ele não é necessário para matrículas nas outras línguas.

A matrícula em disciplinas do CEL é feita de maneira natural, na DAC, durante o
período de matrícula. \url{bit.ly/2zZaBUQ}

\subsection{Avaliações de professor*s}

Achou que a professora ou professor ensinou muito mal? Falou da vida, do
universo e tudo mais -- menos sobre a disciplina? Foi incoerente? Ou, pelo
contrário, achou a professora ou professor incrível e a sala do CB a oitava
maravilha do mundo? Não adianta xingar nem elogiar no Twitter!

Nas últimas aulas de cada semestre, todas as professoras e professores devem
disponibilizar um formulário de avaliação. Esse é o momento para que você possa
separar os acertos dos erros, portanto preencha com seriedade. Os dados serão
analisados pelas Comissões de Graduação de cada unidade e os comentários
escritos serão repassados para a professora ou professor.

O IC também faz seu próprio formulário de avaliação da Graduação. Completamente
online, esse formulário é preenchido clicando-se em links disparados para seu
e-mail institucional. O CACo acompanha o processo de envio dos emails para
assegurar todos de que não é associar a aluna ou aluno à avaliação que ele
preencheu.

Além dos formulários, a PRG (Pró-reitoria de Graduação) realiza o Programa de
Avaliação da Graduação no fim de cada semestre. Trata-se de uma pesquisa
online semelhante aos formulários de cada unidade, porém unificada para toda a
Unicamp e mais abrangente em suas perguntas.

O GDE (\url{gde.ir}) também tem um sistema de avaliação de professores, cuja
nota costuma ser usada pelas alunas ou alunos como um dos critérios no momento
de decidir com que professor puxar uma matéria, mas prefira o feedback de
veteranas e veteranos.

Durante o semestre, ocorre a Reunião de Avaliação de Curso. A data e o horário
serão divulgados pelas unidades e pelo CACo. Essa é uma oportunidade de passar
para as coordenadorias do curso não suas impressões sobre professoras,
professores e disciplinas, mas sobre qualquer assunto relacionado ao curso.
Antes da reunião, o CACo também promove um PipoCACo de Avaliação de Curso,
representanto quem não poderá comparecer à reunião.

Tenha sempre em mente que a nossa percepção sobre o oferecimento de uma
disciplina não é óbvia para as professoras e professores. Preencha todas as
avaliações com sinceridade, use sempre os espaços dedicados a comentários com
críticas construtivas. Além disso, cultive o hábito de realizar uma avaliação
informal da professora ou professor no fim de cada semestre -- mande um e-mail,
por exemplo. O valor deste tipo de avaliação é muito grande.
