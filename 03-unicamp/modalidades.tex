% Este arquivo .tex será incluído no arquivo .tex principal. Não é preciso
% declarar nenhum cabeçalho

\section{Modalidades de engenharia da computação}

Às engenheiras e aos engenheiros, esta seção serve para dar uma breve
explicação sobre as duas modalidades de graduação em engenharia de computação.
Mas, primeiro, {\emph{``o que é uma modalidade?''}} Uma modalidade é uma
subdivisão do curso de engenharia, um enfoque específico da sua formação. São
catálogos alternativos que visam distribuir diferentes disciplinas para a mesma
formação, ou seja, possui ênfases em áreas diferentes para se formar em
engenharia de computação. Há duas modalidades: AA, sistemas de Computação, e
AB, Sistemas e Processos Industriais.

Uma das características da Unicamp é dar ao aluno uma formação \textbf{MUITO}
generalista, então todas as alunas e alunos, independente da modalidade -- ou
sendo até da ciência --, estarão capacitados a atuar em qualquer área da
computação. Assim, a escolha da modalidade é apenas um modo de dar à aluna e ao
aluno a oportunidade de se aprofundar em assuntos que lhe interessem. É
importante ressaltar que ambas as modalidades têm uma grande parte de
disciplinas em comum, então a formação básica é essencialmente a mesma. Perto
do período de escolha, o CACo lhe dará mais detalhes no conjunto de palestras
sobre as modalidades AA e AB que são realizadas no segundo semestre para sanar
todas as suas dúvidas e auxiliá-la.

Observação: não escolha sua modalidade por conta do IC ou da FEEC ficarem mais
próximas da sua casa, por favor{\dots}

\subsection{Modalidade AA: Sistemas de Computação}

Também conhecida como Azóide, é a que mais se assemelha à ciência da computação
por focar mais nas áreas de análise e projeto de algorítmos, matemática
discreta e arquitetura de computadores. As aulas serão ministradas
majoritariamente pelo IC.

\subsection{Modalidade AB: Sistemas e Processos Industriais}

Também conhecida como Bzóide, um pouco mais próxima da engenharia elétrica por
conta da conexão com análise de sinais, sistemas autônomos, embarcados. As
aulas serão oferecidas majoritariamente na FEEC.\\ % pulamos mais uma linha ;)

{\emph{``E como mudo de modalidade?''}} A não ser que você tenha concluído 110
créditos dentre as matérias da modalidade desejada, só mudará de modalidade no
quarto semestre, com um pedido na DAC durante o período de matrícula ou
alteração de matrícula.

Ah, como a Unicamp te dá a oportunidade de escolher livremente suas matérias,
você pode mesclar matérias tanto oferecidas para Azóides quanto para Bzóides
verificando com veteranas e veteranos como foram os oferecimentos de cada lado
e escolhendo o que mais lhe agrada, mas tome cuidado com a questão da
equivalência de disciplinas: ás vezes alguma matéria Azóide não completa a
matéria Bzóide e vice-versa!
