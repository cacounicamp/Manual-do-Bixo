% Este arquivo .tex será incluído no arquivo .tex principal. Não é preciso
% declarar nenhum cabeçalho

\section{Agora eu tenho um e-mail da Unicamp}

Ao ingressar num curso de computação, você recebe pelo menos três contas de
e-mail: do IC, da DAC e do Google Apps for Education. Os dois primeiros são os
principais meios de comunicação da universidade com você, portanto fique
esperta e não deixe de ler esses e-mails regularmente!

Para acessar o webmail do IC, o endereço é
\\\urls{https://webmail2.students.ic.unicamp.br}. O login e a senha são os
mesmos do sistema Linux do IC, que você receberá nas primeiras semanas de
aula de laboratório. Caso tenha dúvidas, dê uma passada na Secretaria de
Graduação, que fica no IC-2 (prédio ao lado das Artes Cênicas e para cima da
Economia) e pergunte!

Uma dica interessante, que muitas vezes passa despercebida, é que no mesmo
documento em que você recebe sua senha, vem indicado um \textit{alias} para o
seu e-mail. Assim, você poderá utlizá-lo de uma forma mais amigável, ao invés
de ser somente o seu próprio RA, por exemplo:

\begin{center}
\texttt{francisco.silva@students.ic.unicamp.br}\\
ao invés de\\
\texttt{ra129873@students.ic.unicamp.br}
\end{center}

O webmail da DAC é a primeira letra do seu nome, seguido dos dígitos do RA e do
sufixo:
\begin{center}
\texttt{dac.unicamp.br}
\end{center}
O e-mail da DAC também é útil. É nele que você será avisada sobre o período
de matrícula, pedidos de matrículas provisórias, avisos de desistências e
trancamentos, além de te informar a respeito de eventos que acontecem na
Unicamp, como feiras, palestras, festivais, eleições. O endereço do webmail da
DAC é \url{webmail.dac.unicamp.br}.

Outro e-mail que também está disponível para as alunas e alunos é uma conta no
Gmail, devido a uma parceria entre a Unicamp e a Google! Mais informações sobre
como acessar e o que ganhamos com isso na seção de servicos da Unicamp, no
final do manual.

Para a galera da engenharia, ainda há o da FEEC, que muitos sequer ficam
sabendo que existe! Só após um semestre ou até um ano depois vão até o SIFEEC
retirar seu login e senha. Ou então ficam sabendo só no segundo ou terceiro ano
de curso e aí há mais de 700 e-mails não lidos. Assim como o do IC, é muito
utilizado para divulgação de eventos, oportunidades de estágios e de iniciação
científica. Para utilizá-lo, você deve ir até a FEEC e procurar o SIFEEC, que é
o local responsável por isso. Fica no segundo andar do prédio de laboratórios
da FEEC (um com escadas amarelas). O endereço do webmail da FEEC é
\url{webmail.fee.unicamp.br} e a senha é a mesma do Linux da FEEC.

\begin{figure}[b!]
  \centering
  \includegraphics[width=.3\textwidth]{img/alem_da_graduacao/email.jpg}
\end{figure}

Você também pode redirecionar os e-mails que receber nas contas do IC, FEEC e
DAC para qualquer outra conta (no Gmail, por exemplo).

\subsection{Redirecionamento de e-mails}
Para efetuar o redirecionamento do e-mail institucional para outro e-mail, siga
os passos abaixo para cada e-mail institucional que você tiver.

\subsubsection{E-mail da DAC}

\begin{compactenumerate}
\item Acesse \url{www.dac.unicamp.br}
\item Acesse Estudantes
\item Acesse E-mail e ferramentas Google
\item Faça login
\item Clique no ícone de engrenagem na parte superior direita
\item Clique em Configurações
\item Clique em Encaminhamento e POP/IMAP
\item Clique em Adicionar um endereço de encaminhamento
\end{compactenumerate}

\subsubsection{E-mail da FEEC}

\begin{compactenumerate}
\item Acesse \url{webmail.fee.unicamp.br}
\item Faça login
\item Acesse Options
\item Acesse Mail Forwarding
\end{compactenumerate}

\subsubsection{E-mail do IC}

O IC disponibiliza 2 sistemas de e-mail, disponibilizados em 2 endereços. Para
facilitar sua vida, utilize este primeiro, que é o mais atualizado:

\begin{compactenumerate}
\item \urls{https://webmail2.students.ic.unicamp.br}
\item Faça login
\item Na seção Filtros, clique em Encaminhar
\item Coloque seu endereço de email no campo e grave as mudanças.
\end{compactenumerate}

\begin{compactenumerate}
\item \urls{https://webmail.students.ic.unicamp.br}
\item Faça login
\item Na seção Filtros, adicionar nova regra
\item Mude a condição para ``All''
\item Mude ação para redirecionar
\item Coloque seu endereço de email no campo e aplique as mudanças.
\end{compactenumerate}

\subsection{Suporte técnico}

Além dos serviços de e-mails, ao entrar na Unicamp, você também ganha acesso a
diversos outros serviços, como a VPN da Unicamp, acesso remoto ao IC etc.

Obviamente configurar essas coisas não é trivial, o IC mantem uma página com
instruções de como usar os seus serviços comuns, como configurar seu cliente de
email para acessar o servidor do IC e acessar o IC remotamente.

Sempre que precisar de ajuda no IC, confira se não há uma resposta para seu
problema nesta página:\\
\url{suporte.ic.unicamp.br/index.php/Alunos}

% TODO: Encontrar informações sobre o suporte da FEEC e da DAC.

\subsection{Tenha uma conta no Gmail!}

Nós sabemos que é difícil. Muitas vezes você vem usando um serviço de e-mail
durante anos, seja ele Hotmail, Yahoo, Uol, Terra, Zipmail, e mudar é
trabalhoso. Mas confie na gente: agora vai ser mais fácil. Você está começando
uma vida nova na universidade, quando já tiver cadastrado seu endereço antigo
em vários serviços acadêmicos e divulgado entre os novos contatos que vai
fazer, será bem mais complicado.

Entre as vantagens do Gmail, podemos destacar:

\begin{compactitemize}
\item Muito espaço, nunca mais apague nada;
\item visualização em threads, útil para acompanhar discussões;
\item bom aplicativo para smartphones;
\item filtros e marcadores infinitos para te ajudar;
\item uma pesquisa que funciona;
\item interface mais polida que as da concorrência.
\end{compactitemize}

\emph{Até a Unicamp} está tentando te obrigar a usar Gmail, como dissemos!
\shrug

Qualquer dúvida, novamente, procure uma veterana ou veterano!

Finalizando, não deixe de estar sempre informado sobre os acontecimentos ou
divulgações da Unicamp, do CACo, da AAACEC e da Conpec, essas três entidades
compostas por alunas e alunos de computação.

O CACo mantém uma lista de email para discussão geral de coisas relacionadas à
computação e à Unicamp em \url{bit.ly/cacounicamp}.
