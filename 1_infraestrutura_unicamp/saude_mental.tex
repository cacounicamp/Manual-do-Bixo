\section{Cuidando da Saúde Mental}

Vamos ser sinceres contigo: na faculdade, a gente tem muita coisa boa e muita coisa
ruim também. Pronto, falamos. Nada é perfeito, né? Pela relevância que o curso toma nas
nossas vidas, isso tem um peso que pode acarretar vários efeitos psicológicos na gente.

Levantamentos demonstram altos índices de estresse, ansiedade e depressão na pós-graduação. Calma, você está lendo o manual certo: é que muitos dos fatores que tornam o ambiente acadêmico insalubre na pós também estão presentes na graduação.

Por isso, o melhor que temos a fazer é entender isso e saber o que fazer pra encarar esse problema sem sofrer demais com ele.

\subsection{Estudar dá (ou é) trabalho}

Você pode ter ouvido - e até mencionamos aqui no manual - que a universidade tem muito rolê. Todo fim de semana vai ter festa em algum canto de Barão Geraldo, mas essa é apenas uma pequena parte de nossa vida. Primeiro porque a gente estuda, e MUITO. 

A carga de cobrança é imensa e, muitas vezes, injusta. Infelizmente, a maioria de nosses professores nas Exatas não têm formação em licenciatura ou pedagogia, o que es deixa sem ferramentas de didática essenciais. Por isso, é grande o risco de a gente acabar com a sensação de falta de tempo, ansiedade, dificuldade nos relacionamentos e até mesmo sentindo culpa quando descansamos ou nos divertimos (aquele "nossa, eu devia estar estudando"). 

Tudo isso pesa ainda mais quando lembramos do quanto a nossa cultura negligencia a saúde mental. Até porque o acesso universal a psicólogues infelizmente não está disponível no sistema de saúde gratuito e, ainda que tentemos optar por pagar, eles são caros.

Mas e aí, o que fazer? Você deve ter ouvido falar nas temidas provas de cálculo ou física, mas não esqueça: estudantes unides jamais serão vencides! Encontre um grupinho de estudos, participe das monitorias e converse com es professores sobre a matéria quando der, a única coisa é: NÃO DESISTA! Você lutou muito para chegar até aqui, não vai ser uma matéria que vai te derrubar, né?


\subsection{You Are (Not) Alone}

É no espaço da universidade que estamos submetidos a essa infinidade de cobranças, a conflitos para os quais não estamos preparados e, como em todo espaço em que temos que enfrentar a hostilidade do mundo, você pode se sentir só. No entanto, é também nesse espaço que podemos fazer amizades, encontrar companheires que nos ajudam nos momentos mais difíceis. Então, não se esqueça que a faculdade não é e jamais deveria ser uma competição. Você tem como ajudar sues colegues e tem onde buscar apoio quando precisar, como colegues e veteranes que passaram ou passam pelas mesmas coisas que você. Aja com franqueza e sinceridade para consigo mesmo, busque entender o que acontece com você e se comunique, sem medo de pedir ajuda ou desabafar. Ter dificuldades não te faz melhor ou pior e o \textbf{seu rendimento acadêmico não significa tudo na vida}. Nosso sistema de educação é terrivelmente falho e não deve ditar o que uma pessoa é ou deixa de ser. Em suma, bitolar não é saudável!

Por isso, não cobre muito de si e não se negligencie.  Busque o que te fortalece, conte com a comunidade, converse com o CACo, as demais entidades e veternes, apresente ideias, se abra ou mesmo denuncie aquilo que te faz mal. Juntes podemos melhorar a nós e nosso entorno. Se for preciso, estamos 100\% dispostes a lutar para que você tenha a melhor experiência possível aqui! Por fim, nesse mundo que individualiza tudo, lembre-se que a saúde mental é uma questão coletiva.

\subsection{SAPPE - Serviço de Apoio Psicológico e Psiquiátrico}

A universidade oferece serviços de assistência psicológica e psiquiátrica gratuitamente através do SAPPE, órgão ligado à Pró-Reitoria de Graduação (PRG). Ele oferece
algumas modalidades de atendimento, divididas em duas principais categorias: o
atendimento regular e o pronto atendimento.

O atendimento regular é um tratamento mais contínuo que é feito geralmente em
quatro sessões, a menos que precise ser aumentado, de acordo com a opinião
profissional. Para utilizar o serviço você deve preencher uma Ficha de Cadastro para
atendimento regular de acordo com os horários disponíveis no site do SAPPE (disponível
abaixo) na recepção do Serviço (endereço e instruções de chegada abaixo) e agendar a
participação no Grupo de Recepção. Após a participação no grupo, você pode agendar
uma Entrevista de Triagem que te encaminhará para o tratamento adequado.

\textbf{Observação}: o Grupo de Recepção é uma palestra que explica o funcionamento do
serviço, mas ela parece ser montada para te desmotivar a usá-lo. Os períodos de espera
entre cada etapa também podem ser bem longos e servem para esse mesmo fim. Então,
se você realmente sente a necessidade de buscar esta orientação, \textbf{não desista}.

O SAPPE também tem pronto atendimento. Essa é uma sessão única voltada para
quando você estiver passando por uma crise ou emergência, mas com os horários
limitadíssimos. Nesse caso, você deve passar na sede do SAPPE nos dias e horários
disponíveis no site.

\textbf{Localização}: R. Sérgio B. de Holanda, 251, 1o piso Em frente ao prédio da DAC; mesmo
prédio do SAE e CLE, ao lado do Bandejão.

%\textbf{Horário}: de segunda a sexta, das 8h30 às 20h, no período letivo.

\begin{tags}
    \tel{19 3521-6643} ~~~ou~~~ \tel{19 3521-6644} \\
    \facebook{SAPPE-Unicamp-112906617163477} \sep \email{sappeass@unicamp.br}
\end{tags}

\begin{figure}
    \centering
    \includegraphics[width=0.7\textwidth]{imagens/sappe.png}
\end{figure}