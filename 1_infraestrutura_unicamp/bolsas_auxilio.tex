\section{Bolsas-Auxílio e Inscrição}



\begin{multicols}{2}
[Todas as bolsas listadas abaixo são centralizadas pelo SAE. Você tem um período de cerca de uma semana para se inscrever no edital de seleção, então procure as
informações pertinentes no site do DEAPE, fique atente ao seu e-mail institucional e não perca o prazo!]

\textbf{BITA:}  Benefício de Isenção da Taxa de Alimentação, consiste no acesso gratuito aos restaurantes universitários para o café da manhã, almoço e jantar. \\

\textbf{BAS:} Bolsa Auxílio-Social, na qual o aluno participa de projetos dentro de diversas áreas da Universidade, sempre com a orientação de profissionais nas áreas de competência, professores das unidades da Unicamp, coordenadores e outros profissionais. A carga horária é de 10 horas semanais e o aluno recebe mensalmente uma bolsa.\\

\textbf{BAT:} Bolsa de Auxílio Transporte, o aluno contemplado recebe uma bolsa mensal com valor vigente em Campinas de dois passes municipais por dia útil do mês em questão. \\

\textbf{BAEF:} Bolsa Estudo Formação,  destinada a alunos de graduação que estão com 75\% do curso concluído, tem como objetivo a participação em projetos voltados para o seu curso. Estes projetos são cadastrados por docentes e técnicos administrativos que orientam os bolsistas. A carga horária é de 20 horas semanais, também recebendo mensalmente uma bolsa. \\

\textbf{BAI:} Bolsa Instalação, destinada a es alunes ingressante que encontrem dificuldades financeiras no início da graduação. E alune contemplado com a bolsa recebe um pagamento único. \\

\textbf{PME:} Programa de Moradia Estudantil, trata-se de fornecimento de moradia gratuita para alunes com dificuldades em se manter em uma residência com recursos próprios. Os prédios da moradia estudantil da Unicamp ficam na Av. Santa Isabel, 1125. Como já mencionado nesse manual, há transporte gratuito da moradia para o campus. Para mais informações entre no site: 
 \site{https://deape.unicamp.br/permanencia/moradia-estudantil/sobre}. \\

\textbf{BE:} Bolsa Emergência, atende alunes que passam por dificuldades econômicas emergenciais, portanto não possui um processo seletivo convencional através de editais e deve ser solicitada no Sistema Integrado de Gestão \url{https://sistemas.sae.unicamp.br/sig/}. \\

E alune contemplado com a bolsa deverá cumprir de 10 a 40 horas semanais em algum projeto social e recebe um único pagamento. \\
\vspace{1em}
\hrule
\vspace{1em}
Há ainda outras bolsas fornecidas pelo SAE como bolsa PAPI e a Aluno Artista, sobre as quais você pode encontrar mais informações no site. E claro, além dos auxílios de permanência citados aqui, há ainda as bolsas acadêmicas como a PIBIC, voltada para a Iniciação Científica. Sobre essas, entramos em maiores detalhes na seção Vida Acadêmica.


\end{multicols}

\begin{table}[H]
    \centering
    \rowcolors{2}{gray!10}{white}
    %%% Increase spacing
    \bgroup
    \def\arraystretch{1.5}
    \begin{tabular}{|c|c|}
        \hline
        Bolsa & Valor (R\$) \\
        \hline
        BAS & 857,00 + valor de 2 passes de transporte público/dia\\
        \hline
        BAI & 444,00 \\
        \hline
        BAEF & 1142,00 + valor de 2 passes de transporte público/dia\\
        \hline
        BAM & 688,00 \\
        \hline
        BAT & valor de 2 passes de transporte público/dia\\
        \hline
        BE & Valor BAS\\
        \hline
    \end{tabular}
    \egroup
    \caption{Valores de Bolsas para o ano de 2025}
    \label{tab:my_label}
\end{table}