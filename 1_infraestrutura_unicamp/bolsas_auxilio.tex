\section{Bolsas-Auxílio e Inscrição}



\begin{multicols}{2}
[Todas as bolsas listadas abaixo são centralizadas pelo SAE. Você tem um período
de cerca de uma semana para se inscrever no edital de seleção, então procure as
informações pertinentes no site do SAE, fique atente ao seu e-mail institucional e
não perca o prazo!Todas as bolsas listadas abaixo são centralizadas pelo SAE. Você tem um período
de cerca de uma semana para se inscrever no edital de seleção, então procure as
informações pertinentes no site do SAE, fique atente ao seu e-mail institucional e
não perca o prazo!]

\textbf{BITA:}  Benefício de Isenção da Taxa de Alimentação, consiste no acesso franque-ado aos restaurantes universitários para o café da manhã, almoço e jantar. \\

\textbf{BAS:} Bolsa Auxílio-Social, na qual o aluno atividades em projetos dentro de diversas áreas da Universidade, sempre com a orientação de profissionais nas áreas de competência, professores das unidades da Unicamp, coordenadores e outros profissionais. A carga horária é de 40 horas mensais e o aluno recebe mensalmente uma bolsa que em 2022 era de R\$ 747,10. \\

\textbf{BAT:} Bolsa de Auxílio Transporte,  o aluno contemplado recebe uma bolsa mensal com valor vigente em campinas de dois passes municipais por dia útil do mês em questão. \\

\textbf{BAEF:} Bolsa Estudo Formação,  destinada a alunos de graduação que estão com 75\% do curso concluído, tem como objetivo a participação em projetos voltados para o seu curso. Estes projetos são cadastrados por docentes e técnicos administrativos que orientam os bolsistas. A carga horária é de 20 horas semanais e o aluno recebe mensalmente uma bolsa que em 2022 era de R\$ 996,13. \\

\textbf{BAI:} Bolsa Instalação, destinada ao aluno ingressante que encontra dificuldades financeiras no início da graduação. O aluno contemplado com a bolsa recebe um pagamento único no valor da bolsa que em 2022 era de R\$ 387,47. \\

\textbf{PME:} Programa de Moradia Estudantil, trata-se de fornecimento de moradia gratuita para alunos com dificuldades em se manter em uma residência com recursos próprios. Os prédios da moradia estudantil da Unicamp ficam na Av. Santa Isabel numero 1125. Como já menciona-do nesse manual, há transporte gratuito da moradia pra o campus. Para mais informações entre no site: 
 www.pme.unicamp.br. \\

\textbf{Bolsa Emergência:} atende alunos que passam por dificuldades econômicas emergenciais, portanto não possui um processo seletivo convencional através de editais e deve ser solicitada nos Sistema Integrado de Gestão (https://sistemas.sae.unicamp.br/sig/). \\

O aluno contemplado com a bolsa  deverá cumprir de 10 a 40 horas semanais em algum projeto social e recebe um único pagamento no valor da bolsa que em 2022 era de R\$ 747,10. \\
\vspace{1em}
\hrule
\vspace{1em}
Há ainda outras bolsas fornecidas pelo SAE como bolsa PAPI e a Aluno Artista, sobre as quais você pode encontrar mais informações no site. E claro, além dos auxílios de permanência citados aqui, há ainda as bolsas acadêmicas como a PIBIC, voltada para a Iniciação Científica. Sobre essas, entramos em maiores detalhes na  na seção Vida Acadêmica. 



\end{multicols}