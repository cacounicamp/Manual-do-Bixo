\section{Hemocentro}

Essa é para quem é ou para quem quer ser doadora ou doador de sangue. O centro de
hematologia e hemoterapia (Hemocentro) é o órgão da Unicamp responsável pela coleta
e doação de sangue.

Para ser doadore voluntárie, é necessário portar um documento com foto e seguir os critérios para a doação de sangue. É realizada uma entrevista que tem como objetivo
caracterizar riscos para e receptore e verificar contra-indicações a doação. Para saber
mais sobre o processo, é só visitar o site: \url{www.hemocentro.unicamp.br}.

O setor de Hemoterapia desenvolve campanhas permanentes de conscientização
sobre a doação voluntária de sangue em municípios da região, visando o engajamento
de toda a comunidade em seu trabalho e na manutenção de estoques seguros para o
atendimento das necessidades transfusionais dos pacientes.

O Hemocentro fica localizado acima do HC, próximo ao CECOM. Portanto, se quiser se deslocar até lá, usar o circular interno é uma boa opção.
