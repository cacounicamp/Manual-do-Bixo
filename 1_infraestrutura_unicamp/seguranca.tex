\section{Mantenha-se em segurança} \label{seguranca}

\subsection{Campus Tranquilo}

O Campus Tranquilo era um programa de segurança do campus implementado após
formulação com a comunidade acadêmica e profissionais de segurança pública como
alternativa, em 2013, à tentativa da Polícia Militar de estabelecer presença permanente
no campus. O argumento é que a polícia pode cumprir um papel truculento e repressivo
- sobretudo em protestos contra cortes de verbas. Afinal, mais de uma vez, em várias
universidades, como na USP, funcionários e estudantes tomaram bomba de gás e foram
detidos pela PM diante de votações importantes do Conselho Universitário.

Após uma decisão judicial, funcionários da FUNCAMP treinados no programa foram
demitidos em massa e a universidade re-terceirizou o serviço. Essa força de trabalho
agora têm condições mais precárias e muito menos treinamento. O resultado é que a
guarda patrimonial se tornou mais policialesca, com casos de abordagens agressivas.

O patrulhamento patrimonial continua mas a maioria des guardinhas se apresenta
amigavelmente, sendo possível pedir escolta durante a noite. Também há uma
ambulância em caso de emergências de saúde.

\subsection{Botão de pânico}

O Botão do Pânico é um app disponibilizado pela própria Unicamp, que tem como
objetivo registrar situações de pânico dentro do campus. O app está disponível tanto na
Play Store quanto na App Store, sendo que os credenciais para login são seu RA e a sua
senha da DAC. O ideal é já fazer o download do aplicativo e fazer o login, pois numa
situação de risco não haverá tempo para isso. O aplicativo utiliza sua localização para
pedir emergencialmente a presença de algum responsável da segurança da Unicamp.

\subsection{Grupo da Unicamp no Facebook}

O grupo da Unicamp no Facebook tem vários objetivos, como integração e diversão,
no entanto também pode ser utilizado para divulgação de informações. De vez em
quando rolam no grupo algumas publicações sobre uma situação de risco que ocorreu
em Barão Geraldo, explicando o que aconteceu e onde especificamente aconteceu. Por
isso, fique de olho!

\subsection{Página Avisa as Minas}

As dependências de Barão Geraldo também são perigosas para mulheres. Para não só
evitar, mas também combater situações de assédios verbal e físico, perseguição e
ameaças, foi criada a página "Avisa as Minas de Barão/Unicamp" (tagueado por \#AMBU),
para deixar todo e qualquer tipo de aviso que você ache importante sobre essas
questões de segurança citadas. Nela, é possível postar alertas o mais rápido possível,
então lembre sempre de ativar para receber as notificações e mandar nos grupos de sala
para que cada vez mais meninas sejam alertadas.