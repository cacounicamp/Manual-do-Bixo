\section{Moradia: luta e disputa}

{\footnotesize \color{gray} Texto adaptado, atualizado e ampliado do site do Cêntro Acadêmico do IEL (CAL).}

Em 1986, foi organizado o movimento estudantil TABA: sem ter onde morar, 60
estudantes ocuparam salas do CB por mais de dois anos, reivindicando uma moradia
gratuita aos estudantes de baixa-renda. Essa ocupação estudantil ficou conhecida como
"A Taba", em referência a tabas indígenas que são caracterizadas por uma cultura de
coletividade.

O reitor da época, Paulo Renato Costa Souza, prometeu que 1.500 vagas –
correspondente a cerca de 10\% do total de estudantes daquela época – seriam
entregues até o mês de julho de 1989 e, no caso de não cumprimento das cláusulas que
garantiam a construção da moradia, ele assinou um documento que autorizava o DCE
Unicamp a ocupar novamente o campus, assim como o Movimento TABA havia feito. Em
1990, a Moradia foi inaugurada com novecentas e quatro vagas, já abaixo do prometido.

Na greve de 2016, os estudantes reivindicaram a expansão das vagas e conseguiram
que o reitor José Tadeu Jorge assinasse um documento que a garantisse. Com o fim da
greve, criaram-se grupos de trabalho (GTs) para analisar e avaliar a ampliação e os
programas de permanência estudantil da universidade. Eventualmente, os GTs foram
dissolvidos e o terreno, que estava em processo de compra, ficou embargado em
cartórios. Finalmente, membros da administração central da universidade admitiram em
reunião com membros do movimento estudantil que a nova moradia "não vai sair".

Além disso, há diversos problemas na infraestrutura que indicam uma falta de
manutenção generalizada, como por exemplo estruturas que foram evacuadas sob
perigo de desmoronamento; problemas elétricos, de drenagem nas ruas e de infiltração
de água nas casas; dificuldades na separação do lixo reciclável; e vazamentos. Há
muitos casos de superlotação, com casas abrigando até sete estudantes, e vários casos
de furto, principalmente de bicicletas dos moradores.

Uma moradia estudantil digna é um espaço determinante para a garantia da
democracia no acesso e permanência no Ensino Superior - incluindo os campi que
sequer têm programa de moradia estudantil, em Piracicaba e em Limeira.

Por fim, há uma disputa de projeto sobre a moradia: ao invés de construir mais casas,
a reitoria tem "investido" na Bolsa Moradia - um valor de cerca de R\$ 400 pago a
estudantes para se manterem através de contratos de aluguel. O impacto na organização
estudantil é enorme, pois es bolsistas estão pulverizades e não têm um espaço para se
encontrar e auto-organizar, caso precisem defender suas pautas.

Os milhões de reais por ano vão pelos ares em aluguéis que em pouco tempo poderiam arcar com a construção de uma moradia inteira. Crises financeiras cada vez mais constantes - lembre-se que ainda estamos vivendo consequências da "crise de 2008" -
podem pôr em xeque essas bolsas, o que não ocorreria numa moradia construída. O tiro
cego da reitoria acerta, contudo, o mesmo alvo: a especulação imobiliária de Barão, que
só tem a agradecer.

\vspace{1em}
\textbf{O que você não viu:}
\vspace{-0.5em}
\begin{itemize}
    \setlength\itemsep{-0.2em}
    \item \textbf{1986}: Sem ter onde morar, estudantes estabelecem o movimento TABA e acampam no CB até a conclusão da construção da moradia.
    \item \textbf{2016}: Greve estudantil é lançada e dura 5 meses, ocupando a reitoria.
    \item \textbf{2017}: Longa sessão do CONSU aprova cotas étnico-raciais e sociais, diante de um ato com mais de 600 pessoas.
    \item \textbf{2019}: Es primeires estudantes cotistas entram na universidade. Estudantes organizam uma rede de apoio solidário para viabilizar que indígenas e cotistas tenham condições suficientes para se estabelecer nas primeiras semanas, o que ocorreu também em 2020.
    \item \textbf{2023}: Durante a greve de 2023 que ocupou o IMECC, estudantes conquistaram o compromisso da reitoria para a expansão da moradia e transporte para a universidade no fim de semana.
\end{itemize}

\begin{figure}[htbp]
    \centering

    \includegraphics[width=0.45\textwidth]{imagens/ocupacao_moradia.jpg}
    \hfill
    \includegraphics[width=0.45\textwidth]{imagens/vista_moradia.jpg}

    \caption{1. Ocupação da moradia em 2011 exige a construção de 3000 vagas para cobrir o déficit ; 2. Vista aérea da moradia}
\end{figure}