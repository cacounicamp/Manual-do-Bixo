\section{A palavra é permanência}

A ideia é simples: agora que você entrou na universidade, não vá embora tão cedo e
sem o diploma. Você precisa permanecer. Isso envolve uma porção de fatores e
necessidades diferentes para cada estudante.

Ter o que comer e onde morar: permanência. Ter estabilidade emocional e saúde
mental: permanência. Ter como se locomover até a universidade e outros locais de
estudo: permanência. Ter acesso aos livros, conteúdos, softwares, computadores,
equipamentos, sistema de saúde: tudo isso é permanência.

Qualquer um desses pilares, se cair, pode te deixar numa situação extremamente
vulnerável e até mesmo forçar a sua evasão da graduação. Afinal de contas, todos esses
fatores impactam diretamente no seu desempenho acadêmico e na sua capacidade de
sobreviver enquanto cumpre o currículo. Mas como garantimos a nossa permanência?

\subsection{Dois modelos de universidade}

Um dos jeitos é o "se vira". Para estudantes com famílias mais bem estruturadas
economicamente, é razoavelmente simples garantir tudo isso: pais pagam o aluguel; o
bandejão tem um preço confortável mas você nem precisa dele; pode pagar os próprios
livros; você tem seu computador. E quem não tiver essas coisas garantidas que se vire e
procure um emprego. Parece coisa de universidade particular, não é? Um dos lugares
onde a política do "se vira" é a norma são países em que o Ensino Superior é todo
privatizado, como os EUA, onde os estudantes já chegaram a acumular US\$ 1.5 trilhão
em dívidas para pagar despesas com a faculdade\footnote{\textit{Student Loan Debt Statistics In 2019: A \$1.5 Trillion Crisis. Forbes, 2019.}}. A problemática é tão grave que há
estudantes morando dentro de carros para não perder a vaga em universidades
prestigiadas\footnote{\textit{Number of Homeless Students Soars. US News, 2019.}} – o que para alguns pode ser visto como mais uma história de
superação, é uma realidade dura e injusta para milhares de estudantes.

Colocando isso em perspectiva fica claro como a universidade pública
democratiza o conhecimento. Ainda assim, ela não garante automaticamente a permanência de todes es seus alunes. Não muito tempo atrás, mesmo que você conseguisse entrar em uma universidade pública como a Unicamp, apesar de não ter que pagar mensalidade, ainda ia ser bem difícil se virar pra permanecer por aqui.

Assim, o outro jeito de garantir permanência é que a sociedade como um todo arque com os custos do cumprimento do direito constitucional à educação dos estudantes através da realização de Políticas de Permanência. Mesmo que o investimento seja alto, o retorno trazido por ume profissional de ponta à sociedade mais do que paga esse investimento.

É importante pontuar que as Políticas de Permanência que temos hoje tiveram de ser conquistadas. Estudantes lutaram pesado, bravamente, para que hoje tenhamos direito a Moradia Estudantil, ao Bandeijão, as bolsas de auxilio social, entre outras Políticas de Permanência. Além delas, também foram conquistadas Políticas Afirmativas como a bonificação para alunos de escolas públicas no vestibular, cotas étnico-raciais e o vestibular indígena, que fazem com que a universidade pública incorpore cada vez mais a nossa sociedade. Como você vai ver no próximo texto, “Moradia: luta e disputa”, as coisas ainda estão longe de serem perfeitas, essas lutas não pararam no tempo, ainda temos muito o que conquistar e compartilhando esse conhecimento conseguimos garantir que o que já foi conquistado permaneça em vigência, ajudando a todes.

\begin{figure}[htpb]
    \centering
    \includegraphics[width=0.75\linewidth]{imagens/Placa_unicamp.png}
\end{figure}